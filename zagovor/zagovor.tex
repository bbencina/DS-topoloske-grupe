\documentclass[a4paper, 12pt]{beamer}

\usetheme{CambridgeUS}
\usecolortheme{beaver}
\usefonttheme{structuresmallcapsserif}


\usepackage[slovene]{babel}
\usepackage[utf8]{inputenc}
\usepackage[T1]{fontenc}
\usepackage{lmodern}
\usepackage{units}
\usepackage{eurosym}
\usepackage{amsmath}
\usepackage{amssymb}
\usepackage{amsthm}
\usepackage{amsfonts}
\usepackage{mathtools}
\usepackage{graphicx}
\usepackage{color}
%\usepackage{url}
\usepackage{hyperref}
\usepackage{enumerate}
\usepackage{enumitem}
\usepackage{pifont}

\definecolor{bostonuniversityred}{rgb}{0.8, 0.0, 0.0}

\newenvironment{matematika}[1]{
\textcolor{bostonuniversityred}{\underline{\textsc{#1:}}}
}{
}

\title{Topološke grupe}
\author[Benjamin Benčina]{Avtor: Benjamin Benčina\newline \newline \footnotesize Mentor: Marko Kandić}
\institute[FMF]{Fakulteta za matematiko in fiziko}
\date{\today}

\begin{document}
	
\titlepage

\begin{frame}
\frametitle{Vsebina}
\begin{itemize}[label = \ding{227}]
	\item definicija topološke grupe
	\item primeri
	\item s čim sem se ukvarjal
	\item metrizabilnost in višji separacijski aksiomi
\end{itemize}
\end{frame}

\begin{frame}
\frametitle{Definicija topološke grupe}
\begin{matematika}{Definicija}
	\emph{Topološka grupa} je grupa $(G, *)$ s topologijo $\tau$, glede na katero sta strukturni preslikavi zvezni.
\end{matematika}
\newline
\newline
Strukturni preslikavi:
\begin{itemize}[label=\ding{227}]
	\item Množenje: $\mu : G \times G \to G$, $(x, y) \mapsto xy$.
	\item Invertiranje: $\iota : G \to G$, $x \mapsto x^{-1}$.
\end{itemize}
\end{frame}

\begin{frame}
\frametitle{Primeri}
\begin{matematika}{Primer}
Grupa $\mathbb{R}^n$ je topološka grupa za vsak $n \in \mathbb{N}$.\newline\newline
\end{matematika}
\begin{matematika}{Primer}
Naj bo $G$ poljubna neskončna grupa in naj bo $\tau$ topologija končnih komplementov na $G$. Topološki prostor $(G, \tau)$ ni topološka grupa za nobeno operacijo.
\end{matematika}
\end{frame}

\begin{frame}
\frametitle{Teme}
\begin{itemize}[label = \ding{227}]
	\item osnovne lastnost
	\item kvocientni prostori topoloških grup
	\item trije izreki o topoloških izomorfizmih
	\item metrizabilnost
	\item višji separacijski aksiomi
\end{itemize}
\end{frame}

\begin{frame}
\frametitle{Okolice}
\begin{matematika}{Trditev}
Za topološko grupo $G$ in bazo $\mathcal{U}$ odprtih okolic enote $e$ veljajo naslednje trditve:
\begin{itemize}
	\item (1) za vsako $U \in \mathcal{U}$ obstaja taka $V \in \mathcal{U}$, da je $V^2 \subset U$,
	\item (2) za vsako $U \in \mathcal{U}$ obstaja taka $V \in \mathcal{U}$, da je $V^{-1} \subset U$,
	\item (3) za vsako $U \in \mathcal{U}$ in vsak $x \in U$ obstaja taka $V \in \mathcal{U}$, da je $xV \subset U$,
	\item (4) za vsako $U \in \mathcal{U}$ in vsak $x \in U$ obstaja taka $V \in \mathcal{U}$, da je $xVx^{1} \subset U$.
\end{itemize}
\end{matematika}
\end{frame}

\begin{frame}
\frametitle{Okolice}
\begin{matematika}{Izrek}
Naj bo $G$ grupa in $\mathcal{U}$ družina podmnožic grupe $G$, za katero veljajo lastnosti (1)-(4). Naj bodo poljubni končni preseki množic iz $\mathcal{U}$ neprazni. Tedaj je družina $\lbrace xU ;\; x \in G,\; U \in \mathcal{U} \rbrace$ odprta podbaza za neko topologijo na $G$. S to topologijo je $G$ topološka grupa. Družina $\lbrace Ux ;\; x \in G,\; U \in \mathcal{U} \rbrace$ je podbaza za isto topologijo.

Če velja še, da za vsaki množici $U, V \in \mathcal{U}$ obstaja taka množica $W \in \mathcal{U}$, da je $W \subset U \cap V$, potem sta družini $\lbrace xU ;\; x \in G,\; U \in \mathcal{U} \rbrace$ in $\lbrace Ux ;\; x \in G,\; U \in \mathcal{U} \rbrace$ tudi bazi za to topologijo.
\end{matematika}
\end{frame}

\begin{frame}
\frametitle{Kvocientni prostori topoloških grup}
\begin{itemize}[label = \ding{227}]
	\item kvocientna topologija
	\item lastnosti kvocientnega prostora topoloških grup
	\item kvocientna topološka grupa
\end{itemize}
\end{frame}

\begin{frame}
\frametitle{Izreki o topoloških izomorfizmih}
\begin{itemize}[label = \ding{227}]
	\item topološki izomorfizem
	\item trije izreki o izomorfizmih
	\item dodatne topološke predpostavke na preslikave in podgrupe
\end{itemize}
\end{frame}

\begin{frame}
\frametitle{Izrek o psevdometriki}
\begin{matematika}{Izrek}
	Naj bo $\{ U_k \}_{k=1}^{\infty}$ družina simetričnih okolic enote $e$ topo\-loš\-ke grupe $G$ z lastnostjo $U_{k+1}^2 \subset U_k$ za vsak $k\in\mathbb{N}$. Potem obstaja taka levoinvariantna psevdometrika $\sigma$, da velja:
	\begin{itemize}[label=\ding{227}]
		\item $\sigma$ je enakomerno zvezna na levi uniformni strukturi na $G \times G$,
		\item $\sigma (x, y) = 0 \iff y^{-1}x \in \bigcap_{k=1}^{\infty}U_k$,
		\item $\sigma (x, y) \leq 2^{-k+2}$, če je $y^{-1}x \in U_k$,
		\item $2^{-k} \leq \sigma (x, y)$, če $y^{-1}x \notin U_k$.
	\end{itemize}
	Če poleg tega velja še, da $x U_k x^{-1} = U_k$ za vse $x \in G$ in $k\in\mathbb{N}$, je $\sigma$ tudi desnoinvariantna in velja:
	\begin{itemize}[label=\ding{227}]
		\item $\sigma (x^{-1}, y^{-1}) = \sigma (x, y)$ za vsaka $x, y \in G$.
	\end{itemize}
\end{matematika}
\end{frame}

\begin{frame}
\frametitle{Metrizabilnost}
\begin{matematika}{Izrek}
Naj bo $G$ topološka grupa, ki zadošča separacijskemu aksiomu $T_0$. Tedaj je $G$ metrizabilen topološki prostor natanko tedaj, ko obstaja števna baza odprtih okolic enote.
\end{matematika}
\end{frame}

\begin{frame}
\frametitle{Povsem regularnost}
\begin{matematika}{Definicija}
Topološki prostor $X$ zadošča separacijskemu aksiomu $T_{3\frac{1}{2}}$, če za vsako zaprto množico $F \subset X$ in točko $a \in X \setminus F$ obstaja zvezna funkcija $\psi: X \to [0,1]$, da je $\psi(a) = 1$ in $\psi|_F = 0$.
\end{matematika}\newline

Topološki prostor je \emph{povsem regularen}, če zadošča separacijskima aksiomoma $T_1$ in $T_{3\frac{1}{2}}$. \newline

\begin{matematika}{Izrek}
Vsaka topološka grupa $G$, ki zadošča separacijskemu aksiomu $T_0$, je povsem regularen topološki prostor.
\end{matematika}
\end{frame}

\begin{frame}
\frametitle{Parakompaktnost}
\begin{matematika}{Definicija}
\begin{itemize}
\item
Naj bo $X$ topološki prostor. Družina podmnožic $\mathcal{V}$ je pofinitev družine podmnožic $\mathcal{U}$, če za vsako množico $V \in \mathcal{V}$ obstaja takšna množica $U \in \mathcal{U}$, da je $V \subset U$.
\item
Topološki prostor $X$ je \emph{parakompakten}, če ima vsako njegovo odprto pokritje kakšno pofinitev, ki je lokalno končno odprto pokritje prostora $X$.\newline
\end{itemize}
\end{matematika}
\begin{matematika}{Izrek}
Vsak parakompakten Hausdorffov topološki prostor je normalen.
\end{matematika}
\end{frame}

\begin{frame}
\frametitle{Normalnost}
\begin{matematika}{Izrek}
Vsaka lokalno kompaktna topološka grupa $G$, ki zadošča separacijskemu aksiomu $T_0$, je parakompakten topološki prostor.
\end{matematika}\newline\newline

\begin{matematika}{Posledica}
Vsaka lokalno kompaktna topološka grupa, ki zadošča separacijskemu aksiomu $T_0$, je normalna.
\end{matematika}
\end{frame}

\begin{frame}
\centering
Hvala za vašo pozornost!
\end{frame}

\end{document}