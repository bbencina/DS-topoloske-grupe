\documentclass[a4paper, 12pt]{beamer}

\usetheme{CambridgeUS}
\usecolortheme{beaver}
\usefonttheme{structuresmallcapsserif}


\usepackage[slovene]{babel}
\usepackage[utf8]{inputenc}
\usepackage[T1]{fontenc}
\usepackage{lmodern}
\usepackage{units}
\usepackage{eurosym}
\usepackage{amsmath}
\usepackage{amssymb}
\usepackage{amsthm}
\usepackage{amsfonts}
\usepackage{mathtools}
\usepackage{graphicx}
\usepackage{color}
%\usepackage{url}
\usepackage{hyperref}
\usepackage{enumerate}
\usepackage{enumitem}
\usepackage{pifont}

\definecolor{bostonuniversityred}{rgb}{0.8, 0.0, 0.0}

\newenvironment{matematika}[1]{
\textcolor{bostonuniversityred}{\underline{\textsc{#1:}}}
}{
}

\title{Topološke grupe}
\author{Benjamin Benčina}
\institute[FMF]{Fakulteta za matematiko in fiziko}
\date{\today}

\begin{document}
	
\titlepage
	
\begin{frame}
\begin{matematika}{Definicija}
	\emph{Topološka grupa} je grupa $(G, *)$ s topologijo $\tau$, glede na katero sta strukturni operaciji zvezni.
\end{matematika}
\newline
\newline
Strukturni operaciji:
\begin{itemize}[label=\ding{227}]
	\item Množenje: $\mu : G \times G \to G$, $(x, y) \mapsto xy$.
	\item Invertiranje: $\iota : G \to G$, $x \mapsto x^{-1}$.
\end{itemize}

\end{frame}

\begin{frame}
\frametitle{Nekaj lastnosti}
\begin{matematika}{Izrek}
Naj bo $G$ topološka grupa in $a \in G$ njen element. Leva translacija $x \mapsto ax$, desna translacija $x \mapsto xa$, invertiranje $x \mapsto x^{-1}$ in konjugiranje $x \mapsto axa^{-1}$ so homeomorfizmi na $G$.
\end{matematika} \newline

\begin{matematika}{Opomba}
Vemo že, da so to avtomorfizmi grupe $G$.
\end{matematika}
\end{frame}

\begin{frame}
\frametitle{Nekaj lastnosti}
\begin{matematika}{Izrek}
Naj bo $G$ topološka grupa in $\mathcal{U}$ baza odprtih okolic enote $e$. Veljajo naslednje trditve:\pause
\begin{itemize}[label=\ding{227}]
	\item za vsako okolico $U \in \mathcal{U}$ obstaja taka okolica $V \in \mathcal{U}$, da velja $V^2 \subset U$,\pause
	\item za vsako okolico $U \in \mathcal{U}$ obstaja taka okolica $V \in \mathcal{U}$, da velja $V^{-1} \subset U$,\pause
	\item za vsako okolico $U \in \mathcal{U}$ in element $x \in U$ obstaja taka okolica $V \in \mathcal{U}$, da velja $xV \subset U$,\pause
	\item za vsako okolico $U \in \mathcal{U}$ in element $x \in U$ obstaja taka okolica $V \in \mathcal{U}$, da velja $xVx^{-1} \subset U$.
\end{itemize}
\end{matematika}
\end{frame}

\begin{frame}
\frametitle{Nekaj lastnosti}
\begin{matematika}{Trditev}
Vsaka topološka grupa ima odprto bazo okolic enote $e$, sestavljeno iz simetričnih množic $U = U^{-1}$.
\end{matematika} \newline

\begin{matematika}{Posledica}
Za vsako okolico $U$ enote $e$ topološke grupe $G$, obstaja taka okolica $V$ enote $e$, da velja $\overline{V} \subset U$.
\end{matematika}
\end{frame}

\begin{frame}
\frametitle{Regularnost}
\begin{itemize}[label=\ding{227}]
\item Vemo: $T_2 \implies T_1 \implies T_0$.
\item Za topološke grupe: $T_0 \iff T_2$. \pause
\item Še več:
\end{itemize}

\begin{matematika}{Izrek}
Vsaka topološka grupa, ki zadošča separacijskemu aksiomu $T_0$, je regularen topološki prostor.
\end{matematika}
\end{frame}

\begin{frame}
\frametitle{Metrika in pseudometrika}
\begin{matematika}{Definicija}
	\emph{Pseudometrika} na množici $X$ je funkcija $d: X \times X \to [0,\infty)$, ki zadošča pogojem:
	\begin{itemize}[label=\ding{227}]
		\item $d(x, x) = 0$,
		\item $d(x, y) = d(y, x)$,
		\item $d(x, z) \leq d(x, y) + d(y, z)$.
	\end{itemize}
\end{matematika}
\vspace{12pt}
Pseudometriki do metrike manjka: $d(x, y) = 0 \iff x = y$.
\end{frame}

\begin{frame}
\frametitle{Izrek o pseudometriki}
\begin{matematika}{Izrek}
	Naj bo $\{ U_k \}_{k=1}^{\infty}$ družina simetričnih okolic enote $e$ topo\-loš\-ke grupe $G$ z lastnostjo $U_{k+1}^2 \subset U_k$ za vsak $k\in\mathbb{N}$. Potem obstaja taka levoinvariantna pseudometrika $\sigma$, da velja:
	\begin{itemize}[label=\ding{227}]
		\item $\sigma$ je enakomerno zvezna na levi uniformni strukturi na $G \times G$,
		\item $\sigma (x, y) = 0 \iff y^{-1}x \in \bigcap_{k=1}^{\infty}U_k$,
		\item $\sigma (x, y) \leq 2^{-k+2}$, če je $y^{-1}x \in U_k$,
		\item $2^{-k} \leq \sigma (x, y)$, če $y^{-1}x \notin U_k$.
	\end{itemize}
	Če poleg tega velja še, da $x U_k x^{-1} = U_k$ za vse $x \in G$ in $k\in\mathbb{N}$, je $\sigma$ tudi desnoinvariantna in velja:
	\begin{itemize}[label=\ding{227}]
		\item $\sigma (x^{-1}, y^{-1}) = \sigma (x, y)$ za vsaka $x, y \in G$.
	\end{itemize}
\end{matematika}
\end{frame}

\begin{frame}
\frametitle{Metrizabilnost}
\begin{matematika}{Izrek}
Naj bo $G$ topološka grupa, ki zadošča separacijskemu aksiomu $T_0$. Tedaj je $G$ metrizabilen topološki prostor natanko tedaj, ko obstaja števna baza odprtih okolic enote.
\end{matematika}
\end{frame}

\begin{frame}
\frametitle{Povsem regularnost}
\begin{matematika}{Definicija}
Topološki prostor $X$ zadošča separacijskemu aksiomu $T_{3\frac{1}{2}}$, če za vsako zaprto množico $F \subset X$ in točko $a \in X \setminus F$ obstaja zvezna funkcija $\psi: X \to \mathbb{R}$, da je $\psi(a) = 0$ in $\psi|_F = 1$.
\end{matematika}\newline

Topološki prostor je \emph{povsem regularen}, če zadošča separacijskima aksiomoma $T_1$ in $T_{3\frac{1}{2}}$. \newline

\begin{matematika}{Izrek}
Vsaka topološka grupa $G$, ki zadošča separacijskemu aksiomu $T_0$, je povsem regularen topološki prostor.
\end{matematika}
\end{frame}

\begin{frame}
\frametitle{Normalnost}
\begin{itemize}[label=\ding{227}]
\item Ali velja tudi $T_0 \implies T_4$? \pause Ne. \pause
\item Kaj manjka? \newline \pause
\end{itemize}

\begin{matematika}{Izrek}
Vsaka lokalno kompaktna topološka grupa $G$, ki zadošča separacijskemu aksiomu $T_0$, je normalen topološki prostor.
\end{matematika}
\end{frame}

\end{document}