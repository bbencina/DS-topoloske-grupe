\documentclass[a4paper, 12pt]{beamer}

\usetheme{CambridgeUS}
\usecolortheme{beaver}
\usefonttheme{structuresmallcapsserif}


\usepackage[slovene]{babel}
\usepackage[utf8]{inputenc}
\usepackage[T1]{fontenc}
\usepackage{lmodern}
\usepackage{units}
\usepackage{eurosym}
\usepackage{amsmath}
\usepackage{amssymb}
\usepackage{amsthm}
\usepackage{amsfonts}
\usepackage{mathtools}
\usepackage{graphicx}
\usepackage{color}
%\usepackage{url}
\usepackage{hyperref}
\usepackage{enumerate}
\usepackage{enumitem}
\usepackage{pifont}

\definecolor{bostonuniversityred}{rgb}{0.8, 0.0, 0.0}

\newenvironment{matematika}[1]{
\textcolor{bostonuniversityred}{\underline{\textsc{#1:}}}
}{
}

\title{Topološke grupe}
\author{Benjamin Benčina}
\institute[FMF]{Fakulteta za matematiko in fiziko}
\date{\today}

\begin{document}
%%%%%%%%%%%%%%%%%%%%%%%%%%%%%%%%%%%%%%%%
%%%%%%%%%%%%%%%TEST AREA%%%%%%%%%%%%%%%%


%%%%%%%%%%%%%%%%%%%%%%%%%%%%%%%%%%%%%%%%

\titlepage

\begin{frame}
\begin{matematika}{Definicija}
Neprazna množica $G$ z binarno operacijo $*$ je \emph{grupa}, če:
\begin{itemize}[label=\ding{227}]
\item je množica $G$ zaprta za (binarno) operacijo $*$,
\item je operacija $*$ asociativna v množici $G$,
\item obstaja tak element $e \in G$, da za vsak element $x \in G$ velja enakost 
\[ x*e = e*x = x, \]
\item za vsak element $x \in G$ obstaja element $y \in G$, da velja enakost
\[ x*y = y*x = e. \]
\end{itemize}
Označimo: $(G, *)$ ali včasih samo $G$.
\end{matematika}
\end{frame}

\begin{frame}
\begin{matematika}{Definicija}
\emph{Topologija} na neprazni množici $X$ je družina $\tau \subseteq 2^{X}$ z lastnostmi:
\begin{itemize}[label=\ding{227}]
\item $X \in \tau$, $\emptyset \in \tau$,
\item $U \in \tau$ in $V \in \tau \implies U \cap V \in \tau$,
\item $\lbrace U_{\lambda} \rbrace_{\lambda \in \Lambda} \subseteq \tau \implies \bigcup\limits_{\lambda \in \Lambda}^{} U_{\lambda} \in \tau$.
\end{itemize}
\end{matematika}
\end{frame}

\begin{frame}
\begin{matematika}{Definicija}
\emph{Topološka grupa} je grupa $(G, *)$ s topologijo $\tau$, glede na katero sta strukturni operaciji zvezni.
\end{matematika}
\newline
\newline
Strukturni operaciji:
\begin{itemize}[label=\ding{227}]
\item Množenje: $\mu : G \times G \to G$, $(x, y) \mapsto xy$.
\item Invertiranje: $\iota : G \to G$, $x \mapsto x^{-1}$.
\end{itemize}

\end{frame}

\begin{frame}
\frametitle{Primeri}
\begin{itemize}[label=\ding{227}]
\item<1-> poljubna grupa z diskretno ali trivialno topologijo,
\item<2-> aditivna grupa realnih števil z običajno topologijo $\tau_e$,
\item<3-> enotska krožnica v $\mathbb{R}^n$ s podedovanim množenjem in relativno topologijo,
\item<4-> grupa linarnih izomorfizmov $\mathbb{GL}(n,\mathbb{R})$ z matričnim množenjem, gledana kot podprostor $n^2$-razsežnega Evklidskega prostora
\end{itemize}
\end{frame}

\begin{frame}
\frametitle{Motivacija}
\begin{itemize}[label=\ding{227}]
\item<1-> Zanimive so same po sebi,
\item<2-> uporabne so v harmonični analizi (Fourierjeve vrste, integrali...),
\item<3-> pojavijo se v teoriji Liejevih grup,
\item<4-> povezane so z nekaterimi proglemi v fiziki itd.
\end{itemize}
\end{frame}

\begin{frame}
\frametitle{Kvocientni prostori}
\begin{matematika}{Definicija}
Naj bo $G$ topološka grupa in $H$ njena podgrupa. Kot podprostor $H$ prostora $G$ z relativno topologijo je $H$ topološka grupa.
\end{matematika}

\pause

\begin{matematika}{Definicija}
Naj bo $G$ topološka grupa in $H$ njena podgrupa. Naj bo $\varphi$ naravni homomorfizem $x \mapsto xH$. Definiramo topologijo $\tau (G/H)$ na $G/H$ tako: $\{ xH | x \in X \}$ je v $G/H$ odprta natanko tedaj, ko je $\varphi^{-1}(\{ xH | x \in X \})$ odprta v $G$.
\end{matematika} \newline

\pause

\begin{matematika}{Izrek}
Naj bo $G$ topološka grupa in $N$ njena podgrupa edinka. Grupa $G/N$ z zgoraj definirano topologijo je topološka grupa.
\end{matematika}
\end{frame}

\begin{frame}
\frametitle{Izreki o izomorfizmih}
\begin{matematika}{Izrek}
Naj bosta $G$ in $H$ grupi ter $\varphi: G \to H$ homomorfizem. Tedaj $G/Ker(\varphi) \cong Im(\varphi)$. \\
Če je $\varphi$ še surjektiven, potem $G/Ker(\varphi) \cong H$.
\end{matematika}
\newline

\begin{matematika}{Izrek}
Naj bo $G$ grupa, $H$ njena podgrupa in $N$ njena podgrupa edinka. Tedaj $(HN)/N \cong H/(H \cap N)$.
\end{matematika}
\newline

\begin{matematika}{Izrek}
Naj bo $G$ grupa, $N$ in $K$ njeni podgrupi edinki in naj velja $N \subseteq K \subseteq G$. Tedaj $(G/N)/(K/N) \cong G/K$.
\end{matematika}
\end{frame}



\begin{frame}
\frametitle{Izreki o izomorfizmih na kvocientnih prostorih}
\begin{matematika}{Izrek}
Naj bosta $G$ in $H$ topološki grupi in $f: G \to H$ zvezen homomorfizem in kvocientna preslikava (surjektiven in $V$ odprta v $H \iff f^{-1}(V)$ odprta v $G$). Tedaj $G/Ker(f) \cong H$.
\end{matematika}
\newline

\begin{matematika}{Izrek}
Naj bo $G$ topološka grupa, $N$ njena podgrupa edinka in $H$ njena podgrupa. Denimo, da je $N$ zaprta v $G$, $H$ lokalno kompaktna in unija največ števno mnogo kompaktnih podprostorov ($\sigma$-kompaktna) in $HN$ tudi lokalno kompaktna. Tedaj $(HN)/N \cong H/(H \cap N)$.
\end{matematika}
\newline

\begin{matematika}{Izrek}
Naj bo $G$ topološka grupa, $N$ in $K$ njeni podgrupi edinki in naj velja $N \subseteq K \subseteq G$. Tedaj $(G/N)/(K/N) \cong G/K$.
\end{matematika}
\end{frame}

\begin{frame}
\frametitle{Metrizabilnost}
\begin{matematika}{Definicija}
\emph{Metrika} na množici $X$ je nenegativna funkcija $d: X \times X \to [0,\infty)$, ki zadošča pogojem:
\begin{itemize}[label=\ding{227}]
\item $d(x, y) \geq 0$,
\item $d(x, y) = 0 \iff x = y$,
\item $d(x, y) = d(y, x)$,
\item $d(x, z) \leq d(x, y) + d(y, z)$.
\end{itemize}
\end{matematika}

\pause

\begin{matematika}{Definicija}
\emph{Pseudo-metrika} na množici $X$ je nenegativna funkcija $d: X \times X \to [0,\infty)$, ki zadošča pogojem:
\begin{itemize}[label=\ding{227}]
\item $d(x, y) \geq 0$,
\item $d(x, x) = 0$,
\item $d(x, y) = d(y, x)$,
\item $d(x, z) \leq d(x, y) + d(y, z)$.
\end{itemize}
\end{matematika}
\end{frame}

\begin{frame}
\frametitle{Metrizabilnost}
\begin{matematika}{Izrek}
Naj bo $\{ U_k \}_{k=1}^{\infty}$ družina simetričnih okolic enote $e$ topološke grupe $G$ z lastnostjo $U_{k+1}^2 \subset U_k$ za vsak $k\in\mathbb{N}$. Potem obstaja taka levoinvariantna pseudo-metrika $\sigma$, da velja:
\begin{itemize}[label=\ding{227}]
\item $\sigma$ je enakomerno zvezna na levi uniformni strukturi na $G \times G$,
\item $\sigma (x, y) = 0 \iff y^{-1}x \in \bigcap_{k=1}^{\infty}U_k$,
\item $\sigma (x, y) \leq 2^{-k+2}$, če je $y^{-1}x \in U_k$,
\item $2^{-k} \leq \sigma (x, y)$, če $y^{-1}x \notin U_k$.
\end{itemize}
Če poleg tega velja še, da $x U_k x^{-1} = U_k$ za vse $x \in G$ in $k\in\mathbb{N}$, je $\sigma$ tudi desnoinvariantna in velja:
\begin{itemize}[label=\ding{227}]
\item $\sigma (x^{-1}, y^{-1}) = \sigma (x, y)$ za vsaka $x, y \in G$.
\end{itemize}
\end{matematika}
\end{frame}

\begin{frame}
\frametitle{Metrizabilnost}
\begin{matematika}{Izrek}
Topološka grupa $G \in T_0$ je metrizabilna natanko tedaj, ko obstaja števna, odprta baza okolic za enoto $e$. V tem primeru lahko za metriko vzamemo kar levoinvariantno pseudo-metriko iz prejšnjega izreka. 
\end{matematika}
\end{frame}

\begin{frame}
\frametitle{Separacijski aksiomi}
\begin{itemize}[label=\ding{227}]
\item $T_0$, $T_1$, $T_2$, $T_3$, $T_4$,
\end{itemize}

\pause

\begin{matematika}{Definicija}
Naj bo $X$ topološki prostor in $A,B \subset X$ njegovi disjunktni podmnožici. \emph{Urisonova funkcija} za množici $A$ in $B$ je zvezna funkcija $\varphi : X \to [0,1]$, za katero velja $\varphi |_A \equiv 0$ in $\varphi |_B \equiv 1$.
\end{matematika}
\newline

\pause

\begin{matematika}{Definicija}
Topološki prostor $X$ zadošča separacijskemu aksiomu $T_{3 \frac{1}{2}}$ natanko tedaj, ko za vsako zaprto podmnožico $A \subset X$ in vsako točko $y \notin A$ obstaja Urisonova funkcija. \\
Če $X \in T_1$ in $X \in T_{3 \frac{1}{2}}$, rečemo, da je X \emph{povsem regularen} topološki prostor.
\end{matematika}

\end{frame}

\begin{frame}
\frametitle{Separacijski aksiomi}
\begin{itemize}[label=\ding{227}]
\item Vemo: $X \in T_2 \implies X \in T_1 \implies X \in T_0$.
\end{itemize}

\pause

\begin{matematika}{Izrek}
Naj bo $G$ topološka grupa, $a \in G$ in $F \subset G$ zaprta podmnožica, ki ne vsebuje $a$. Naj bo $G \in T_0$. Tedaj za $F$ in $a$ obstaja Urisonova funkcija in $G \in T_{3 \frac{1}{2}}$.
\end{matematika}
\newline

\pause

Z drugimi besedami, $T_0$ zadošča za povsem regularnost.
\newline

\pause

Nas topološko grupna struktura lahko pripelje še dlje (do $T_4$)?

\end{frame}

\begin{frame}
\frametitle{Cilji:}
\begin{itemize}[label=\ding{227}]
\item<1-> pokazati, da na kvocientnih topoloških grupah veljajo analogni izreki o izomorfizmih (homeomorfizmih);
\item<2-> karakterizirati metrizabilnost na topoloških grupah;
\item<3-> študirati separacijske aksiome na topoloških grupah in dokazati, da lahko iz $T_0$ pridemo do $T_{3 \frac{1}{2}}$;
\item<4-> najti protiprimer za $T_0 \implies T_4$;
\item<5-> študirati separacijske aksiome in metrizabilnost na kvocientnih topoloških grupah (izreki tipa "2 od 3").
\end{itemize}
\end{frame}



\end{document}