\documentclass[mat1]{fmfdelo}
% \documentclass[fin1]{fmfdelo}
% \documentclass[isrm1]{fmfdelo}
% \documentclass[mat2]{fmfdelo}
% \documentclass[fin2]{fmfdelo}
% \documentclass[isrm2]{fmfdelo}

% naslednje ukaze ustrezno napolnite
\avtor{Benjamin Benčina}

\naslov{Topološke grupe}
\title{Topological groups}

% navedite ime mentorja s polnim nazivom: doc.~dr.~Ime Priimek,
% izr.~prof.~dr.~Ime Priimek, prof.~dr.~Ime Priimek
% uporabite le tisti ukaz/ukaze, ki je/so za vas ustrezni
\mentor{doc.~dr.~Marko Kandič}
% \mentorica{}
% \somentor{}
% \somentorica{}
% \mentorja{}{}
% \mentorici{}{}

\letnica{2019} % leto diplome

%  V povzetku na kratko opišite vsebinske rezultate dela. Sem ne sodi razlaga organizacije dela --
%  v katerem poglavju/razdelku je kaj, pač pa le opis vsebine.
\povzetek{povzetek HERE}

%  Prevod slovenskega povzetka v angleščino.
\abstract{ABSTRACT HERE}

% navedite vsaj eno klasifikacijsko oznako --
% dostopne so na www.ams.org/mathscinet/msc/msc2010.html
\klasifikacija{43-00}
\kljucnebesede{grupa topologija} % navedite nekaj ključnih pojmov, ki nastopajo v delu
\keywords{group topology} % angleški prevod ključnih besed

\zapisiMetaPodatke  % poskrbi za metapodatke in veljaven PDF/A-1b standard

% aktivirajte pakete, ki jih potrebujete
% \usepackage{tikz}
\usepackage[slovene]{babel}
\usepackage[utf8]{inputenc}
\usepackage[T1]{fontenc}
\usepackage{lmodern}
\usepackage{amsmath}
\usepackage{amssymb}
\usepackage{amsthm}
\usepackage{amsfonts}
\usepackage{mathtools}

% za številske množice uporabite naslednje simbole
\newcommand{\R}{\mathbb R}
\newcommand{\N}{\mathbb N}
\newcommand{\Z}{\mathbb Z}
\newcommand{\C}{\mathbb C}
\newcommand{\Q}{\mathbb Q}

% matematične operatorje deklarirajte kot take, da jih bo Latex pravilno stavil
% \DeclareMathOperator{\conv}{conv}

% vstavite svoje definicije ...
%  \newcommand{}{}

\begin{document}

\section{Uvod}
Tukaj bom napisal kratek uvod o uporabi združene topološke in algebraične strukture, zgodovinske primere uporabe topoloških grup in nato na kratko opisal strukturo dela.

\section{Kaj je topološka grupa}
%Tukaj bom navedel definicijo topološke grupe in formuliral ter dokazal nekaj osnovnih trditev oz. lem, ki pokažejo, da ima definicija smisel (topološka podgrupa, odprtost in zaprtost).
\begin{definicija}
Neprazna množica $G$ z binarno operacijo $*$ je \emph{grupa}, če:
\begin{enumerate}
\item je množica $G$ zaprta za (ponavadi binarno) operacijo $*$,
\item je operacija $*$ asociativna v množici $G$,
\item v $G$ obstaja tak element $e$ (imenujemo ga \emph{enota}), da za vsak element $x$ množice $G$ veljajo enakosti \[ x*e = e*x = x, \]
\item za vsak element $x$ množice $G$ obstaja element $y$ tudi iz množice $G$, da veljajo enakosti \[ x*y = y*x = e. \]
\end{enumerate}
Oznaka za grupo je ($G$, $*$) ali samo $G$, če je operacija znana ali drugače očitna.
\end{definicija}

Iz zgornje definicije je takoj razvidno, da nam grupna struktura na množici porodi dve strukturni preslikavi:
\begin{itemize}
\item \emph{množenje} $\mu: G \times G \to G$, $(x, y) \mapsto x*y$,
\item \emph{invertiranje} $\iota: G \to G$, $x \mapsto x^{-1}$.
\end{itemize}

\begin{definicija}
\emph{Topologija} na neprazni množici $X$ je družina podmnožic $\tau \subseteq 2^X$ z lastnostmi:
\begin{enumerate}
\item $X \in \tau$, $\emptyset \in \tau$,
\item za poljubni dve množici $U,V \in \tau$ je tudi presek $U \cap V \in \tau$,
\item za poljubno poddružino $\lbrace U_{\lambda} \rbrace_{\lambda \in \Lambda} \subseteq \tau$ je tudi unija $\bigcup\limits_{\lambda \in \Lambda}^{} U_{\lambda} \in \tau$.
\end{enumerate}
Množici $X$, opremljeni s topologijo $\tau$, rečemo \emph{topološki} prostor $(X, \tau)$ in množice v družini $\tau$ označimo za \emph{odprte} množice v topološkem prostoru $X$. \emph{Zaprte} množice definiramo kot komplemente odprtih množic glede na množico $X$.
\end{definicija}

S pomočjo odprtih in zaprtih množic topološkega prostora $X$ lahko sedaj definiramo zveznost preslikave med dvema topološkima prostoroma.

\begin{definicija}
Preslikava $f: (X, \tau_1) \to (Y, \tau_2)$ je \emph{zvezna}, kadar je praslika preslikave $f$ vsake odprte množice v topološkem prostoru $(Y, \tau_2)$ oprta tudi v topološkem prostoru $(X, \tau_1)$.
\end{definicija}

Končno lahko strukturi združimo in povežemo ter definiramo pojem topološke grupe.
\begin{definicija}
\emph{Topološka grupa} je grupa $(G, *)$ opremljena s tako topologijo $\tau$ na množici $G$, da sta za $\tau$ strukturni operaciji množenja in invertiranja zvezni. 
\end{definicija}

\subsection{Primeri topoloških grup}
Tukaj bom sistematično opisal nekaj osnovnih ali očitnih primerov topoloških grup in za nekaj primerov tudi dokazal, da ustrezajo definiciji.

\section{Separacijsi aksiomi in metrizabilnost}
Opis poglavja, kaj me bo zanimalo in pomožne trditve, ki jih nisem navedel že v prejšnjem poglavju.
Definicije aksiomov in metrizabilnosti.

\subsection{Metrizabilnost}
Izrek o pseudo-metriki, karakterizacija metrizabilnosti top. grup.

\subsection{Separacijski aksiomi do T$_{3 \frac{1}{2}}$}
Še dokaz izreka t0 -> t31/2.

\subsection{Separacijski aksiom T$_4$}
Protiprimer za t0 -> t4. Kaj nam še manjka + dokaz.

\section{Kvocienti topoloških grup}
Opis poglavja, definicija top. kvocienta in rel. topologije ter dokaz, da teorija deluje tudi za kvociente.
Nekaj osnovnih trditev oz. lem.
Pomožne trditve za glavni izrek, ki povzame kvociente.
Izrek + dokaz.

\section{Izreki o izomorfizmih}
Opis poglavja in naše želje o obliki analognih izrekov.
Algebraični izreki o izomorfizmih (brez dokaza).
Vsak analogen izrek posebej s pomožnimi trditvami v svojem podrazdelku.

\section{Izreki tipa ``2 od 3''}
Kasneje.



\section*{Slovar strokovnih izrazov}

\geslo{}{}
\geslo{}{}

% seznam uporabljene literature
\begin{thebibliography}{99}

%\bibitem{}

\end{thebibliography}

\end{document}

