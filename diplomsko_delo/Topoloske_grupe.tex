\documentclass[mat1]{fmfdelo}
% \documentclass[fin1]{fmfdelo}
% \documentclass[isrm1]{fmfdelo}
% \documentclass[mat2]{fmfdelo}
% \documentclass[fin2]{fmfdelo}
% \documentclass[isrm2]{fmfdelo}

% naslednje ukaze ustrezno napolnite
\avtor{Benjamin Benčina}

\naslov{Topološke grupe}
\title{Topological groups}

% navedite ime mentorja s polnim nazivom: doc.~dr.~Ime Priimek,
% izr.~prof.~dr.~Ime Priimek, prof.~dr.~Ime Priimek
% uporabite le tisti ukaz/ukaze, ki je/so za vas ustrezni
\mentor{doc.~dr.~Marko Kandič}
% \mentorica{}
% \somentor{}
% \somentorica{}
% \mentorja{}{}
% \mentorici{}{}

\letnica{2019} % leto diplome

%  V povzetku na kratko opišite vsebinske rezultate dela. Sem ne sodi razlaga organizacije dela --
%  v katerem poglavju/razdelku je kaj, pač pa le opis vsebine.
\povzetek{povzetek HERE}

%  Prevod slovenskega povzetka v angleščino.
\abstract{ABSTRACT HERE}

% navedite vsaj eno klasifikacijsko oznako --
% dostopne so na www.ams.org/mathscinet/msc/msc2010.html
\klasifikacija{43-00}
\kljucnebesede{grupa topologija} % navedite nekaj ključnih pojmov, ki nastopajo v delu
\keywords{group topology} % angleški prevod ključnih besed

\zapisiMetaPodatke  % poskrbi za metapodatke in veljaven PDF/A-1b standard

% aktivirajte pakete, ki jih potrebujete
% \usepackage{tikz}
\usepackage[slovene]{babel}
\usepackage[utf8]{inputenc}
\usepackage[T1]{fontenc}
\usepackage{lmodern}
\usepackage{amsmath}
\usepackage{amssymb}
\usepackage{amsthm}
\usepackage{amsfonts}
\usepackage{mathtools}

% za številske množice uporabite naslednje simbole
\newcommand{\R}{\mathbb R}
\newcommand{\N}{\mathbb N}
\newcommand{\Z}{\mathbb Z}
\newcommand{\C}{\mathbb C}
\newcommand{\Q}{\mathbb Q}

% matematične operatorje deklarirajte kot take, da jih bo Latex pravilno stavil
% \DeclareMathOperator{\conv}{conv}

% vstavite svoje definicije ...
%  \newcommand{}{}

\begin{document}

\section{Uvod}
Tukaj bom napisal kratek uvod o uporabi združene topološke in algebraične strukture, zgodovinske primere uporabe topoloških grup in nato na kratko opisal strukturo dela.

\section{Kaj je topološka grupa}
Tukaj bom navedel definicijo topološke grupe in formuliral ter dokazal nekaj osnovnih trditev oz. lem, ki pokažejo, da ima definicija smisel (topološka podgrupa, odprtost in zaprtost).

\subsection{Primeri topoloških grup}
Tukaj bom sistematično opisal nekaj osnovnih ali očitnih primerov topoloških grup in za nekaj primerov tudi dokazal, da ustrezajo definiciji.

\section{Separacijsi aksiomi in metrizabilnost}
Opis poglavja, kaj me bo zanimalo in pomožne trditve, ki jih nisem navedel že v prejšnjem poglavju.
Definicije aksiomov in metrizabilnosti.

\subsection{Metrizabilnost}
Izrek o pseudo-metriki, karakterizacija metrizabilnosti top. grup.

\subsection{Separacijski aksiomi do T$_{3 \frac{1}{2}}$}
Še dokaz izreka t0 -> t31/2.

\subsection{Separacijski aksiom T$_4$}
Protiprimer za t0 -> t4. Kaj nam še manjka + dokaz.

\section{Kvocienti topoloških grup}
Opis poglavja, definicija top. kvocienta in rel. topologije ter dokaz, da teorija deluje tudi za kvociente.
Nekaj osnovnih trditev oz. lem.
Pomožne trditve za glavni izrek, ki povzame kvociente.
Izrek + dokaz.

\section{Izreki o izomorfizmih}
Opis poglavja in naše želje o obliki analognih izrekov.
Algebraični izreki o izomorfizmih (brez dokaza).
Vsak analogen izrek posebej s pomožnimi trditvami v svojem podrazdelku.

\section{Izreki tipa ``2 od 3''}
Kasneje.



\section*{Slovar strokovnih izrazov}

\geslo{}{}
\geslo{}{}

% seznam uporabljene literature
\begin{thebibliography}{99}

%\bibitem{}

\end{thebibliography}

\end{document}

