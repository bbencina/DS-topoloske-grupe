\documentclass[mat1]{fmfdelo}
% \documentclass[fin1]{fmfdelo}
% \documentclass[isrm1]{fmfdelo}
% \documentclass[mat2]{fmfdelo}
% \documentclass[fin2]{fmfdelo}
% \documentclass[isrm2]{fmfdelo}

% naslednje ukaze ustrezno napolnite
\avtor{Benjamin Benčina}

\naslov{Topološke grupe}
\title{Topological groups}

% navedite ime mentorja s polnim nazivom: doc.~dr.~Ime Priimek,
% izr.~prof.~dr.~Ime Priimek, prof.~dr.~Ime Priimek
% uporabite le tisti ukaz/ukaze, ki je/so za vas ustrezni
\mentor{doc.~dr.~Marko Kandić}
% \mentorica{}
% \somentor{}
% \somentorica{}
% \mentorja{}{}
% \mentorici{}{}

\letnica{2019} % leto diplome

%  V povzetku na kratko opišite vsebinske rezultate dela. Sem ne sodi razlaga organizacije dela --
%  v katerem poglavju/razdelku je kaj, pač pa le opis vsebine.
\povzetek{povzetek HERE}

%  Prevod slovenskega povzetka v angleščino.
\abstract{ABSTRACT HERE}

% navedite vsaj eno klasifikacijsko oznako --
% dostopne so na www.ams.org/mathscinet/msc/msc2010.html
\klasifikacija{43-00}
\kljucnebesede{grupa topologija} % navedite nekaj ključnih pojmov, ki nastopajo v delu
\keywords{group topology} % angleški prevod ključnih besed

\zapisiMetaPodatke  % poskrbi za metapodatke in veljaven PDF/A-1b standard

% aktivirajte pakete, ki jih potrebujete
% \usepackage{tikz}
\usepackage[slovene]{babel}
\usepackage[utf8]{inputenc}
\usepackage[T1]{fontenc}
\usepackage{lmodern}
\usepackage{amsmath}
\usepackage{amssymb}
\usepackage{amsthm}
\usepackage{amsfonts}
\usepackage{mathtools}
\usepackage{enumitem}

% za številske množice uporabite naslednje simbole
\newcommand{\R}{\mathbb R}
\newcommand{\N}{\mathbb N}
\newcommand{\Z}{\mathbb Z}
\newcommand{\C}{\mathbb C}
\newcommand{\Q}{\mathbb Q}

\newcommand{\Ucurl}{\mathcal{U}}
\newcommand{\closure}[1]{\overline{#1}}
% matematične operatorje deklarirajte kot take, da jih bo Latex pravilno stavil
% \DeclareMathOperator{\conv}{conv}

% vstavite svoje definicije ...
%  \newcommand{}{}

\begin{document}

\section{Uvod}

\section{Preliminarna poglavja}

\subsection{Operacije na množicah}\label{sec:opnamnozicah}
Vse operacije na množicah, če ne bo drugače zaznamovano, delujejo na elementih. Tako je na primer produkt množic $U$ in $V$ enak \[U * V = \lbrace u * v ; u \in U, v \in V \rbrace, \] inverz množice $U$ pa je \[ U^{-1} = \lbrace u^{-1} ; u \in G \rbrace. \] Tukaj se v obeh primerih predpostavlja, da so množice vložene v neki grupi, kjer so operacije na elementih smiselno definirane. Grupno strukturo bom bolj podrobno opisal v naslednjem podrazdelku.

Pomembnejša izjema temu pravilu so operacije na množicah v smislu relacij. Predpostavimo torej, da imamo množico $X$ in nas zanimajo podmnožice kartezičnega produkta $X \times X$. Inverz take množice $U$ je potem \[ U^{-1} = \lbrace (y, x) ; (x, y) \in U \rbrace, \]
analogna operacija množenju pa je tukaj kompozitum množic \[ V \circ U = \lbrace (x, z) ; \text{ obstaja element } y \in X, \text{ da je } (x, y) \in V \text{ in } (y, z) \in U \rbrace. \]
Takšno dojemanje operacij bo vedno posebej označeno.

\subsection{Teorija grup}
\begin{definicija}\label{def:grupa}
Neprazna množica $G$ z binarno operacijo $*$ je \emph{grupa}, če:
\begin{enumerate}
\item je množica $G$ zaprta za (ponavadi binarno) operacijo $*$,
\item je operacija $*$ asociativna v množici $G$,
\item v $G$ obstaja tak element $e$ (imenujemo ga \emph{enota}), da za vsak element $x$ množice $G$ veljajo enakosti \[ x*e = e*x = x, \]
\item za vsak element $x$ množice $G$ obstaja element $y$ tudi iz množice $G$, da veljajo enakosti \[ x*y = y*x = e. \]
\end{enumerate}
Oznaka za grupo je ($G$, $*$) ali samo $G$, če je operacija znana ali drugače očitna.
\end{definicija}

Iz zgornje definicije je razvidno, da nam grupna struktura na množici porodi dve strukturni preslikavi:
\begin{itemize}
\item \emph{množenje} $\mu: G \times G \to G$, $(x, y) \mapsto x*y$,
\item \emph{invertiranje} $\iota: G \to G$, $x \mapsto x^{-1}$.
\end{itemize}

Definiramo lahko nekaj tipov preslikav med grupami.

\begin{definicija}\label{def:algpreslikave}
Naj bo $f: (G, *) \to (\widetilde{G}, \star)$ preslikava med dvema grupama.
\begin{enumerate}
\item Preslikava $f$ je \emph{homomorfizem}, če za vsaka dva elementa $a, b \in G$ velja $f(a*b) = f(a) \star f(b)$.
\item Preslikava $f$ je \emph{izomorfizem}, če je bijektivni homomorfizem.
\end{enumerate}
\end{definicija}

\begin{definicija}\label{def:podgrupa}
Naj bo $G$ grupa za operacijo $*$.
\begin{enumerate}
\item Podmnožica $H$ grupe $G$ je \emph{podgrupa}, če je tudi sama grupa za operacijo $*$.
\item Množici $aH = \lbrace a*h ; h \in H \rbrace$ pravimo \emph{levi odsek} grupe $G$ elementa $a \in G$ po podgrupi $H$. Na enak način definiramo definiramo \emph{desne odseke} $Ha$.
\item Podgrupi $H$ grupe $G$ rečemo podgrupa \emph{edinka}, če za vsak element $a \in G$ velja, da je levi odsek enak desnemu.
\item Množici $G/H = \lbrace aH ; a \in G \rbrace$ rečemo \emph{kvocient} grupe $G$ po podgrupi $H$.
\item \emph{Naravna preslikava} na kvocient $G/H$ je preslikava $\varphi: G \to G/H$, $a \mapsto aH$.
\end{enumerate}
\end{definicija}

\begin{trditev}
Če je podgrupa $N$ grupe $G$ podgrupa edinka, je kvocient $G/N$ grupa za operacijo $*$, kjer je $aH*bH = (a*b)H$, naravna preslikava $\varphi$ pa je homomorfizem grup.
\end{trditev}

\subsection{Topološki prostori}
\begin{definicija}\label{def:topologija}
\emph{Topologija} na neprazni množici $X$ je družina podmnožic $\tau \subseteq 2^X$ z lastnostmi:
\begin{enumerate}
\item $X \in \tau$, $\emptyset \in \tau$,
\item za poljubni dve množici $U,V \in \tau$ je tudi presek $U \cap V \in \tau$,
\item za poljubno poddružino $\lbrace U_{\lambda} \rbrace_{\lambda \in \Lambda} \subseteq \tau$ je tudi unija $\bigcup\limits_{\lambda \in \Lambda}^{} U_{\lambda} \in \tau$.
\end{enumerate}
Množici $X$, opremljeni s topologijo $\tau$, rečemo \emph{topološki} prostor $(X, \tau)$ in množice v družini $\tau$ označimo za \emph{odprte} množice v topološkem prostoru $X$. \emph{Zaprte} množice definiramo kot komplemente odprtih množic glede na množico $X$.
\end{definicija}

\begin{definicija}\label{def:baza}
Naj bo $(X, \tau)$ topološki prostor.
\begin{enumerate}
\item Podmnožica $B \subset \tau$ je \emph{baza} za topologijo $\tau$, če je vsaka množica iz topologije $\tau$ unija nekaterih množic iz $B$.
\item Podmnožica $P$ je \emph{podbaza} za topologijo $\tau$, če je družina vseh presekov končno mnogo množic iz $P$ neka baza za topologijo $\tau$.
\end{enumerate}
\end{definicija}

\begin{definicija}\label{def:okolica}
Naj bo $(X, \tau)$ topološki prostor.
\begin{enumerate}
\item Množica $U \subseteq X$ je \emph{okolica za točko} $x \in X$, če obstaja taka odprta množica $V \in \tau$, da velja $V \subseteq U$ in $x \in V$.
\item Množica $U \subseteq X$ je \emph{okolica} množice $A \subseteq X$, če obstaja taka odprta množica $V \in \tau$, da velja $V \subseteq U$ in $A \subseteq V$.
\item Če je okolica $U$ iz zgornjih dveh primerov tudi sama odprta množica, jo imenujemo \emph{odprta okolica}.
\item Družina okolic $\Ucurl_x = \lbrace U_\lambda$; $\lambda \in \Lambda \rbrace$ za točko $x \in X$ se imenuje \emph{baza okolic} za $x$, če za poljubno okolico $V$ za točko $x$ velja, da obstaja tak $\lambda \in \Lambda$, da je $U_\lambda \subseteq V$.
\end{enumerate}
\end{definicija}

\begin{definicija}\label{def:notranjost}
Naj bo $(X, \tau)$ topološki prostor in $A \subseteq X$.
\begin{enumerate}
\item Točka $a \in A$ je \emph{notranja točka} množice $A$, če je $A$ okolica za točko $a$.
\item \emph{Notranjost} množice $A$ je množica vseh njenih notranjih točk. Notranjost množice označimo z $int(A)$. Očitno velja $int(A) \subseteq A$ in tudi $int(A) = A \iff A \in \tau$.
\item \emph{Zaprtje} množice $A$ je najmanjša zaprta množica v $X$, ki vsebuje $A$. Zaprtje množice označimo z $\closure{A}$. Očitno velja $A \subseteq \closure{A}$ in tudi $\closure{A} = A \iff A$ je zaprta množica.
\end{enumerate}
\end{definicija}

S pomočjo odprtih in zaprtih množic topološkega prostora $X$ lahko sedaj definiramo zveznost in odprtost preslikave med dvema topološkima prostoroma ter pojem homeomorfizma.

\begin{definicija}\label{def:toppreslikave}
Naj bo $f: (X, \tau_1) \to (Y, \tau_2)$ preslikava med topološkima prostoroma.
\begin{enumerate}
\item Preslikava $f$ je \emph{zvezna}, kadar je praslika preslikave $f$ vsake odprte množice v topološkem prostoru $(Y, \tau_2)$ odprta tudi v topološkem prostoru $(X, \tau_1)$.
\item Preslikava $f$ je \emph{odprta}, kadar je slika preslikave $f$ vsake odprte množice v topološkem prostoru $(X, \tau_1)$ odprta tudi v topološkem prostoru $(Y, \tau_2)$.
\item Preslikava $f$ je \emph{homeomorfizem}, če je bijektivna, zvezna in ima zvezen inverz.
\end{enumerate}
\end{definicija}

V svojem delu bom uporabljal še dve posebni topologiji.

\begin{definicija}
Naj bo $X$ topološki prostor s topologijo $\tau$ in $A \subseteq X$. \emph{Inducirana} ali \emph{relativna topologija} na množici $A$, inducirana s $\tau$, je družina množic $\lbrace A \cap U ; U \in \tau \rbrace$. Množici $A$ rečemo \emph{topološki podprostor} prostora $X$.
\end{definicija}

\begin{definicija}
Naj bosta $X$ in $Y$ topološka prostora s topologijama $\tau_1$ in $\tau_2$. \emph{Produktna topologija} na kartezičnemu produktu $X \times Y$ je družina množic $\lbrace U \times V ; U \in \tau_1, V \in \tau_2 \rbrace$.
\end{definicija}

\begin{definicija}\label{def:kompakt}
Naj bo $X$ topološki prostor.
\begin{enumerate}
\item Družini $\mathcal{A}$ množic rečemo \emph{pokritje} topološkega prostora $X$, če je $X \subseteq \bigcup \mathcal{A}$.
\item Družini $\mathcal{B} \subseteq \mathcal{A}$ rečemo \emph{podpokritje} topološkega prostora $X$, če je $\mathcal{B}$ tudi sama pokritje za $X$.
\item Topološki prostor je \emph{kompakten}, če vsako njegovo odprto pokritje, tj. pokritje z odprtimi množicami, vsebuje kakšno končno podpokritje.
\item Topološki prostor je \emph{lokalno kompakten}, če ima vsaka točka $x \in X$ kakšno kompaktno okolico.
\end{enumerate}
\end{definicija}

\begin{definicija}\label{def:parakompakt}
\begin{enumerate}
\item Naj bosta $\mathcal{U}$ in $\mathcal{V}$ družini podmnožic topološkega prostora $X$. Družina $\mathcal{V}$ je \emph{pofinitev} družine $\mathcal{U}$, če za vsako množico $V \in \mathcal{V}$ obstaja takšna množica $U \in \mathcal{U}$, da je $V \subset U$.
\item Družina podmnožic $\mathcal{U}$ topološkega prostora $X$ je \emph{lokalno končna}, če ima vsaka točka $x \in X$ okolico, ki seka samo končno mnogo množic iz družine $\mathcal{U}$.
\item Topološki prostor $X$ je \emph{parakompakten}, če ima vsako njegovo odprto pokritje kakšno pofinitev, ki je lokalno končno odprto pokritje prostora $X$.
\end{enumerate}
\end{definicija}

\begin{definicija}\label{def:sepaks}
	Topološki prostor $(X, \tau)$ zadošča separacijskemu aksiomu
	\begin{enumerate}
		\item $T_0$, če za poljubni različni točki $a, b \in X$ obstaja okolica $V$ za eno od točk $a, b$, ki ne vsebuje druge od točk $a, b$;
		\item $T_1$, če za poljubno točko $a \in X$ in različno točko $b \in X$ obstaja okolica $V$ za točko $a$, ki ne vsebuje točke $b$;
		\item $T_2$, če za poljubni različni točki $a, b \in X$ obstajata disjunktni okolici za točki $a$ in $b$;
		\item $T_3$, če za poljubno zaprto množico $A \subseteq X$ in točko $b \in X\backslash A$ obstajata disjunktni okolici za množico $A$ in točko $b$;
		\item $T_4$, če za poljubni disjunktni zaprti množici $A, B \subseteq X$ obstajata disjunktni okolici za množici $A$ in $B$.
	\end{enumerate}
\end{definicija}

\begin{opomba}
	\begin{enumerate}
		\item Iz definicije je razvidno, da $T_2 \implies T_1 \implies T_0$.
		\item Topološkemu prostoru, ki zadošča separacijskemu aksiomu $T_2$, pravimo \emph{Hausdorffov} topološki prostor.
		\item Topološku prostoru, ki zadošča $T_1+T_3$ pravimo \emph{regularen} topološki prostor.
		\item Topološku prostoru, ki zadošča $T_1+T_4$, pravimo \emph{normalen} topološki prostor.
	\end{enumerate}
\end{opomba}

\section{Kaj je topološka grupa}


Končno lahko strukturi združimo in povežemo ter definiramo pojem topološke grupe.
\begin{definicija}\label{def:topgrupa}
\emph{Topološka grupa} je grupa $(G, *)$ opremljena s tako topologijo $\tau$ na množici $G$, da sta za $\tau$ strukturni operaciji množenja in invertiranja zvezni. 
\end{definicija}

Potrebujemo le še tip preslikave med topološkimi grupami, ki bo ohranjal tako algebraično kot topološko strukturo.
\begin{definicija}\label{def:topizo}
Preslikava med dvema topološkima grupama je \emph{topološki izomorfizem}, če je izomorfizem in homeomorfizem.
\end{definicija}


\begin{trditev}\label{trd:trans}
Naj bo $G$ topološka grupa in $a \in G$. Leva translacija $x \mapsto ax$ in desna translacija $x \mapsto xa$ za $a$ sta homeomorfizma iz $G$ v $G$. Prav tako je preslikava invertiranja homeomorfizem iz $G$ v $G$.
\end{trditev}

\begin{trditev}\label{trd:okolice}
Za topološko grupo $G$ in odprto bazo okolic $\Ucurl$ enote $e$ veljajo naslednje trditve:
\begin{enumerate}
\item za vsako množico $U \in \Ucurl$ obstaja taka množica $V \in \Ucurl$, da velja $V^{2} \subset U$;
\item za vsako množico $U \in \Ucurl$ obstaja taka množica $V \in \Ucurl$, da velja $V^{-1} \subset U$;
\item za vsako množico $U \in \Ucurl$ in vsak element $x \in U$ obstaja taka množica $V \in \Ucurl$, da velja $xV \subset U$;
\item za vsako množico $U \in \Ucurl$ in vsak element $x \in G$ obstaja taka množica $V \in \Ucurl$, da velja $xVx^{-1} \subset U$.
\end{enumerate}

Naj bo $G$ sedaj grupa (ne topološka) in $\Ucurl$ družina podmnožic množice $G$, za katero veljajo zgornje štiri lastnosti. Naj bodo poljubni končni preseki množic iz $\Ucurl$ neprazni. Tedaj je družina $\lbrace xU \rbrace$, kjer $U \in \Ucurl$ in $x \in G$ odprta podbaza za neko topologijo na $G$. S to topologijo je $G$ topološka grupa. Družina $\lbrace Ux \rbrace$ je podbaza za isto topologijo.

Če velja še, da za vsaki množici $U,V \in \Ucurl$ obstaja množica $W \in \Ucurl$, da velja $W \subset U \cap V$, potem sta družini $\lbrace xU \rbrace$ in $\lbrace Ux \rbrace$ tudi bazi za to topologijo.
\end{trditev}

\begin{trditev}\label{trd:sim}
Vsaka topološka grupa $G$ ima bazo odprtih okolic $\Ucurl$ enote $e$, da za vsako okolico $U$ velja $U = U^{-1}$.
\end{trditev}

\begin{opomba}
Lastnosti množic iz trditve \ref{trd:sim} pravimo simetričnost.
\end{opomba}

\begin{posledica}\label{pos:sim}
Za vsako okolico $U$ enote $e$ topološke grupe $G$ obstaja taka okolica $V$ enote $e$, da velja $V^{-1} \subset U$.
\end{posledica}


\subsection{Primeri topoloških grup}




\section{Kvocienti topoloških grup}

\begin{trditev}\label{trd:toppodgrupa}
Naj bo $G$ topološka grupa in $H$ njena podgrupa. Če $H$ opremimo z relativno topologijo, potem je tudi $H$ topološka grupa.
\end{trditev}

\begin{trditev}\label{trd:zaprtost}
Naj bosta $A$ in $B$ podmnožici topološke grupa $G$. Veljajo naslednje trditve:
\begin{enumerate}
\item $\closure{A}\ \closure{B} \subset \closure{A B}$,
\item $(\closure{A})^{-1} = \closure{A^{-1}}$,
\item $x \closure{A} y = \closure{x A y}$ za vsaka dva $x, y \in G$.
\end{enumerate}

Če $G$ ustreza še separacijskemu aksiomu $T_0$, velja tudi:
\begin{enumerate}[resume]
\item če za vsaka dva elementa $a \in A$ in $b \in B$ velja enakost $ab = ba$, potem velja enakost $ab = ba$ tudi za vsaka dva elementa $a \in \closure{A}$ in $b \in \closure{B}$.
\end{enumerate}
\end{trditev}

\begin{trditev}\label{trd:odpzap}
Naj bo $G$ topološka grupa in $H$ njena podgrupa. $H$ je odprta natanko tedaj, ko ima neprazno notranjost. Vsaka odprta podgrupa $H$ topološke grupa $G$ je tudi zaprta.
\end{trditev}

\begin{trditev}\label{trd:podgrupaunija}
Naj bo $U$ simetrična okolica enote $e$ v topološki grupi $G$. Potem je $L = \bigcup_{n=1}^{\infty} U^n$ odprta in zaprta podgrupa topološke grupe $G$.
\end{trditev}


\begin{izrek}\label{izr:topkvocienta}
Naj bo $G$ topološka grupa, $H$ njena podgrupa in $\varphi: G \to G/H$ naravna preslikava. Definiramo $\theta(G/H) = \lbrace U ; \varphi^{-1}(U)$ odprta v $G \rbrace$.
Veljajo naslednje trditve:
\begin{enumerate}
\item družina $\theta(G/H)$ je topologija na kvocientu $G/H$,
\item glede na topologijo $\theta(G/H)$ je $\varphi$ zvezna preslikava,
\item družina $\theta(G/H)$ je najmočnejša topologija na kvocientu $G/H$, glede na katero je $\varphi$ zvezna preslikava,
\item $\varphi: G \to G/H$ je odprta preslikava.
\end{enumerate}
\end{izrek}

Družini $\theta(G/H)$ pravimo \emph{kvocientna topologija}, kvocientu $G/H$ pa \emph{kvocientni prostor}.


\begin{trditev}\label{trd:okolicevkvoc}
Naj bo $G$ topološka grupa, $H$ njena podgrupa in $U, V$ tako okolici enote $e$ v $G$, da velja $V^{-1}V \subset U$. Naj bo $\varphi: G \to G/H$ naravna preslikava. Potem velja $\closure{\varphi(V)} \subset \varphi(U)$.
\end{trditev}

\begin{izrek}
Za topološko grupo $G$ in njeno podgrupo $H$ veljajo naslednje trditve:
\begin{enumerate}
\item kvocientni prostor $G/H$ je diskreten natanko tedaj, ko je $H$ odprta v $G$,
\item če je $H$ zaprta v $G$, potem je kvocient $G/H$ regularen topološki prostor,
\item če kvocientni prostor $G/H$ zadošča separacijskemu aksiomu $T_0$, potem je $H$ zaprta v $G$ in velja, da je kvocient $G/H$ regularen topološki prostor.
\end{enumerate}
\end{izrek}

\begin{izrek}\label{izr:kvocpovzetek}
Naj bo $H$ podgrupa edinka topološke grupe $G$. Naj bo kvocient $G/H$ opremljen s kvocientno topologijo $\theta$. Veljajo naslednje trditve:
\begin{enumerate}
\item kvocient $G/H$ je topološka grupa s topologijo $\theta$,
\item naravni homomorfizem je odprta in zvezena preslikava,
\item kvocient $G/H$ je diskreten natanko tedaj, ko je podgrupa $H$ odprta v $G$,
\item kvocient $G/H$ zadošča separacijskemu aksiomu $T_0$ natanko tedaj, ko je podgrupa $H$ zaprta v $G$.
\end{enumerate}
\end{izrek}

\section{Izreki o izomorfizmih}

\begin{trditev}\label{trd:homogenkvoc}
	Naj bo $G$ topološka grupa in $H$ njena podgrupa. Naj bo za vsak element $a \in G$ na kvocientu $G/H$ definirana preslikava $\psi_a$ s predpisom $\psi_a(xH) = (ax)H$.
	Za vsak element $a \in G$ je $\psi_a$ homeomorfizem na prostoru $G/H$.
\end{trditev}

\begin{opomba}\label{opo:homogenkvoc}
	Če za vsaki dve točki $x, y$ topološkega prostora $X$ velja, da na prostoru $X$ obstaja homeomorfizem, ki preslika točko $x$ v točko $y$, rečemo, da je $X$ \emph{homogen} topološki prostor. Zgornja trditev pravi, da je kvocientni prostor $G/H$ homogen topološki prostor.
\end{opomba}

\begin{trditev}\label{trd:kvockompakt}
	Naj bo $G$ (lokalno) kompaktna topološka grupa in naj bo $H$ njena podgrupa. Potem je tudi kvocietni prostor $G/H$ (lokalno) kompakten.
\end{trditev}

\subsection{Prvi izrek o izomorfizmih}
\begin{izrek}[Prvi izrek o izomorfizmih za topološke grupe]\label{izr:prvitopizrek}
Naj bosta $G$ in $\widetilde{G}$ topološki grupi. Naj bo $f: G \to \widetilde{G}$ odprt, zvezen homomorfizem. Potem je $H :=$ ker$f$ podgrupa edinka v grupi $G$ in množice $f^{-1}(\tilde{x})$, kjer je $\tilde{x} \in \widetilde{G}$, so disjunktni odseki podgrupe $H$ v grupi $G$. Preslikava $\Phi:\widetilde{G} \to G/H$ s predpisom $\tilde{x} \mapsto f^{-1}(\tilde{x})$ je topološki izomorfizem.
\end{izrek}

\subsection{Drugi izrek o izomorfizmih}
\begin{izrek}\label{izr:preddrugi}
Naj bo $G$ topološka grupa, $A$ njena podgrupa in $H$ podgrupa edinka grupe $G$. Naj bo $\tau$ izomorfizem iz kvocienta $(AH)/H$ v kvocient $A/(A \cap H)$ s predpisom $\tau (aH) = a(A \cap H)$, kjer je $a \in A$. Potem $\tau$ slika odprte množice iz $(AH)/H$ v odprte množice iz $A/(A \cap H)$.
\end{izrek}

\begin{izrek}[Drugi izrek o izomorfizmih za topološke grupe]\label{izr:drugitopizrek}
Naj bodo objekti $G$, $A$, $H$ in $\tau$ isti kakor v izreku \ref{izr:preddrugi}. Naj bo podgrupa $A$ še lokalno kompaktna in $\sigma$-kompaktna, naj bo $H$ zaprta v $G$ in $AH$ lokalno kompaktna. Tedaj je $\tau$ homeomorfizem ter topološki grupi $(AH)/H$ in $A/(A \cap H)$ sta topološko izomorfni.
\end{izrek}


\subsection{Tretji izrek o izomorfizmih}
\begin{izrek}\label{izr:predtretji}
Naj bo $G$ topološka grupa z enoto $e$ in naj bo $\widetilde{G}$ topološka grupa z enoto $\tilde{e}$. Naj bo $f$ odprt, zvezen homomorfizem iz grupe $G$ v grupo $\widetilde{G}$. Naj bo $\widetilde{H}$ podgrupa edinka grupe $\widetilde{G}$. Označimo $H = f^{-1}(\widetilde{H})$ in $N = f^{-1}(\tilde{e})$ ($N$ je jedro homomorfizma $f$). Potem so grupe $G/H$, $\widetilde{G}/\widetilde{H}$ in $(G/N)/(H/N)$ topološko izomorfne.
\end{izrek}

Izrek lahko preoblikujemo v obliko, ki je bolj podobna algebraični različici in ne vsebuje pomožne topološke grupe $\widetilde{G}$.
\begin{izrek}[Tretji izrek o izomorfizmih za topološke grupe]\label{izr:tretjitopizrek}
Naj bo $G$ topološka grupa in $H,N$ taki njeni podgrupi edinki, da velja $N \subset H$. Potem sta kvocientni topološki grupi $G/H$ in $(G/N)/(H/N)$ topološko izomorfni.
\end{izrek}

\section{Izreki tipa ``2 od 3''}

\section{Separacijski aksiomi in metrizabilnost}

\begin{definicija}
Topološki prostor $X$ zadošča separacijskemu aksiomu $T_{3 \frac{1}{2}}$, če za poljubno zaprto množico $A \subseteq X$ in točko $b \in X\backslash A$ obstaja zvezna realna funkcija $\psi$ definirana na $G$, da je $\psi (b) = 0$ in $\psi (x) = 1$ za vsak $x \in A$.
\end{definicija}

\begin{opomba}
Topološku prostoru, ki zadošča $T_1+T_{3 \frac{1}{2}}$, pravimo \emph{povsem regularen} topološki prostor.
\end{opomba}


\begin{trditev}\label{pos:reghaus}
	\begin{enumerate}
		\item Vsak povsem regularen topološki prostor je regularen.
		\item Vsak normalen topološki prostor je povsem regularen.
	\end{enumerate}
\end{trditev}


\begin{izrek}\label{izr:t3}
	Vsaka topološka grupa $G$, ki zadošča separacijskemu aksiomu $T_0$ je regularen topološki prostor.
\end{izrek}

\subsection{Metrizabilnost}

\begin{definicija}\label{def:metrika}
	\emph{Pseudometrika} na neprazni množici $X$ je preslikava $d: X \times X \to  [0, \infty)$, ki zadošča naslednjim pogojem:
	\begin{enumerate}
		\item za vsaki dve točki $x, y \in X$ velja $\rho (x, y) \geq 0$ in $\rho (x, x) = 0$;
		\item za vsaki dve točki $x, y \in X$ velja $\rho (x, y) = \rho (y, x)$;
		\item za vsake tri točke $x, y, z \in X$ velja $\rho (x, z) \leq \rho (x, y) + \rho (y, z)$.
	\end{enumerate}
	Če za preslikavo $d$ velja še
	\begin{enumerate}[resume]
		\item $d(x,y) = 0 \iff x = y$,
	\end{enumerate}
	potem ji rečemo \emph{metrika}.
\end{definicija}

\begin{definicija}\label{def:uniform}
Naj bo $X$ neprazna množica.
\begin{enumerate}
\item Neprazna poddružina $\mathcal{F} \subset \mathcal{P}(X)$ je \emph{filter} množice $X$, če ima naslednje lastnosti:
\begin{enumerate}
\item družina $\mathcal{F}$ ne vsebuje prazne množice,
\item za vsako množico $F \in \mathcal{F}$ je vsaka taka množica $E \in X$, za katero velja $F \subseteq E$, tudi v družini $\mathcal{F}$,
\item če sta množici $E$ in $F$ v družini $\mathcal{F}$, je tudi množica $E \cap F$ v družini $\mathcal{F}$.
\end{enumerate}
\item Filter $\mathcal{U}$ na množici $X \times X$ definira \emph{uniformno strukturo} na množici $X$, če ima naslednje lastnosti:
\begin{enumerate}
\item vsaka množica $U \in \mathcal{U}$ ima diagonalo množice $X$ $\Delta = \lbrace (x, x) ; x \in X \rbrace$ za svojo podmnožico,
\item za vsako množico $U \in \mathcal{U}$ je tudi množica $U^{-1} \in \mathcal{U}$,
\item za vsako množico $U \in \mathcal{U}$ obstaja taka množica $V \in \mathcal{U}$, da velja $V \circ V \subseteq U$.
\end{enumerate}
Množici z uniformno stukturo rečemo tudi \emph{uniformni prostor}.
\end{enumerate}
\end{definicija}

\begin{opomba}
V zgornji definiciji so operacije na množicah mišljene v smislu relacij (glej podrazdelek \ref{sec:opnamnozicah}).
\end{opomba}

\begin{trditev}\label{trd:topguniform}
Vsaka topološka grupa je uniformni prostor.
\end{trditev}

\begin{izrek}\label{izr:pseudometrika}
	Naj bo $\lbrace U_k \rbrace_{k = 1}^{\infty}$ tako zaporedje simetričnih okolic enote $e$ v topološki grupi $G$, da za vsak $k \in \N$ velja $U_{k+1}^2 \subset U_k$. Označimo $H = \bigcap_{k=1}^{\infty} U_k$. Potem obstaja taka levoinvariantna pseudometrika $\sigma$ na $G$ z naslednjimi lastnostmi:
	\begin{enumerate}
		\item $\sigma$ je enakomerno zvezna na levi uniformni strukturi od $G \times G$;
		\item $\sigma (x, y) = 0$ natanko tedaj, ko $y^{-1}x \in H$;
		\item $\sigma (x, y) \leq 2^{-k+2}$, če $y^{-1}x \in U_k$;
		\item $2^{-k} \leq \sigma (x, y)$, če $y^{-1}x \notin U_k$.
	\end{enumerate}
	
	Če velja še $x U_k x^{-1} = U_k$ za vsak $x \in G$ in $k \in \N$, potem je $\sigma$ tudi desnoinvariantna in velja
	\begin{enumerate}[resume]
		\item $\sigma (x^{-1}, y^{-1}) = \sigma (x, y)$ za vsaka dva elementa $x, y \in G$.
	\end{enumerate}
\end{izrek}

\begin{definicija}\label{def:metrizabilnost}
Topološki prostor $X$ je \emph{metrizabilen}, če njegova topologija $\tau$ izhaja iz kakšne metrike $d$ na množici $X$, tj. baza topologije $\tau$ je družina odprtih krogel $\lbrace K(x, \epsilon); x \in X, \epsilon \in \R \rbrace$.
\end{definicija}

\begin{izrek}\label{izr:metrizabilnost}
	Topološka grupa $G$, ki zadošča separacijskemu aksiomu $T_0$, je metrizabilen topološki prostor natanko tedaj, ko obstaja števna baza odprtih okolic enote $e$.
\end{izrek}

\subsection{Separacijski aksiomi do T$_{3 \frac{1}{2}}$}

\begin{izrek}\label{izr:t3pol}
	Naj bo $G$ topološka grupa, ki zadošča separacijskemu aksiomu $T_0$. Naj bo $a \in G$ točka in $F$ zaprta podmnožica v $G$, ki ne vsebuje $a$. Potem obstaja taka zvezna realna funkcija $\psi$ definirana na $G$, da je $\psi (a) = 0$ in $\psi (x) = 1$ za vsak $x \in F$.
	
	Drugače: vsaka $T_0$ topološka grupa je povsem regularna.
\end{izrek}

\subsection{Separacijski aksiom T$_4$}
\begin{izrek}\label{izr:t4protiprimer}
	Če je $m$ katerokoli neštevno kardinalno število, potem je $\Z^{m}$ nenormalna povsem regularna topološka grupa.
\end{izrek}

\begin{trditev}\label{trd:parkompnorm} % Mrčun, Topologija, Trditev 4.21
Vsak parakompakten Hausdorffov topološki prostor je normalen.
\end{trditev}

\begin{izrek}\label{izr:t4}
	Vsaka lokalno kompaktna topološka grupa, ki zadošča separacijskemu aksiomu $T_0$, je normalen topološki prostor.
\end{izrek}

\section*{Slovar strokovnih izrazov}

\geslo{}{}
\geslo{}{}

% seznam uporabljene literature
\begin{thebibliography}{99}

\bibitem{bib:uniform}
S.~Bhowmik, \emph{Introduction to Uniform Spaces}, 10.13140/RG.2.1.3743.8967, junij 2014, [ogled 1.~4.~2019], dostopno na \url{https://www.researchgate.net/publication/305196408_INTRODUCTION_TO_UNIFORM_SPACES}.
\bibitem{bib:aha1}
E.~Hewitt in K.~A.~Ross, \emph{Abstact Harmonic Analysis I}, Springer-Verlag, New York, 1979.
\bibitem{bib:top}
J.~Mrčun, \emph{Topologija}, Izbrana poglavja iz matematike in računalništva \textbf{44} DMFA-založništvo, Ljubljana, 2008.

\end{thebibliography}

\end{document}

