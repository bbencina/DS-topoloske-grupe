\documentclass[mat1]{fmfdelo}
% \documentclass[fin1]{fmfdelo}
% \documentclass[isrm1]{fmfdelo}
% \documentclass[mat2]{fmfdelo}
% \documentclass[fin2]{fmfdelo}
% \documentclass[isrm2]{fmfdelo}

% naslednje ukaze ustrezno napolnite
\avtor{Benjamin Benčina}

\naslov{Topološke grupe}
\title{Topological groups}

% navedite ime mentorja s polnim nazivom: doc.~dr.~Ime Priimek,
% izr.~prof.~dr.~Ime Priimek, prof.~dr.~Ime Priimek
% uporabite le tisti ukaz/ukaze, ki je/so za vas ustrezni
\mentor{doc.~dr.~Marko Kandić}
% \mentorica{}
% \somentor{}
% \somentorica{}
% \mentorja{}{}
% \mentorici{}{}

\letnica{2019} % leto diplome

%  V povzetku na kratko opišite vsebinske rezultate dela. Sem ne sodi razlaga organizacije dela --
%  v katerem poglavju/razdelku je kaj, pač pa le opis vsebine.
\povzetek{
Namen tega diplomskega dela je predstaviti pojem topološke grupe in dokazati nekaj temeljnih izrekov iz študija topoloških grup. Definirana je topološka grupa in opisane so njene osnovne lastnosti. Obravnavane so topološke podgrupe in kvocientni topološki prostori topoloških grup. Pokazano je, da za topološke grupe veljajo podobni trije izreki o topoloških izomorfizmih kot za grupe. Na topološko grupo sta vpeljani leva in desna uniformna struktura, glede na kateri je vsaka topološka grupa uniformni prostor. Na topološki grupi je nato skonstruirana levoinvariantna psevdometrika. Karakterizirana je metrizabilnost za Hausdorffove topološke grupe in dokazano je, da sta za topološke grupe povsem regularnost in separacijski aksiom $T_0$ ekvivalentna. Skonstruiran je primer povsem regularne topološke grupe, ki ni normalna. Za regularne topološke prostore so navedene karakterizacije parakompaktnosti. Dokazano je, da je vsaka lokalno kompaktna Hausdorffova topološka grupa parakompaktna in posledično normalna.
}

%  Prevod slovenskega povzetka v angleščino.
\abstract{
The goal of this thesis is to present the concept of a topological group and to prove some fundamental theorems from the study of topological groups. We define a topological group and describe its basic properties. We look at topological subgroups and quotient topological spaces of topological groups. We show that for topological groups three topological isomorphism theorems hold which are similar to those for groups. We introduce left and right uniform structures on a topological group and then show that every topological space is also a uniform space. We then construct a left invariant pseudo-metric on a topological group. We characterize metrizability for Hausdorff topological groups and we prove that complete regularity and the $T_0$ separation axiom are equivalent for topological groups. We construct an example of a completely regular topological group which is not a normal topological space. For regular topological spaces we list different characterizations of paracompactness. We then prove that every locally compact Hausdorff topological group is paracompact, and hence a normal topological space.
}

% navedite vsaj eno klasifikacijsko oznako --
% dostopne so na www.ams.org/mathscinet/msc/msc2010.html
\klasifikacija{22A05, 54D15, 54D20, 54D45, 54E35}
\kljucnebesede{topološka grupa, separacijski aksiomi, metrizabilnost, povsem regularnost, parakompaktnost} % navedite nekaj ključnih pojmov, ki nastopajo v delu
\keywords{topological group, separation axioms, metrizability, complete regularity, paracompactness} % angleški prevod ključnih besed

\zapisiMetaPodatke  % poskrbi za metapodatke in veljaven PDF/A-1b standard

% aktivirajte pakete, ki jih potrebujete
% \usepackage{tikz}
\usepackage[slovene]{babel}
\usepackage[utf8]{inputenc}
\usepackage[T1]{fontenc}
\usepackage{lmodern}
\usepackage{amsmath}
\usepackage{amssymb}
\usepackage{amsthm}
\usepackage{amsfonts}
\usepackage{mathtools}
\usepackage{enumitem}

% za številske množice uporabite naslednje simbole
\newcommand{\R}{\mathbb R}
\newcommand{\N}{\mathbb N}
\newcommand{\Z}{\mathbb Z}
\newcommand{\C}{\mathbb C}
\newcommand{\Q}{\mathbb Q}

\newcommand{\Ucurl}{\mathcal{U}}
\newcommand{\closure}[1]{\overline{#1}}
\newcommand{\setcomp}[1]{{#1}^{\mathsf{c}}}
% matematične operatorje deklarirajte kot take, da jih bo Latex pravilno stavil
% \DeclareMathOperator{\conv}{conv}

\DeclareMathOperator{\interior}{int}
\DeclareMathOperator{\pr}{pr}

% vstavite svoje definicije ...
%  \newcommand{}{}

\begin{document}

\section{Uvod}
Topološke grupe so relativno novo, a bogato področje matematike. Od razvoja pojma grupe iz Galoisovega preučevanja polinomskih enačb v prvi polovici 19. stoletja in prve formalizacije topološkega prostora v samem začetku 20. stoletja ni trajalo dolgo, da sta se pojma združila v skupno strukturo. Struktura topoloških grup je bila obsežno preučevana v letih med $1925$ in $1940$ in je še danes plodovito področje. Topološke grupe se pojavljajo na mnogih področjih. Najbolj znana so teorija Lievih grup, algebraična topologija in predvsem harmonična analiza.

Vendar pa namen tega diplomskega dela ni uporaba topoloških grup, temveč predstavitev pojma kot takega. Najprej bomo v kratkem poglavju ponovili osnove teorije grup in splošne topologije. Nato bomo definirali topološko grupo, si ogledali njene podstrukture in dokazali nekaj osnovnih izrekov o topoloških grupah. Najbolj pomembna za nas bo ugotovitev, da sta za topološko grupo regularnost in separacijski aksiom $T_0$ ekvivalentna. V naslednjem razdelku bomo pokazali, da tako kot za grupe tudi za topološke grupe ob dodatnih topoloških predpostavkah veljajo trije izreki o topoloških izomorfizmih. Zadnja dva razdelka sta namenjena metrizabilnosti in višjim separacijskim aksiomom. Najprej bomo pokazali, kdaj natanko je Hausdorffova topološka grupa metrizabilna, nato bomo definirali pojem povsem regularnosti in dokazali, da sta za topološko grupo regularnost in povsem regularnost ekvivalentna pojma. V zadnjem poglavju bomo navedli primer povsem regularne topološke grupe, ki ni normalna. Proti koncu bomo definirali pojem parakompaktnosti in z njim dokazali, da so lokalno kompaktne Hausdorffove topološke grupe normalni topološki prostori.

\section{Preliminarna poglavja}
V tem razdelku bomo ponovili že znane pojme iz algebre in splošne topologije, ki nam bodo kasneje prišli prav. Dogovorili se bomo tudi o zapisu operacij na množicah. Navedli bomo nekaj definicij in trditev, za katere privzemamo, da so bralcu že poznane.

\subsection{Operacije na množicah}\label{sec:opnamnozicah}
Vse operacije na množicah, če ne bo drugače za\-zna\-mo\-va\-no, delujejo na elementih. Tako je na primer produkt množic $U$ in $V$ enak \[U \cdot V = \lbrace u \cdot v ; u \in U, v \in V \rbrace, \] inverz množice $U$ pa je \[ U^{-1} = \lbrace u^{-1} ; u \in G \rbrace. \] Tukaj v obeh primerih predpostavimo, da so množice vložene v neki grupi, kjer so operacije na elementih smiselno definirane. Grupno strukturo bomo bolj podrobno opisali v naslednjem podrazdelku.

Pomembnejša izjema temu pravilu so operacije na množicah v smislu relacij. Predpostavimo torej, da imamo množico $X$ in nas zanimajo podmnožice kartezičnega produkta $X \times X$. Inverz take množice $U$ je definiran kot \[ U^{-1} = \lbrace (y, x) ; (x, y) \in U \rbrace, \]
analogna operacija množenju pa je kompozitum množic \[ V \circ U = \lbrace (x, z) ; \text{ obstaja tak element } y \in X, \text{ da je } (x, y) \in V \text{ in } (y, z) \in U \rbrace. \]
Takšni notaciji operacij bosta vedno posebej označeni.

\subsection{Teorija grup}
V tem podrazdelku bomo ponovili nekaj osnovnih algebraičnih pojmov, predvsem iz teorije grup.

Neprazna množica $G$ z binarno operacijo $*$ je \emph{grupa}, če:
\begin{enumerate}
\item je zaprta za operacijo $*$,
\item je operacija $*$ asociativna na množici $G$,
\item v $G$ obstaja tak element $e$ (imenujemo ga \emph{enota}), da za vsak element $x$ množice $G$ velja \[ x*e = e*x = x, \]
\item za vsak element $x$ množice $G$ obstaja tak element $y \in G$, da velja \[ x*y = y*x = e. \]
\end{enumerate}
Oznaka za grupo je ($G$, $*$) ali samo $G$, če je operacija znana ali drugače očitna. Od tukaj naprej bo zapis operacije vedno multiplikativen, razen če bo drugače povdarjeno. To pomeni, da bo grupna operacija označena s $\cdot$ ali pa bo izpuščena.

Med grupami lahko definiramo nekaj tipov preslikav. Našteli bomo dva, ki ju bomo v nadaljevanju najbolj potrebovali.
Preslikava $f\colon G \to \widetilde{G}$ je \emph{homomorfizem} grup, če za vsaka dva elementa $a, b \in G$ velja $f(a\cdot b) = f(a)\cdot f(b)$.
Preslikava je \emph{izomorfizem} grup, če je bijektivna in homomorfizem grup.

V nadaljevanju si bomo ogledali nekaj podstruktur grupe.
Podmnožica $H$ grupe $G$ je \emph{podgrupa}, če je tudi sama grupa za isto operacijo.
Množicama $aH = \lbrace ah ; h \in H \rbrace$ in $Ha = \lbrace ha ; h \in H \rbrace$ zaporedoma pravimo \emph{levi} in \emph{desni odsek} grupe $G$ elementa $a \in G$ po podgrupi $H$.

Podgrupi $H$ grupe $G$ rečemo podgrupa \emph{edinka}, če za vsak element $a \in G$ velja \[aHa^{-1} \subseteq H.\]
Množici $G/H = \lbrace aH ; a \in G \rbrace$ rečemo \emph{kvocientna množica} grupe $G$ po podgrupi $H$.
\emph{Naravna preslikava} na kvocientno množico $G/H$ je preslikava $\varphi\colon G \to G/H$, definirana s predpisom $a \mapsto aH$.
Kvocientna množica v splošnem ni grupa, razen v primeru, ko je $H$ podgrupa edinka.
Če je $N$ podgrupa edinka grupe $G$, je kvocientna množica $G/N$ grupa za operacijo $*$, kjer je $aN*bN = (a\cdot b)N$, naravna preslikava $\varphi$ pa je homomorfizem grup, ki ji rečemo \emph{naravni homomorfizem}.

\subsection{Topološki prostori}
V tem podrazdelku bomo ponovili nekaj pojmov iz splo\-šne topologije, ki jih bomo kasneje podrobneje obravnavali na topoloških grupah.

\emph{Topologija} na neprazni množici $X$ je neprazna družina podmnožic $\tau \subseteq 2^X$ z lastnostmi:
\begin{enumerate}
\item $X \in \tau$, $\emptyset \in \tau$,
\item za poljubni dve množici $U,V \in \tau$ je tudi presek $U \cap V \in \tau$,
\item za poljubno poddružino $\lbrace U_{\lambda} \rbrace_{\lambda \in \Lambda} \subseteq \tau$ je tudi unija $\bigcup\limits_{\lambda \in \Lambda}^{} U_{\lambda} \in \tau$.
\end{enumerate}
Množici $X$, opremljeni s topologijo $\tau$, rečemo \emph{topološki} prostor, ki ga označimo z $(X, \tau)$. Množice v družini $\tau$ imenujemo \emph{odprte} množice v topološkem prostoru $X$, \emph{zaprte} množice pa definiramo kot komplemente odprtih množic glede na množico $X$.

Družina $\mathcal{B}$ je \emph{baza} za topologijo $\tau$, če je vsaka množica iz topologije $\tau$ unija nekaterih množic iz $\mathcal{B}$, družina $\mathcal{P}$ pa je \emph{podbaza} za topologijo $\tau$, če je družina vseh končnih presekov množic iz $\mathcal{P}$ baza za topologijo $\tau$.

Množica $U \subseteq X$ je \emph{okolica} za točko $x \in X$, če obstaja taka odprta množica $V \in \tau$, da velja $x \in V \subseteq U$. Na podoben način lahko definiramo okolico dane množice.
Množica $U \subseteq X$ je okolica množice $A \subseteq X$, če obstaja taka odprta množica $V \in \tau$, da velja $A \subseteq V \subseteq U$.
Družina okolic $\Ucurl_x$ točke $x \in X$ se imenuje \emph{baza okolic} za $x$, če za poljubno okolico $V$ točke $x$ velja, da obstaja tak $U \in \Ucurl_x$, da je $U \subseteq V$.

Točka $a \in A$ je \emph{notranja točka} množice $A$, če je $A$ okolica za točko $a$.
\emph{Notranjost} množice $A$ je množica vseh njenih notranjih točk. Notranjost množice označimo z $\interior(A)$. Očitno velja $\interior(A) \subseteq A$ in da je $\interior(A) = A$ natanko tedaj, ko je $A \in \tau$.
\emph{Zaprtje} množice $A$ je najmanjša zaprta množica v $X$, ki vsebuje $A$. Zaprtje množice označimo z $\closure{A}$. Očitno velja $A \subseteq \closure{A}$ in da je $\closure{A} = A$ natanko tedaj, ko je $A$ zaprta množica.

S pomočjo odprtih in zaprtih množic topološkega prostora $X$ lahko sedaj de\-fi\-ni\-ra\-mo zveznost in odprtost preslikav med topološkimi prostori ter pojem ho\-me\-o\-mor\-fiz\-ma.

Naj bo $f\colon (X, \tau_1) \to (Y, \tau_2)$ preslikava med topološkima prostoroma.
Preslikava $f$ je \emph{zvezna}, kadar je praslika vsake odprte množice v topološkem prostoru $(Y, \tau_2)$ s preslikavo $f$ odprta tudi v topološkem prostoru $(X, \tau_1)$.
Preslikava $f$ je \emph{odprta}, kadar je slika vsake odprte množice v topološkem prostoru $(X, \tau_1)$ s preslikavo $f$ odprta tudi v topološkem prostoru $(Y, \tau_2)$.
Preslikava $f$ je \emph{homeomorfizem}, če je bijektivna, zvezna in ima zvezen inverz.

Osnovna podstruktura topološkega prostora je topološki podprostor.
Najprej vzemimo topološki prostor $X$ s topologijo $\tau$ in množico $A \subseteq X$. \emph{Inducirana} ali \emph{relativna topologija} na množici $A$, inducirana s topologijo $\tau$, je družina množic $\lbrace A \cap U ; U \in \tau \rbrace$. Prostoru $A$ rečemo \emph{topološki podprostor} prostora $X$.

Oglejmo si še produkt topoloških prostorov.
Naj bosta $X$ in $Y$ topološka prostora s topologijama $\tau_1$ in $\tau_2$. \emph{Produktna topologija} na kartezičnemu produktu $X \times Y$ je topologija, generirana z bazo $\lbrace U \times V ; U \in \tau_1, V \in \tau_2 \rbrace$. \emph{Produkt} topoloških prostorov $X$ in $Y$ je topološki prostor $X \times Y$, opremljen s produktno topologijo. Na produktu topoloških prostorov lahko definiramo projekcijski preslikavi $\pr_x\colon X \times Y \to X$, $\pr_x(x, y) = x$ in $\pr_y\colon X \times Y \to Y$, $\pr_y(x, y) = y$, ki sta zvezni in odprti preslikavi glede na primerne topologije.

V nadaljevanju si bomo ogledali nekaj pojmov povezanih s kompaktnostjo to\-po\-loš\-kih prostorov, ki si jih bomo kasneje ogledali v kontekstu topoloških grup, definirali pa bomo tudi nove.
Družini množic $\mathcal{A}$ rečemo \emph{pokritje} topološkega prostora $X$, če je $X = \bigcup \mathcal{A}$, družini $\mathcal{B} \subseteq \mathcal{A}$ pa rečemo \emph{podpokritje} topološkega prostora $X$, če je $\mathcal{B}$ tudi sama pokritje za $X$.
Topološki prostor je \emph{kompakten}, če vsako njegovo odprto pokritje, tj. pokritje z odprtimi množicami, vsebuje kakšno končno podpokritje.
Topološki prostor je \emph{lokalno kompakten}, če ima vsaka točka $x \in X$ kakšno kompaktno okolico.

Ponovimo še do sedaj obravnavane separacijske aksiome.
Topološki prostor $(X, \tau)$ zadošča separacijskemu aksiomu
\begin{enumerate}
\item[] $T_0$, če za poljubni različni točki $a, b \in X$ obstaja okolica $V$ za eno od točk $a, b$, ki ne vsebuje druge.
\item[] $T_1$, če za poljubno točko $a \in X$ in točko $b \in X\setminus\lbrace a \rbrace$ obstaja okolica $V$ točke $a$, ki ne vsebuje točke $b$.
\item[] $T_2$, če za poljubni različni točki $a, b \in X$ obstajata disjunktni okolici za točki $a$ in $b$.
\item[] $T_3$, če za poljubno zaprto množico $A \subseteq X$ in točko $b \in X\backslash A$ obstajata disjunktni okolici za množico $A$ in točko $b$.
\item[] $T_4$, če za poljubni disjunktni zaprti množici $A, B \subseteq X$ obstajata disjunktni okolici za množici $A$ in $B$.
\end{enumerate}

Topološkemu prostoru, ki zadošča separacijskemu aksiomu $T_2$, pravimo \emph{Hausdorffov} topološki prostor. Iz zgornje definicije je razvidno, da vsak Hausdorffov topološki prostor zadošča separacijskemu aksiomu $T_1$ in s tem tudi $T_0$.
Topološkemu prostoru, ki zadošča $T_1$ in $T_3$ pravimo \emph{regularen},
topološkemu prostoru, ki zadošča $T_1$ in $T_4$, pa \emph{normalen} topološki prostor.

Separacijske aksiome v povezavi z metrizabilnostjo na topoloških grupah si bomo podrobneje ogledali v kasnejših poglavjih.

\section{Kaj je topološka grupa}
V tem razdelku bomo združili pojem topološkega prostora s pojmom grupe in ju povezali v pojem topološke grupe. Ogledali si bomo nekaj primerov in temeljnih lastnosti topološke grupe.

\subsection{Definicija topološke grupe}
Iz definicije grupe je razvidno, da nam grupna struktura na množici porodi dve strukturni preslikavi:
\begin{itemize}
\item \emph{množenje} $\mu\colon G \times G \to G$, $(x, y) \mapsto xy$,
\item \emph{invertiranje} $\iota\colon G \to G$, $x \mapsto x^{-1}$.
\end{itemize}

S pomočjo strukturnih preslikav bomo sedaj definirali topološko grupo.
\begin{definicija}\label{def:topgrupa}
\emph{Topološka grupa} je grupa $G$ opremljena s takšno topologijo $\tau$, da sta glede na $\tau$ strukturni preslikavi množenja in invertiranja zvezni. 
\end{definicija}

Potrebujemo še tip preslikave med topološkimi grupami, ki bo ohranjal tako algebraično kot topološko strukturo.
\begin{definicija}\label{def:topizo}
Preslikava med dvema topološkima grupama je \emph{to\-po\-loš\-ki izo\-mor\-fi\-zem}, če je izomorfizem grup in homeomorfizem topoloških prostorov.
\end{definicija}


\subsection{Primeri topoloških grup}
Oglejmo si nekaj osnovnih primerov topoloških grup.
\begin{primer}
	Vsaka grupa $G$ je za diskretno topologijo $\tau_d = 2^G$ in trivialno topologijo $\tau_t = \lbrace \emptyset, G \rbrace$ topološka grupa, saj je glede na njiju zvezna vsaka preslikava na $G$ ali $G \times G$.
\end{primer}

\begin{primer}
	Realna števila za operacijo $+$ so topološka grupa z evklidsko topologijo.
	
	Preverimo, da sta strukturni preslikavi $\mu(x, y) = x + y$ in $\iota(x) = -x$ zvezni.
	Po definiciji produktne topologije sta projekcijski preslikavi zvezni. Strukturna preslikava seštevanja $\mu = \pr_x + \pr_y$ je zvezna kot vsota zveznih preslikav.
	
	Zveznost invertiranja je dovolj preveriti na baznih množicah evklidske topologije. Vzemimo interval $(a, b)$. Tudi $\iota^{-1}((a, b)) = \iota((a, b)) = (-b, -a)$ je bazična množica in zato odprta. Torej je invertiranje zvezna preslikava.
\end{primer}

\begin{primer}
	Spomnimo se, da je $\C \cong \R^2$, zato je
	enotska krožnica v kompleksni ravnini $S = \lbrace z \in \C ; |z| = 1 \rbrace$ s podedovanim množenjem topološka grupa za relativno topologijo kot podprostor $\R^2$.
\end{primer}

\begin{primer}\label{pri:haus}
	Realna števila za operacijo $+$ niso topološka grupa s topologijo $\tau = \lbrace (a, \infty) ; a \in \R \rbrace$. Vzemimo odprto množico $(a, \infty)$ in si oglejmo njeno prasliko glede na preslikavo invertiranja. Ker $\iota^{-1}((a, \infty)) = (-\infty, -a)$ ni odprta množica v topologiji $\tau$, prostor $(\R, \tau)$ ni topološka grupa. Kasneje bomo pokazali, da realna števila s to topologijo niso topološka grupa za nobeno operacijo.
\end{primer}

\subsection{Osnovne lastnosti topoloških grup}
Najprej ponovimo nekaj preslikav iz teorije grup, ki jih bomo skozi celotno delo pogosto uporabljali. \emph{Leva} in \emph{desna translacija} za element $a \in G$ sta zaporedoma definirani s predpisoma $l_a\colon x \mapsto ax$ in $r_a\colon x \mapsto xa$. \emph{Konjugiranje} z elementom $a \in G$ je definirano s predpisom $\gamma_a\colon x \mapsto axa^{-1}$. Naslednja trditev nam pove, zakaj bomo te preslikave v nadaljevanju veliko uporabljali.

\begin{trditev}\label{trd:trans}
Naj bo $G$ topološka grupa in $a \in G$.
\begin{enumerate}
\item Leva in desna translacija za element $a$ sta homeomorfizma iz $G$ v $G$.\label{podtrd:ldt}
\item Invertiranje je homeomorfizem iz $G$ v $G$.
\item Konjugiranje je homeomorfizem iz $G$ v $G$.
\end{enumerate}
\end{trditev}

\begin{dokaz}
Vemo že, da so leva translacija, desna translacija, invertiranje in konjugiranje bijektivne preslikave. Dokazujemo še zveznost pre\-sli\-ka\-ve in njenega inverza.

Ker je konstantna preslikava $\text{c}_a\colon x \mapsto a$ zvezna in ker
lahko levo translacijo zapišemo kot kompozitum zveznih preslikav \[l_a(x) = \mu(\text{c}_a(x), x) = ax,\]
je leva translacija zvezna. Inverzna preslikava levi translaciji za element $a$ je leva translacija za element $a^{-1}$, ki je tudi zvezna. Leva translacija je zato homeomorfizem.

Za dokaz zveznosti desne translacije najprej opazimo, da velja
\[r_a(x) = \mu(x, \text{c}_a(x)) = xa,\]
nato pa sledimo zgornjemu dokazu.

Invertiranje je homeomorfizem, ker je zvezno po definiciji topološke grupe in samo sebi inverz.

Konjugiranje lahko zapišemo kot kompozitum homeomorfizmov \[\gamma = r_{a^{-1}} \circ l_a,\]
nato pa uporabimo točko (\ref{podtrd:ldt}).
\end{dokaz}

\begin{trditev}\label{trd:prododp}
Naj bo $A$ odprta podmnožica topološke grupe $G$. Tedaj sta za vsako množico $B \subseteq G$ množici $AB$ in $BA$ odprti.
\end{trditev}

\begin{dokaz}
Ker je po trditvi \ref{trd:trans} desna translacija homeomorfizem, so odprte tudi vse množice $Ax$ za vsak $x \in G$. Ker je $AB = \bigcup_{b \in B}Ab$, je množica $AB$ odprta kot unija odprtih množic.

Za dokaz odprtosti množice $BA$ opazimo, da so po trditvi \ref{trd:trans} odprte tudi vse množice $xA$ za vsak $x \in X$, potem pa sledimo zgornjemu dokazu.
\end{dokaz}

\begin{trditev}\label{trd:prodkomp}
Za kompaktni podmnožici $A$ in $B$ topološke grupe $G$ velja, da je množica $AB$ kompaktna.
\end{trditev}

\begin{dokaz}
Iz splošne topologije vemo, da je kompaktnost končno multiplikativna lastnost. Množica $A \times B$ je zato kompaktna podmnožica v prostoru $G \times G$. Ker je operacija množenja $\mu$ zvezna preslikava, je množica $\mu (A \times B) = AB$ kompaktna v prostoru $G$.
\end{dokaz}

Sledi nekaj trditev, ki se nanašajo na okolice enote topološke grupe. Dokazali bomo nekaj lastnosti baze odprtih okolic enote topološke grupe, nato pa pokazali kaj mora veljati, da je družina podmnožic topološke grupe baza njene topologije.

\begin{trditev}\label{trd:okolice}
Za topološko grupo $G$ in bazo $\Ucurl$ odprtih okolic enote $e$ veljajo naslednje trditve:
\begin{enumerate}
\item za vsako množico $U \in \Ucurl$ obstaja taka množica $V \in \Ucurl$, da velja $V^{2} \subset U$;\label{last:oko1}
\item za vsako množico $U \in \Ucurl$ obstaja taka množica $V \in \Ucurl$, da velja $V^{-1} \subset U$;\label{last:oko2}
\item za vsako množico $U \in \Ucurl$ in vsak element $x \in U$ obstaja taka množica $V \in \Ucurl$, da velja $xV \subset U$; \label{last:oko3}
\item za vsako množico $U \in \Ucurl$ in vsak element $x \in G$ obstaja taka množica $V \in \Ucurl$, da velja $xVx^{-1} \subset U$.\label{last:oko4}
\end{enumerate}
\end{trditev}

\begin{dokaz}
Naj bo $U \in \Ucurl$  odprta okolica enote $e$. Ker je množenje zvezno, obstaja v produktni topologiji na $G \times G$ bazična okolica $W = V_1 \times V_2$ elementa $(e, e)$, za katero velja $V_1V_2 \subset U$. Po definiciji produktne topologije sta $V_1$ in $V_2$ odprti okolici enote $e$ v $G$. Tedaj je $V' = V_1 \cap V_2$ okolica za $e$. Po definiciji baze okolic obstaja $V \in \Ucurl$, da velja $V \subseteq V'$. Ker je $V \subseteq V' \subseteq V_1$ in $V \subseteq V' \subseteq V_2$, velja \[V^2 \subseteq V'^2 \subseteq V_1V_2 \subset U.\] To dokaže prvo trditev.

Ker je invertiranje zvezno, obstaja v $G$ odprta okolica $W$ enote $e$, za katero velja $W^{-1} \subset U$. Po definiciji baze okolic obstaja okolica $V \in \Ucurl$, za katero velja $V \subseteq W$. Potem je $V^{-1} \subseteq W^{-1}$. Velja \[V^{-1} \subset W^{-1} \subset U.\] To dokaže drugo trditev.

Vzemimo poljubno točko $x \in U$. Naj bo $W = x^{-1}U$. Ker je po trditvi \ref{trd:trans} leva translacija homeomorfizem, je $W$ odprta okolica enote $e$. Če vzamemo $V \in \Ucurl$, $V \subset W$, ki obstaja po definiciji baze okolic, velja \[xV \subset xW = xx^{-1}U = U,\]
kar dokaže tretjo trditev.

Naj bo $x \in U$ poljubna točka. Ker je po trditvi \ref{trd:trans} konjugiranje homeomorfizem, je množica $W = x^{-1}Ux$ odprta okolica enote $e$. Če vzamemo $V \in \Ucurl$, $V \subset W$, ki obstaja po definiciji baze okolic, velja
\[ xVx^{-1} \subset xWx^{-1} = xx^{-1}Uxx^{-1} = U, \]
kar dokaže še četrto trditev.
\end{dokaz}

\begin{izrek}\label{izr:bazaokolice}
Naj bo $G$ grupa in $\Ucurl$ družina podmnožic množice $G$, za katero veljajo vse štiri lastnosti iz trditve \ref{trd:okolice}. Naj bodo poljubni končni preseki množic iz $\Ucurl$ neprazni. Tedaj je družina $\lbrace xU \rbrace$, kjer $U \in \Ucurl$ in $x \in G$, odprta podbaza za neko topologijo na $G$. S to topologijo je $G$ topološka grupa. Družina $\lbrace Ux \rbrace$ je podbaza za isto topologijo.

Če velja še, da za vsaki množici $U,V \in \Ucurl$ obstaja taka množica $W \in \Ucurl$, da velja $W \subset U \cap V$, potem sta družini $\lbrace xU \rbrace$ in $\lbrace Ux \rbrace$ tudi bazi za to topologijo.
\end{izrek}

\begin{dokaz}
Za vsako množico $U \in \Ucurl$ obstaja taka množica $V \in \Ucurl$, da velja $V^2 \subset U$. Ker je $V \in \Ucurl$, obstaja taka množica $W \in \Ucurl$, da velja $W^{-1} \subset V$. Ker velja $V \cap W \neq \emptyset$, je
\[ e \in VW^{-1} \subset V^2 \subset U. \]
Vsaka množica $U \in \Ucurl$ torej vsebuje enoto $e$.
Družina $\lbrace xU ; x \in G, U \in \Ucurl \rbrace$ je torej res podbaza neke topologije na $G$, saj je pokritje za $G$.

Za vsako izbiro množic $U_1,\dots,U_n \in \Ucurl$ obstajajo take množice $V_1,\dots,V_n \in \Ucurl$, da velja $V_k^2 \subset U_k$ za $k = 1,\dots,n$. Potem velja
\[ \left( \bigcap_{k=1}^n V_k \right)^2 \subseteq \bigcap_{k=1}^n V_k^2 \subset \bigcap_{k=1}^n U_k. \]
Za družino $\widetilde{\Ucurl}$ torej velja lastnost (\ref{last:oko1}) trditve \ref{trd:okolice}.
Ker je \[\left( \bigcap_{k=1}^n V_k \right)^{-1} = \bigcap_{k=1}^n V_k^{-1},\] za družino $\widetilde{\Ucurl}$ velja lastnost (\ref{last:oko2}) trditve \ref{trd:okolice}.
Ker velja \[x\left(\bigcap_{k=1}^n V_k\right) = \bigcap_{k=1}^n (xV_k)\]
in \[ x\left(\bigcap_{k=1}^n V_k\right)x^{-1} = \bigcap_{k=1}^n (xV_kx^{-1}), \]
za družino $\widetilde{\Ucurl}$ veljata tudi lastnosti (\ref{last:oko3}) in (\ref{last:oko4}) trditve \ref{trd:okolice}.

Po definiciji podbaze vse neprazne množice oblike $\bigcap_{k=1}^n(x_kU_k)$, kjer je $x_k \in G$ in $U_k \in \Ucurl$, tvorijo bazo za neko topologijo na prostoru $G$. Vzemimo nek element $y \in \bigcap_{k=1}^n (x_kU_k)$. Naj bo $V_k \in \Ucurl$ takšna množica, ki zadošča lastnosti (\ref{last:oko3}) za element $x_k^{-1}y$, torej da velja $x_k^{-1}yV_k \subset U_k$. Potem velja
\[ y\left(\bigcap_{k=1}^nV_k\right) = \bigcap_{k=1}^n(yV_k) \subset \bigcap_{k=1}^n(x_kU_k). \]
Torej množice oblike $y\widetilde{U}$, kjer je $\widetilde{U} \in \widetilde{\Ucurl}$ tvorijo odprto bazo za element $y$ za vsak element $y \in G$.

Da pokažemo, da je $G$ res topološka grupa, vzemimo poljubna elementa $a,b \in G$ in poljubno množico $\widetilde{U} \in \widetilde{\Ucurl}$. Ker družina množic $\widetilde{\Ucurl}$ zadošča lastnostma (\ref{last:oko1}) in (\ref{last:oko4}) trditve \ref{trd:okolice}, obstajata takšni množici $\widetilde{V}, \widetilde{W} \in \widetilde{\Ucurl}$, da velja $(b^{-1}\widetilde{W}b)\widetilde{V} \subset \widetilde{U}$. Od tod sledi, da je $(a\widetilde{W})(b\widetilde{V}) \subset ab\widetilde{U}$, iz česar sledi, da je operacija množenja zvezna preslikava.
Ker družina množic $\widetilde{\Ucurl}$ zadošča lastnostma (\ref{last:oko2}) in (\ref{last:oko4}) trditve \ref{trd:okolice}, obstaja takšna množica $\widetilde{V} \in \widetilde{\Ucurl}$, da velja $(b\widetilde{V}b^{-1})^{-1} \subset \widetilde{U}$. Od tod sledi, da je $b^{-1}(b\widetilde{V}b^{-1})^{-1} = b^{-1}b (b\widetilde{V})^{-1} = (bV)^{-1} \subset b^{-1}\widetilde{U}$, torej je invertiranje zvezna preslikava. Topološki prostor $G$ je torej topološka grupa.

Iz lastnosti (\ref{last:oko4}) trditve \ref{trd:okolice} je tudi razvidno, da družini $\lbrace xU \rbrace$ in $\lbrace Ux \rbrace$ porodita ekvivalentno topologijo na $G$, saj lahko vsako množico oblike $Ux$ dobimo s konjugiranjem množice $xU$ z elementom $x^{-1}$, konjugiranje pa je po trditvi \ref{trd:trans} homeomorfizem.
\end{dokaz}

\begin{definicija}\label{def:sim}
Množici, za katero velja $U = U^{-1}$, rečemo \emph{simetrična} množica.
\end{definicija}

\begin{trditev}\label{trd:sim}
Vsaka topološka grupa ima bazo $\Ucurl$ odprtih in simetričnih okolic enote.
\end{trditev}

\begin{dokaz}
Naj bo $\mathcal{V}$ neka baza odprtih okolic enote. Za vsako okolico $V \in \mathcal{V}$ definiramo množico $U = V \cap V^{-1}$. Kot presek dveh odprtih množic je $U$ odprta. Ker je $e \in V$ in $e \in V^{-1}$, je $U$ odprta okolica enote, ki je po konstrukciji simetrična. Ker po definiciji preseka velja še $U \subseteq V$, je družina $\Ucurl = \lbrace V \cap V^{-1}; V \in \mathcal{V} \rbrace$ res baza odprtih in simetričnih okolic enote $e$.
\end{dokaz}

Zlahka vidimo, da sta za topološko grupo separacijska aksioma $T_0$ in $T_2$ ekvivalentna, kakor pravi naslednja trditev. Regularnosti in višjim separacijskim aksiomom se bomo posvetili kasneje.
\begin{trditev}\label{trd:t0haus}
Za topološko grupo $G$ so naslednje trditve ekvivalentne.
\begin{enumerate}
	\item Topološka grupa $G$ zadošča separacijskemu aksiomu $T_0$.\label{podtrd:h1}
	\item Množica $\lbrace e \rbrace$ je zaprta v $G$.\label{podtrd:h2}
	\item Topološka grupa $G$ je Hausdorffov topološki prostor.\label{podtrd:h3}
\end{enumerate}
\end{trditev}

\begin{dokaz}
	(\ref{podtrd:h1}) $\Rightarrow$ (\ref{podtrd:h2}):
	Implikacijo bomo dokazali tako, da bomo pokazali, da je množica $G\setminus\lbrace e \rbrace$ okolica za vsako svojo točko, iz česar bo sledilo, da je odprta.
	Vzemimo točko $x \in G\setminus\lbrace e \rbrace$. Ker $G$ zadošča separacijskemu aksiomu $T_0$, obstaja bodisi okolica za točko $x$, ki ne vsebuje enote $e$, bodisi okolica $V$ za enoto $e$, ki ne vsebuje točke $x$. V prvem primeru sledi, da je $G\setminus\lbrace e \rbrace$ okolica za točko $x$.
	Zato lahko brez škode za splošnost predpostavimo, da obstaja okolica $V$ enote $e$, ki ne vsebuje točke $x$. Množica $x^{-1}V$ je potem okolica za točko $x^{-1}$, ki ne vsebuje enote $e$, zato je $\iota(x^{-1}V)$ okolica za točko $x$, ki ne vsebuje enote $e$. Ker velja $\iota(x^{-1}V) \subseteq G \setminus \lbrace e \rbrace$, je $G\setminus\lbrace e \rbrace$ okolica za točko $x$.
	Množica $G\setminus\lbrace e \rbrace$ je torej okolica za vsako svojo točko in je zato odprta. Sledi, da je $\lbrace e \rbrace$ zaprta množica.
	
	
	(\ref{podtrd:h2}) $\Rightarrow$ (\ref{podtrd:h3}):
	Privzemimo, da je $\lbrace e \rbrace$ zaprta množica. Oglejmo si preslikavo $f\colon G \times G \to G$, $(x, y) \mapsto xy^{-1}$. Preslikava $f$ je zvezna kot kompozitum množenja in invertiranja, ki sta zvezni preslikavi po definiciji topološke grupe. Zato je $f^{-1}(\lbrace e \rbrace) = \lbrace (x, x) ; x \in G \rbrace$ zaprta množica v $G \times G$. Po izreku iz splošne topologije je $G$ Hausdorffov topološki prostor.
	
	(\ref{podtrd:h3}) $\Rightarrow$ (\ref{podtrd:h1}):
	Sledi iz definicije separacijskega aksioma $T_2$.
\end{dokaz}

Z naslednjo trditvijo si oglejmo še lastnosti operatorja zaprtja na podmnožicah topološke grupe.
\begin{trditev}\label{trd:zaprtost}
	Za podmnožici $A$ in $B$ topološke grupe $G$ veljajo naslednje trditve:
	\begin{enumerate}
		\item $\closure{A}\ \closure{B} \subset \closure{A B}$,\label{podtrd:zap1}
		\item $(\closure{A})^{-1} = \closure{A^{-1}}$,\label{podtrd:zap2}
		\item $x \closure{A} y = \closure{x A y}$ za vsaka dva elementa $x, y \in G$.\label{podtrd:zap3}
		
		\item Če $G$ ustreza separacijskemu aksiomu $T_0$ in za vsaka dva elementa $a \in A$ in $b \in B$ velja enakost $ab = ba$, potem velja enakost $ab = ba$ tudi za vsaka dva elementa $a \in \closure{A}$ in $b \in \closure{B}$.\label{podtrd:zap4}
	\end{enumerate}
\end{trditev}

\begin{dokaz}
	Naj bosta $A$ in $B$ podmnožici topološke grupe $G$. Spomnimo se, da je poljubna preslikava $f$ zvezna natanko tedaj, ko velja $f(\closure{A}) \subseteq \closure{f(A)}$. Če je preslikava homeomorfizem, velja enačaj.
	
	Za dokaz točke (\ref{podtrd:zap1}) upoštevamo le, da je operacija množenja zvezna preslikava po definiciji topološke grupa. Potem po zgornjem velja
	\[ \closure{A}\;\closure{B} = \mu(\closure{A}\times\closure{B}) \subseteq \closure{\mu(A\times B)} = \closure{AB}\]
	
	Enakost v točki (\ref{podtrd:zap2}) bo sledila iz tega, da je invertiranje po trditvi \ref{trd:trans} homeomorfizem. Vzemimo množico $A \subset G$. Ker je invertiranje homeomorfizem, je $\iota(\closure{A}) = \closure{\iota(A)}$. Torej je $\closure{A}^{-1} = \closure{A^{-1}}$.
	
	Enakost v točki (\ref{podtrd:zap3}) bo sledila iz tega, da sta leva in desna translacija po trditvi \ref{trd:trans} homeomorfizma. Naj bosta $x, y \in G$. Tedaj je tudi $f = r_y \circ l_x$ homeomorfizem. Velja torej
	\[ x\closure{A}y = f(\closure{A}) = \closure{f(A)} = \closure{xAy}. \]
	
	Za dokaz točke (\ref{podtrd:zap4}) privzemimo še, da $G$ zadošča separacijskemu aksiomu $T_0$ in da velja $ab = ba$ za vsaka dva elementa $a \in A$ in $b \in B$. Preslikava $f\colon (a,b) \mapsto aba^{-1}b^{-1}$ je zvezna, saj je kompozitum množenj in invertiranj:
	\[ f(a, b) = \mu(\mu(a, b),\mu(\iota(a), \iota(b))). \]
	Ker je po trditvi \ref{trd:t0haus} množica $\lbrace e \rbrace$ zaprta, je zaprta tudi množica $H = f^{-1}(\lbrace e \rbrace) = \lbrace (a, b) \in G \times G; aba^{-1}b^{-1} = e \rbrace$. Po predpostavki velja $A \times B \subseteq H$.
	Ker je $\closure{A}\times\closure{B} = \closure{A \times B}$ in po predpostavki velja $A \times B \subseteq H$, je $\closure{A}\times\closure{B} \subseteq H$, torej velja $ab = ba$ za vsaka dva elementa $a \in \closure{A}$ in $b \in \closure{B}$.
\end{dokaz}

Še enkrat si oglejmo primer \ref{pri:haus}.
\begin{primer}\label{pri:haus2}
	Zgoraj smo pokazali, da strukturna preslikava invertiranja ni zvezna, v kar pa se lahko prepričamo tudi drugače.
	
	Po trditvi \ref{trd:t0haus} za vsako topološko grupo velja, da zadošča separacijskemu aksiomu $T_0$ natanko tedaj, ko je Hausdorffova. Topološki prostor $(\R, +, \tau)$ zadošča separacijskemu aksiomu $T_0$, saj je za vsaki dve točki $a < b$ množica $(\frac{a+b}{2}, \infty)$ okolica za točko $b$, ki ne vsebuje točke $a$. Hkrati pa je očitno, da ta prostor ni Hausdorffov, saj se vsaki dve neprazni odprti množici sekata. Topološki prostor $(\R, +, \tau)$ torej ne more biti topološka grupa. Še več, topološki prostor $(\R, \tau)$ ni topološka grupa za nobeno operacijo, saj informacije o operaciji v zgornjem premisleku sploh nismo uporabili.
\end{primer}

\begin{primer}
Naj bo $G$ poljubna neskončna grupa in naj bo topologija $\tau$ topologija končnih komplementov, tj. $\tau = \lbrace U \subseteq G ; |G\setminus U| < \infty \rbrace$.
Iz splošne topologije vemo, da je to najšibkejša topologija na $G$, da topološki prostor $G$ zadošča separacijskemu aksiomu $T_1$, ne pa tudi $T_2$. Od tod po enakem razmisleku kot v primeru \ref{pri:haus2} sledi, da topološki prostor $(G, \tau)$ ni topološka grupa za nobeno operacijo.
\end{primer}

\section{Topološke podgrupe in kvocientne grupe}
\subsection{Topološke podgrupe}
V tem podrazdelku si bomo ogledali podgrupe topoloških grup. Najprej bomo pokazali, da je podgrupa topološke grupe tudi topološka grupa in ji rečemo topološka podgrupa, nato pa si bomo ogledali nekaj njenih zanimivih lastnosti.
\begin{trditev}\label{trd:toppodgrupa}
Naj bo $G$ topološka grupa in $H$ njena podgrupa. Če $H$ opremimo z relativno topologijo, potem je tudi $H$ topološka grupa.
\end{trditev}

\begin{dokaz}
Preslikavi $\mu|_{H \times H}$ in $\iota|_H$ sta zvezni glede na relativno topologijo na $H$ kot zožitvi zveznih preslikav na topološki podprostor $H$. Zato je $H$ topološka grupa.
\end{dokaz}

\begin{trditev}\label{trd:odpzap}
Podgrupa $H$ topološke grupe $G$ je odprta natanko tedaj, ko ima ne\-praz\-no not\-ran\-jost. Vsaka odprta podgrupa $H$ topološke grupe $G$ je tudi zaprta.
\end{trditev}

\begin{dokaz}
Denimo, da obstaja element $x \in \interior(H)$. Potem obstaja taka okolica $U$ enote $e$, da je $xU \subset H$. Dokazali bomo, da velja $H \subseteq \interior(H)$. Ker za poljuben $y \in H$ velja \[yU = yx^{-1}xU \subset yx^{-1}H = H,\] je $y \in \interior(H)$. Torej je vsaka točka v podgrupi $H$ notranja točka, kar pomeni, da je $H$ odprta množica.

Obratno, če je $H$ odprta, vsaka njena točka leži tudi v njeni notranjosti. Torej ima $H$ neprazno notranjost.

Privzemimo, da je $H$ odprta podgrupa grupe $G$. Ker je $H$ podgrupa, je $G\setminus H = \bigcup \lbrace xH ; x \notin H \rbrace$. Ker je $H$ odprta in je po trditvi \ref{trd:trans} leva translacija homeomorfizem, je tudi vsaka množica $xH$ odprta. Potem je tudi $G \setminus H$ odprta kot unija odprtih množic, torej je $H$ zaprta množica.
\end{dokaz}

\begin{trditev}\label{trd:podgrupaunija}
Naj bo $U$ simetrična okolica enote $e$ v topološki grupi $G$. Potem je $L = \bigcup_{n=1}^{\infty} U^n$ odprta in zaprta podgrupa topološke grupe $G$.
\end{trditev}

\begin{dokaz}
Ker za $x \in U^k$ in $y \in U^l$ velja $xy \in U^kU^l \subseteq U^{k+l}$, je $L$ zaprta za množenje. Ker je $U$ simetrična, velja tudi $x^{-1} \in (U^{-1})^k = U^k$, torej je $L$ zaprta za invertiranje. Sledi, da je $L$ podgrupa topološke grupe $G$. V njeni notranjosti je zagotovo vsaj enota $e$, saj je $U$ okolica za $e$. Po trditvi \ref{trd:odpzap} je $L$ odprta in zaprta podgrupa topološke grupe $G$.
\end{dokaz}

\begin{posledica}
Naj bo $U$ simetrična okolica enote $e$ v povezani topološki grupi $G$. Potem je $G = \bigcup_{n=1}^{\infty} U^n$.
\end{posledica}

\begin{dokaz}
Po trditvi \ref{trd:podgrupaunija} je podgrupa $L = \bigcup_{n=1}^{\infty} U^n$ odprta in zaprta v $G$. Ker je $G$ povezana in je $L$ neprazna, je $G = L$.
\end{dokaz}

\begin{trditev}\label{trd:povedinka}
Naj bo $C$ komponenta za povezanost topološke grupe $G$, ki vsebuje enoto $e$. Potem je $C$ zaprta podgrupa edinka topološke grupe $G$.
\end{trditev}

\begin{dokaz}
Ker je po trditvi \ref{trd:trans} invertiranje homeomorfizem, je $C^{-1}$ povezana množica, ki vsebuje enoto $e$. Ker je $C$ komponenta za povezanost, je $C^{-1} \subseteq C$. Od tod sledi, da je za vsak $a \in C$ tudi $a^{-1} \in C$, zato je množica $aC$ povezana množica, ki vsebuje enoto $e$, in velja $aC \subseteq C$. Ker to velja za vsak $a \in C$, je $C^2 \subseteq C$. Množica $C$ je zaprta za invertiranje in množenje ter vsebuje enoto $e$, zato je $C$ podgrupa topološke grupe $G$. Ker je $C$ komponenta za povezanost, vemo, da je zaprta v $G$. Preverimo še, da je $C$ podgrupa edinka. Vzemimo poljuben $a \in G$. Ker je konjugiranje po trditvi \ref{trd:trans} homeomorfizem, je množica $aCa^{-1}$ povezana množica, ki vsebuje enoto $e$. Od tod sledi, da je $aCa^{-1} \subseteq C$, torej je $C$ podgrupa edinka topološke grupe $G$.
\end{dokaz}

\subsection{Kvocienti topoloških grup}
V tem podrazdelku bomo na kvocientni množici $G/H$ vpeljali kvocientno topologijo in pokazali nekaj njenih lastnosti. Pokazali bomo, da je v primeru, ko je $H$ podgrupa edinka topološke grupe $G$, tudi $G/H$ topološka grupa. S pomočjo kvocientnih prostorov bomo nato pokazali, da sta za topološko grupo separacijski aksiom $T_0$ in regularnost ekvivalentna pojma.

\begin{izrek}\label{izr:topkvocienta}
Naj bo $G$ topološka grupa, $H$ njena podgrupa in $\varphi\colon G \to G/H$ naravna preslikava. Za družino \[\theta(G/H) = \lbrace U ; \varphi^{-1}(U) \text{ odprta v } G \rbrace\]
veljajo naslednje trditve:
\begin{enumerate}
\item družina $\theta(G/H)$ je topologija na kvocientni množici $G/H$,
\item družina $\theta(G/H)$ je najmočnejša topologija na kvocientni množici $G/H$, glede na katero je $\varphi$ zvezna preslikava,
\item $\varphi: G \to G/H$ je odprta preslikava.
\end{enumerate}
\end{izrek}

\begin{dokaz}
Družina $\theta(G/H)$ vsebuje prazno množico, ker je množica $\varphi^{-1}(\emptyset) = \emptyset$ odprta v $G$. Vsebuje tudi množico $G/H$, ker je množica $\varphi^{-1}(G/H) = G$ odprta v $G$. Vzemimo poljubno unijo odprtih množic $\bigcup_{\lambda \in \Lambda}\lbrace uH ; u \in U_\lambda, U_\lambda \text{ odprta v } G\rbrace$. Ker je $\bigcup_{\lambda \in \Lambda} U_\lambda$ odprta v $G$ in velja $\varphi^{-1}(\bigcup_{\lambda \in \Lambda}\lbrace uH ; u \in U_\lambda, U_\lambda \text{ odprta v } G\rbrace) = \bigcup_{\lambda \in \Lambda} U_\lambda H$, je tudi unija $\bigcup_{\lambda \in \Lambda}\lbrace uH ; u \in U_\lambda, U_\lambda \text{ odprta v } G\rbrace \in \theta(G/H)$. Ker je za vsaki množici $U$ in $V$, ki sta odprti v $G$, tudi presek $U \cap V$ odprt v $G$, in ker je $\varphi^{-1}(\lbrace uH ; u \in U\rbrace \cap \lbrace vH ; v \in V \rbrace) = \varphi^{-1}(\lbrace uH ; u \in U \cap V \rbrace) = (U \cap V)H$, je tudi presek $\lbrace uH ; u \in U\rbrace \cap \lbrace vH ; v \in V \rbrace$ odprt v $G/H$.

Topologije $\theta(G/H)$ je po definiciji najmočnejša topologija na kvocientni množici $G/H$, glede na katero je $\varphi$ zvezna.

Za dokaz odprtosti naravne preslikave vzemimo odprto množico $U \in G$. Po trditvi \ref{trd:prododp} je množica $UH$ odprta v $G$. Ker velja $\varphi^{-1}(\lbrace uH ; u \in U \rbrace) = UH$, je $\varphi(U) = \lbrace uH ; u \in U \rbrace$ odprta v $G/H$.
\end{dokaz}

Topologiji $\theta(G/H)$ pravimo \emph{kvocientna topologija}, topološkemu prostoru $G/H$ pa \emph{kvocientni prostor}. Od tukaj naprej bo kvocientni topološki prostor vedno opremljen s kvocientno topologijo.

\begin{trditev}\label{trd:okolicevkvoc}
Naj bo $G$ topološka grupa, $H$ njena podgrupa in $U, V$ taki okolici enote $e$ v $G$, da velja $V^{-1}V \subset U$. Naj bo $\varphi: G \to G/H$ naravna preslikava. Potem velja $\closure{\varphi(V)} \subset \varphi(U)$.
\end{trditev}

\begin{dokaz}
Vzemimo odsek $xH \in \closure{\varphi(V)}$. Ker je $V$ okolica enote, je množica $\lbrace vxH ; v \in V \rbrace$ okolica odseka $xH$, in ima zato s $\varphi(V)$ neprazen presek. Po definiciji naravne preslikave obstajata točki $v_1, v_2 \in V$, da je $v_1xH = v_2H$. Ker je \[xH = v_1^{-1}v_2H \in \lbrace wH ; w \in V^{-1}V \rbrace \subset \lbrace uH ; u \in U \rbrace = \varphi(U), \]
je $\closure{\varphi(V)} \subset \varphi(U)$.
\end{dokaz}

\begin{posledica}\label{pos:sim}
Za vsako okolico $U$ enote $e$ topološke grupe $G$ obstaja taka okolica $V$ enote $e$, da velja $\closure{V} \subset U$.
\end{posledica}

\begin{dokaz}
Naj bo $U$ poljubna okolica enote $e$ in naj bo $\mathcal{V}$ baza simetričnih okolic enote $e$, ki obstaja po trditvi \ref{trd:sim}. Po trditvi \ref{trd:okolice} obstaja takšna okolica $V \in \mathcal{V}$, da je $V^2 = V^{-1}V \subset U$. Naj bo podgrupa $H = \lbrace e \rbrace$ trivialna podgrupa. Po trditvi \ref{trd:okolicevkvoc} je $\closure{\varphi(V)} \subset \varphi(U)$. Ker je $H$ trivialna podgrupa, je kvocientna množica $G/H$ enaka $G$ in naravna preslikava $\varphi\colon G \to G/H$ je identična preslikava na topološki grupi $G$. Zato velja $\closure{V} = \closure{\varphi(V)} \subset \varphi(U) = U$.
\end{dokaz}

\begin{izrek}\label{izr:kvocreg}
Za topološko grupo $G$ in njeno podgrupo $H$ veljajo naslednje trditve:
\begin{enumerate}
\item kvocientni prostor $G/H$ je diskreten natanko tedaj, ko je $H$ odprta v $G$,\label{podtrd:kvocreg1}
\item če je $H$ zaprta v $G$, potem je kvocient $G/H$ regularen topološki prostor,\label{podtrd:kvocreg2}
\item če kvocientni prostor $G/H$ zadošča separacijskemu aksiomu $T_0$, potem je $H$ zaprta v $G$, kvocientni topološki prostor $G/H$ pa je regularen.\label{podtrd:kvocreg3}
\end{enumerate}
\end{izrek}

\begin{dokaz}
Za dokaz točke (\ref{podtrd:kvocreg1}) privzemimo, da je $H$ odprta v $G$. Ker je leva translacija po trditvi \ref{trd:trans} homeomorfizem, je množica $aH$ odprta množica za vsak element $a \in G$. Ker je $\varphi^{-1}(\lbrace aH \rbrace) = aH$, je množica $\lbrace aH \rbrace$ odprta v $G/H$ za vsak element $aH \in G/H$, torej je $G/H$ diskreten topološki prostor.

Obratno, če je $G/H$ diskreten topološki prostor, potem so vse njegove eno\-e\-le\-ment\-ne podmnožice odprte. V posebnem primeru je tudi $\lbrace H \rbrace$ odprta v $G/H$. Ker je naravna preslikava zvezna, je $H = \varphi^{-1}(\lbrace H \rbrace)$ odprta množica v $G$.

Za dokaz točke (\ref{podtrd:kvocreg2}) privzemimo, da je $H$ zaprta v $G$. Ker je leva translacija po trditvi \ref{trd:trans} homeomorfizem, je zaprta tudi množica $aH$ za vsak element $a \in G$. Po definiciji zaprtosti je $G\setminus aH = \bigcup \lbrace xH ; xH \neq aH \rbrace$ odprta v $G$. Ker je po izreku \ref{izr:topkvocienta} naravna preslikava odprta, je zato komplement vsake točke $\lbrace aH \rbrace$ odprt v $G/H$. Po definiciji zaprtosti je vsaka točka $\lbrace aH \rbrace$ zaprta v $G/H$, kar je ekvivalentno separacijskemu aksiomu $T_1$. 
Naj bo množica $U$ okolica enote $e$ v $G$. Ker po trditvi \ref{trd:sim} obstaja baza simetričnih okolic, potem po trditvi \ref{trd:okolice} obstaja takšna okolica enote $V$, da velja $V^2 = V^{-1}V \subset U$. Torej po trditvi \ref{trd:okolicevkvoc} za vsako okolico $\varphi(U)$ točke $H$ obstaja takšna okolica $\varphi(V)$ točke $H$, da je $\closure{\varphi(V)} \subset \varphi(U)$.
Ker je leva translacija homeomorfizem, to velja za vsako točko $\lbrace aH \rbrace \in G/H$, kar pa je ekvivalentno separacijskemu aksiomu $T_3$. Ker kvocientni prostor $G/H$ zadošča separacijskima aksiomoma $T_1$ in $T_3$, je regularen.

Za dokaz točke (\ref{podtrd:kvocreg3}) privzemimo, da $G/H$ zadošča separacijskemu aksiomu $T_0$. Po izreku \ref{trd:t0haus} je množica $\lbrace H \rbrace$ zaprta. Po definiciji kvocientne topologije je $\lbrace H \rbrace$ zaprta v $G/H$ natanko tedaj, ko je $\varphi^{-1}(\lbrace H \rbrace) = H$ zaprta v $G$. Po točki (\ref{podtrd:kvocreg2}) je $G/H$ regularen topološki prostor.
\end{dokaz}

V naslednjem izreku bomo končno dokazali, da je v primeru, ko je $H$ podgrupa edinka topološke grupe $G$, tudi $G/H$ topološka grupa. Hkrati bomo povzeli glavne lastnosti kvocientnih topoloških grup, ki smo jih že dokazali v tem podrazdelku.
\begin{izrek}\label{izr:kvocpovzetek}
Naj bo $H$ podgrupa edinka topološke grupe $G$ in naj bo $G/H$ kvocientni topološki prostor. Veljajo naslednje trditve:
\begin{enumerate}
\item kvocientni topološki prostor $G/H$ je topološka grupa s kvocientno topologijo $\theta$,\label{podtrd:kvoc1}
\item naravni homomorfizem je odprta in zvezna preslikava,\label{podtrd:kvoc2}
\item kvocientni topološki prostor $G/H$ je diskreten natanko tedaj, ko je podgrupa $H$ odprta v $G$,\label{podtrd:kvoc3}
\item kvocientni topološki prostor $G/H$ je regularen natanko tedaj, ko je pod\-gru\-pa $H$ zaprta v $G$.\label{podtrd:kvoc4}
\end{enumerate}
\end{izrek}

\begin{dokaz}
Za dokaz točke (\ref{podtrd:kvoc1}) bomo uporabili izrek \ref{izr:bazaokolice}. Preverili bomo, da za družino vseh okolic enote $H$ v grupi $G/H$ s kvocientno topologijo veljajo lastnosti (\ref{last:oko1})-(\ref{last:oko4}) trditve \ref{trd:okolice} in zadostili pogojem izreka \ref{izr:bazaokolice}.

Naj bo $\lbrace uH ; u \in U \rbrace$ poljubna okolica enote $H$ v grupi $G/H$, kjer je $U$ poljubna okolica enote $e$ v topološki grupi $G$. Po lastnosti (\ref{last:oko1}) trditve \ref{trd:okolice} obstaja taka okolica $V$ enote $e$, da velja $V^2 \subset U$. Po definiciji kvocientne topologije je množica $\lbrace vH ; v \in V \rbrace$ odprta v $G/H$ in velja \[\lbrace vH ; v \in V \rbrace^2 = \lbrace yH ; y \in V^2 \rbrace \subset \lbrace uH ; u \in U \rbrace.\] Torej za grupo $G/H$ s kvocientno topologijo drži lastnost (\ref{last:oko1}) trditve \ref{trd:okolice}.

Po lastnosti (\ref{last:oko2}) trditve \ref{trd:okolice} obstaja taka okolica $V$ enote $e$, da velja $V^{-1} \subset U$. Po definiciji kvocientne topologije je množica $\lbrace vH ; v \in V^{-1} \rbrace$ odprta in velja \[\lbrace vH ; v \in V \rbrace^{-1} = \lbrace yH ; y \in V^{-1} \rbrace \subset \lbrace uH ; u \in U \rbrace.\] Torej za grupo $G/H$ s kvocientno topologijo drži lastnost (\ref{last:oko2}) trditve \ref{trd:okolice}.

Za dokaz lastnosti (\ref{last:oko3}) vzemimo še poljuben element $u_0H \in \lbrace uH ; u \in U \rbrace$. Po lastnosti (\ref{last:oko3}) trditve \ref{trd:okolice} obstaja taka okolica $V$ enote $e$, da velja $u_0V \subset U$. Potem je $\lbrace vH ; v \in V \rbrace$ okolica enote $H$ v $G/H$ in velja \[u_0H\cdot\lbrace vH ; v \in V\rbrace = \lbrace u_0vH ; v \in V \rbrace \subset uH ; u \in U \rbrace.\] Torej za grupo $G/H$ s kvocientno topologijo drži lastnost (\ref{last:oko3}) trditve \ref{trd:okolice}.

Po lastnosti (\ref{last:oko4}) trditve \ref{trd:okolice} obstaja taka okolica $V$ enote $e$, da velja $u_0Vu_0^{-1} \subset U$. Potem je $\lbrace vH ; v \in V \rbrace$ okolica enote $H$ v $G/H$ in velja
\[u_0H\cdot\lbrace vH ; v \in V\rbrace (u_0H)^{-1} = \lbrace u_0vu_0^{-1}H ; v \in V \rbrace \subset uH ; u \in U \rbrace.\]
Torej za grupo $G/H$ s kvocientno topologijo drži lastnost (\ref{last:oko4}) trditve \ref{trd:okolice}.

Družina vseh okolic enote $H$ v $G/H$ torej zadošča lastnostim (\ref{last:oko1})-(\ref{last:oko4}) trditve \ref{trd:okolice}. Ker vse okolice vsebujejo enoto $H$, so vsi končni preseki teh okolic neprazni. Ker je presek odprtih množic po definiciji topologije odprt in po prejšnji lastnosti neprazen, obstaja za vsaki dve okolici $\lbrace uH ; u \in U \rbrace$ in $\lbrace vH ; v \in V \rbrace$ enote $H$ neka okolica $W \subset U \cap V$ enote $H$. Po izreku \ref{izr:bazaokolice} je grupa $G/H$ res topološka grupa s kvocientno topologijo.

Točka (\ref{podtrd:kvoc2}) sledi direktno iz izreka \ref{izr:topkvocienta}.
Točka (\ref{podtrd:kvoc3}) sledi direktno iz točke (\ref{podtrd:kvocreg1}) izreka \ref{izr:kvocreg}, točka (\ref{podtrd:kvoc4}) pa iz točk (\ref{podtrd:kvocreg2}) in (\ref{podtrd:kvocreg3}) izreka \ref{izr:kvocreg}.
\end{dokaz}

S pomočjo zgornjega izreka lahko sedaj pokažemo, da sta separacijski aksiom $T_0$ in regularnost za topološke grupe ekvivalentna pojma. Kasneje bomo pokazali, da iz separacijskega aksioma $T_0$ sledi še več, in sicer povsem regularnost.

\begin{posledica}\label{izr:t3}
	Vsaka topološka grupa $G$, ki zadošča separacijskemu aksiomu $T_0$, je regularen topološki prostor.
\end{posledica}

\begin{dokaz}
Ker topološka grupa $G$ zadošča separacijskemu aksiomu $T_0$, je po trditvi \ref{trd:t0haus} podgrupa edinka $H = \lbrace e \rbrace$ zaprta v $G$. Po izreku \ref{izr:kvocpovzetek} je $G/H = G$ regularen topološki prostor.
\end{dokaz}

Končno lahko dokažemo še en izrek, povezan s topološkimi podgrupami, ki ga bomo potrebovali kasneje, uporaben pa je tudi drugje.

\begin{definicija}
	Topološki prostor je $\sigma$-kompakten, če ge je možno zapisati kot števno unijo kompaktnih topoloških podprostorov.
\end{definicija}

\begin{izrek}\label{izr:odpzapsigma}
	Naj bo $G$ lokalno kompaktna Hausdorffova topološka grupa. Tedaj v $G$ obstaja odprta, zaprta in $\sigma$-kompaktna podgrupa.
\end{izrek}

\begin{dokaz}
	Ker je $G$ lokalno kompaktna, obstaja kompaktna okolica $K$ enote $e$. Po posledici \ref{pos:sim} obstaja takšna simetrična okolica $U$ enote $e$, da je $\closure{U} \subset K$. Ker so zaprte podmnožice kompaktnih prostorov kompaktne, je tudi $\closure{U}$ kompaktna okolica enote $e$.
	
	Naj bo $L = \bigcup_{n=1}^\infty U^n$. Množica $L$ je po trditvi \ref{trd:podgrupaunija} odprta in zaprta podgrupa topološke grupe $G$.
	Preverimo, da je $\closure{U} \subseteq U^2$. Vzemimo $x \in \closure{U}$. Ker je množica $xU$ odprta okolica za $x$, velja $xU \cap U \neq \emptyset$, iz česar sledi, da obstajata elementa $u_1,u_2 \in U$, da je $xu_1 = u_2$. Od tod dobimo
	\[ x = u_2u_1^{-1} \in UU^{-1} = U^2. \]
	S preprosto uporabo matematične indukcije lahko pokažemo, da za vsak $n \in \N$ velja $\closure{U}^n \subseteq U^{2n}$, zato je $L = \bigcup_{n=1}^\infty \closure{U}^n$. Po trditvi \ref{trd:prodkomp} je za vsak $n \in \N$ množica $\closure{U}^n$ kompaktna, zato je podgrupa $L$ enaka števni uniji kompaktnih množic, torej je $\sigma$-kompaktna.
\end{dokaz}

\begin{posledica}
	Vsaka povezana, lokalno kompaktna Hausdorffova topološka grupa je $\sigma$-kompaktna.
\end{posledica}

\begin{dokaz}
	Naj bo $G$ povezana, lokalno kompaktna Hausdorffova topološka grupa. Po izreku \ref{izr:odpzapsigma} v $G$ obstaja odprta, zaprta in $\sigma$-kompaktna podgrupa $H$. Ker je $H$ podgrupa, gotovo vsebuje vsaj enoto, torej je neprazna. Ker je $G$ povezana in je $H$ odprta in zaprta, je $G = H$. Torej je $G$ $\sigma$-kompaktna topološka grupa. 
\end{dokaz}

\section{Izreki o izomorfizmih}
V tem razdelku bomo pokazali, da za topološke grupe z dodatnimi topološkimi predpostavkami veljajo trije izreki o izomorfizmih za topološke grupe, ki so analogni trem algebraičnim izrekom o izomorfizmih za grupe. Najprej pa dokažimo nekaj pomožnih trditvev.

\begin{trditev}\label{trd:homogenkvoc}
	Naj bo $G$ topološka grupa in $H$ njena podgrupa. Naj bo za vsak element $a \in G$ na kvocientnem topološkem prostoru $G/H$ definirana preslikava $\psi_a$ s predpisom $\psi_a(xH) = (ax)H$.
	Za vsak element $a \in G$ je $\psi_a$ homeomorfizem na prostoru $G/H$.
\end{trditev}

\begin{dokaz}
Preslikava $\psi_a$ je očitno bijektivna za vsak $a \in G$, saj je preslikava $\psi_{a^{-1}}$ njen inverz. Pokazali bomo, da je za vsak $a \in G$ preslikava $\psi_a$ odprta, iz česar bo sledilo, da je preslikava $\psi_{a^{-1}}$ zvezna.
Vzemimo odprto podmnožico $\lbrace uH ; u \in U \rbrace$ prostora $G/H$, kjer je $U \subseteq G$ odprta množica. 
Ker je po trditvi \ref{trd:trans} leva translacija homeomorfizem, je tudi množica $aU$ odprta v $G$ in velja, da je
\[ \psi_a(\lbrace uH ; u \in U \rbrace) = \lbrace auH ; u \in U \rbrace = \lbrace vH ; v \in aU \rbrace \]
odprta množica, zato je preslikava $\psi_a$ odprta.

Ker je $\psi_{a^{-1}}$ odprta preslikava, je tudi $\psi_a$ zvezna preslikava. Torej je $\psi_a$ homeomorfizem.
\end{dokaz}

\begin{trditev}\label{trd:kvockompakt}
	Naj bo $H$ podgrupa (lokalno) kompaktne topološke grupe $G$. Potem je kvocientni topološki prostor $G/H$ (lokalno) kompakten.
\end{trditev}

\begin{dokaz}
Po izreku \ref{izr:topkvocienta} je naravna preslikava $\varphi\colon G \to G/H$ zvezna. Če je $G$ kompaktna topološka grupa, je tudi $\varphi(G) = G/H$ kompakten topološki prostor.

Naj bo topološka grupa $G$ lokalno kompaktna in naj bo množica $K$ kompaktna okolica poljubnega elementa $a \in G$. Ker je po trditvi \ref{trd:trans} leva translacija homeomorfizem, je množica $a^{-1}K$ kompaktna okolica enote $e$. Ker je po izreku \ref{izr:topkvocienta} naravna preslikava $\varphi\colon G \to G/H$ zvezna, je množica $\varphi(a^{-1}K)$ kompaktna okolica enote $H$ v prostoru $G/H$. Po trditvi \ref{trd:homogenkvoc} je potem $\psi_a(\varphi(a^{-1}K))$ kompaktna okolica elementa $aH$. Zato je kvocientni prostor $G/H$ lokalno kompakten.
\end{dokaz}

\begin{definicija}
\emph{Števno kompakten} prostor je tak topološki prostor, pri katerem za vsako števno odprto pokritje obstaja končno podpokritje. \emph{Lokalno števno kompakten} topološki prostor je tak prostor, v katerem ima vsaka točka števno kompaktno okolico. 
\end{definicija}

\begin{trditev}\label{trd:kompni}
Lokalno števno kompakten, regularen topološki prostor $X$ ni unija števno zaprtih množic s prazno notranjostjo.
\end{trditev}

\begin{dokaz}
Trditev bomo dokazali s protislovjem. Naj bo topološki prostor $X$ enak uniji zaprtih množic s prazno notranjostjo, torej $X = \bigcup_{n=1}^\infty A_n$, kjer je za vsak $n \in \N$ množica $A_n$ zaprta in velja $\interior(A) = \emptyset$. Za vsak $n \in \N$ definiramo $D_n = X \setminus A_n$. Očitno je za vsak $n \in \N$ množica $D_n$ odprta. Ker ima za vsak $n \in \N$ množica $A_n$ prazno notranjost, torej ne vsebuje nobene neprazne odprte množice, vsaka odprta množica v topološkem prostoru $X$ seka množico $D_n$, iz česar sledi, da je množica $D_n$ gosta v prostoru $X$. Pokazali bomo, da velja $\bigcap_{n=1}^\infty D_n \neq \emptyset$, kar pomeni, da obstaja nek element prostora $X$, ki ni v nobeni množici $A_n$, torej topološki prostor $X$ ne more biti unija množic $A_n$.

Vzemimo poljuben element $x_0 \in X$ in naj bo množica $K$ njegova števno kompaktna okolica. Ker je prostor $X$ regularen, obstaja takšna odprta okolica $U_0$ elementa $x_0$, da je $U_0 \subset \closure{U_0} \subset K$, kjer je tudi množica $\closure{U_0}$ števno kompaktna. Ker je množica $D_1$ gosta v regularnem topološkem prostoru $X$, ima z vsako neprazno odprto množico neprazen presek, torej obstaja takšna neprazna odprta množica $U_1$, da je $U_1 \subset \closure{U_1} \subset U_0 \cap D_1$. Induktivno za vsak $n \in \N$ definiramo množico $U_n$ na naslednji način. Če smo že definirali množice $U_i$ za $i = 1,\dots,n-1$, naj bo množica $U_n$ takšna neprazna odprta množica, da velja $U_n \subset \closure{U_n} \subset U_{n-1} \cap D_n$. Tako kot pri definiciji množice $U_1$ upoštevamo, da je množica $D_n$ gosta za vsak $n \in \N$ in da je topološki prostor $X$ regularen.
Ker je množica $\closure{U_0}$ števno kompaktna in je za vsak $n \in \N$ množica $\closure{U_n}$ neprazna, je tudi presek $\bigcap_{n=0}^\infty \closure{U_n}$ neprazen. Elementi tega preseka morajo po zgornji konstrukciji ležati v preseku $\bigcap_{n=1}^\infty D_n$, torej res velja $\bigcap_{n=1}^\infty D_n \neq \emptyset$.
\end{dokaz}

\begin{trditev}\label{trd:kompodp}
Naj bo $G$ lokalno kompaktna, $\sigma$-kompaktna topološka grupa in naj bo $f\colon G \to \widetilde{G}$ zvezen in surjektiven homomorfizem v lokalno števno kompaktno topološko grupo $\widetilde{G}$, ki zadošča separacijskemu aksiomu $T_0$. Tedaj je $f$ odprta preslikava.
\end{trditev}

\begin{dokaz}
Pišimo $G = \bigcup_{n=1}^\infty A_n$, kjer je $A_n$ kompaktna množica za vsak $n \in \N$. Naj bo $\Ucurl$ družina vseh simetričnih okolic enote $e$ topološke grupe $G$ in naj bo $\widetilde{\Ucurl}$ družina vseh okolic enote $\tilde{e}$ topološke grupe $\widetilde{G}$. Dokazali bomo, da za vsako okolico $U \in \Ucurl$ obstaja okolica $\widetilde{U} \in \widetilde{\Ucurl}$, da velja $\widetilde{U} \subset f(U)$. Potem lahko vzamemo poljubno odprto podmnožico $B \subset G$. Za vsak $x \in B$ obstaja taka okolica $U \in \Ucurl$, da je $xU$ okolica elementa $x$ in velja $xU \subset B$. Ker obstaja $\widetilde{U} \in \widetilde{\Ucurl}$, da velja $\widetilde{U} \subset f(U)$, imamo
\[ f(x) \in f(x)\widetilde{U} \subset f(x)f(U) = f(xU) \subset f(B), \]
torej je množica $f(B)$ okolica za vsako svojo točko in po definiciji odprta v topološki grupi $\widetilde{G}$. Ker je množica $B$ poljubna, je $f$ odprta preslikava.

Dokažimo torej, da za vsako okolico $U \in \Ucurl$ obstaja okolica $\widetilde{U} \in \widetilde{\Ucurl}$, da velja $\widetilde{U} \subset f(U)$. Vzemimo poljubno okolico $U \in \Ucurl$. Po trditvi \ref{trd:okolice} obstaja takšna množica $V_1 \in \Ucurl$, da velja $V_1^2 \subset U$. Po posledici \ref{pos:sim} obstaja takšna množica $V \in \Ucurl$, da velja $\closure{V} \subset V_1$. Upoštevamo še trditev \ref{trd:zaprtost} in dobimo
\[ U \supset V_1^2 \supset \closure{V}\;\closure{V} = \closure{V^{-1}}\;\closure{V} = \closure{V}^{-1}\closure{V}. \]
Ker je topološka grupa $G$ lokalno kompaktna, lahko vzamemo tako množico $V$, da je množica $\closure{V}$ kompaktna.
Ker je po trditvi \ref{trd:trans} leva translacija homeomorfizem, je družina množic $\lbrace xV ; x \in G \rbrace$ odprto pokritje topološke grupe $G$, in zato tudi množice $A_n$ za vsak $n \in \N$. Ker je množica $A_n$ za vsak $n \in \N$ kompaktna, jo pokrije le končno množic oblike $xV$, kjer je $x \in G$. Množic $A_n$ je števno, torej topološko grupo $G$ pokrije števno množic oblike $xV$, kjer je $x \in G$. Naj bo to števno pokritje družina $\lbrace x_nV \rbrace_{n=1}^\infty$. Potem velja $G = \bigcup_{n=1}^\infty x_nV = \bigcup_{n=1}^\infty x_n\closure{V}$ in ker je preslikava $f$ surjektivna, velja tudi $\widetilde{G} = \bigcup_{n=1}^\infty f(x_n\closure{V}) = \bigcup_{n=1}^\infty f(x_n)f(\closure{V})$.

Pokažimo, da ima množica $f(\closure{V})$ neprazno notranjost. Ker je po trditvi \ref{trd:trans} leva translacija homeomorfizem, je za vsak $n \in \N$ množica $x_n\closure{V}$ kompaktna v topološki grupi $G$. Ker je preslikava $f$ zvezna, je množica $f(x_n)f(\closure{V})$ kompaktna v topološki grupi $\widetilde{G}$ za vsak $n \in \N$. Topološka grupa $\widetilde{G}$ zadošča separacijskemu aksiomu $T_0$, zato je po trditvi \ref{trd:t0haus} Hausdorffova, torej so vse kompaktne množice $f(x_n)f(\closure{V})$ zaprte v $\widetilde{G}$. Če predpostavimo, da ima množica $f(\closure{V})$ prazno notranjost, imajo tudi množice $f(x_n)f(\closure{V})$ prazno notranjost za vsak $n \in \N$, saj je leva translacija po trditvi \ref{trd:trans} homeomorfizem. Potem je topološka grupa $\widetilde{G} = \bigcup_{n=1}^\infty f(x_n)f(\closure{V})$ unija števno zaprtih množic s prazno notranjostjo. Po drugi strani je topološka grupa $\widetilde{G}$ lokalno števno kompakten topološki prostor, ki zadošča separacijskemu aksiomu $T_0$, in je zato po izreku \ref{izr:t3} regularen. Po trditvi \ref{trd:kompni} je to nemogoče, zato ima množica $f(\closure{V})$ neprazno notranjost, torej vsebuje neko neprazno odprto množico $\widetilde{V} \subset \widetilde{G}$.

Izberimo poljubni točki $\tilde{x} \in \widetilde{V}$ in $x \in f^{-1}(\tilde{x})\cap\closure{V}$.
Po konstrukciji množice $V$ potem velja $x^{-1}\closure{V} \subset U$. Ker je $f$ zvezna preslikava, dobimo
\[ f(U) \supset f(x^{-1}\closure{V}) = f(x)^{-1}f(\closure{V}) \supset \tilde{x}^{-1}\widetilde{V}. \]
Množica $\tilde{x}^{-1}\widetilde{V}$ je okolica enote $\tilde{e}$ in je zato vsebovana v družini $\widetilde{\Ucurl}$. S tem smo dokazali zgornjo izjavo in s tem izrek.
\end{dokaz}

\subsection{Prvi izrek o izomorfizmih}
Vzemimo grupi $G$ in $\widetilde{G}$ ter surjektiven homomorfizem $f\colon G \to \widetilde{G}$. Prvi izrek o izomorfizmih za grupe pravi, da je $H := \ker f$ podgrupa edinka grupe $G$ in da sta $G/H$ in $\widetilde{G}$ izomorfni grupi. Naslednji izrek pravi, da ob dodatnih predpostavkah odprtosti in zveznosti homomorfizma $f$ velja topološka različica izreka tudi za topološke grupe.
\begin{izrek}[Prvi izrek o izomorfizmih za topološke grupe]\label{izr:prvitopizrek}
Naj bosta $G$ in $\widetilde{G}$ topološki grupi. Naj bo $f\colon G \to \widetilde{G}$ odprt, zvezen in surjektiven homomorfizem. Preslikava $\Phi\colon\widetilde{G} \to G/\ker f$, definirana s predpisom $\tilde{x} \mapsto f^{-1}(\tilde{x})$, je topološki izomorfizem.
\end{izrek}

\begin{dokaz}
Upoštevajoč prvi izrek o izomorfizmih moramo pokazati le, da je preslikava $\Phi$ homeomorfizem. Vzemimo odprto podmnožico $\widetilde{U} \subset \widetilde{G}$. Zaradi lažje berljivosti pišimo $H = \ker f$.
Ker je preslikava $f$ zvezna, je množica $\bigcup\lbrace f^{-1}(\tilde{x}) ; \tilde{x} \in \widetilde{U} \rbrace = f^{-1}(\widetilde{U})$ odprta v $G$. Ker je množica $\varphi^{-1}(\Phi(\widetilde{U})) = \widetilde{U}HH = \widetilde{U}H$ po trditvi \ref{trd:prododp} odprta v $G$, je množica $\Phi(\widetilde{U}) = \lbrace f^{-1}(\tilde{x}) ; \tilde{x} \in \widetilde{U} \rbrace$ odprta v $G/H$. Od tod sledi, da je preslikava $\Phi$ odprta, torej je $\Phi^{-1}$ zvezna preslikava.

Vzemimo odprto množico $\lbrace uH ; u \in U \rbrace$ v $G/H$, kjer je množica $U \subset G$ odprta. Potem velja
\begin{align*}
\Phi^{-1}(\lbrace uH ; u \in U \rbrace) &= \lbrace \tilde{x} \in \widetilde{G} ; f^{-1}(\tilde{x}) = uH \text{ za nek element } u \in U \rbrace \\
&= \lbrace f(u) ; u \in U \rbrace = f(U).
\end{align*}
Ker je preslikava $f$ po predpostavki odprta, je $f(U)$ odprta množica v $\widetilde{G}$. Torej je preslikava $\Phi^{-1}$ odprta, od koder sledi, da je $\Phi$ zvezna preslikava. Sledi, da je $\Phi$ homeomorfizem.
\end{dokaz}

\subsection{Drugi izrek o izomorfizmih}
Vzemimo grupo $G$, njeno podgrupo $A$ in podgrupo edinko $H$. Drugi izrek o izomorfizmih za grupe pravi, da je produkt podgrup $AH$ tudi podgrupa grupe $G$, da je presek podgrup $A \cap H$ podgrupa edinka podgrupe $A$ in da sta kvocientni grupi $(AH)/H$ in $A/(A \cap H)$ izomorfni grupi z izomorfizmom $\tau\colon (AH)/H \to A/(A \cap H)$, definiranim s predpisom $aH \mapsto (aH)\cap A = a(H \cap A)$, kjer je $a \in A$. Drugi izrek o izomorfizmih za topološke grupe zahteva največ dodatnih predpostavk.
\begin{izrek}[Drugi izrek o izomorfizmih za topološke grupe]\label{izr:drugitopizrek}
Naj bo $G$ topološka grupa, $A$ njena podgrupa in $H$ podgrupa edinka grupe $G$. Naj bo $\tau\colon (AH)/H \to A/(A \cap H)$ izomorfizem s predpisom $\tau (aH) = a(A \cap H)$, kjer je $a \in A$.
\begin{enumerate}
\item Preslikava $\tau$ je odprta preslikava. \label{podtrd:ioi2-1}
\item Če je $A$ še lokalno kompaktna in $\sigma$-kompaktna, $H$ zaprta v $G$ in $AH$ lokalno kompaktna, potem je $\tau$ homeomorfizem ter topološki grupi $(AH)/H$ in $A/(A \cap H)$ sta topološko izomorfni. \label{podtrd:ioi2-2}
\end{enumerate}
\end{izrek}

\begin{dokaz}
Najprej dokažimo (\ref{podtrd:ioi2-1}).
Po definiciji kvocientne topologije je odprta množica v prostoru $(AH)/H$ množica $\lbrace xH ; \; x\in X \rbrace$, kjer je $X \subset A$ taka množica, da je $\varphi^{-1}(\lbrace xH ; \; x\in X \rbrace) = XH$ odprta v podprostoru $AH$ topološke grupe $G$.

Pokažimo, da je $X(A \cap H) = (XH)\cap A$.
Najprej vzemimo $y \in X(A \cap H)$. Potem obstajajo taki elementi $x \in X$, $a \in A$ in $h \in H$, da velja $y = xa = xh$. Očitno sledi, da je $y \in XH$. Ker je $A$ podgrupa in ker je $X \subseteq A$, je $XA \subseteq A$, zato velja $y \in A$. Od tod sledi $y \in (XH)\cap A$, torej $X(A\cap H) \subseteq (XH)\cap A$.
Za dokaz obratne inkluzije vzemimo $y \in (XH)\cap A$. Potem obstajajo taki elementi $x \in X$, $a \in A$ in $h \in H$, da velja $y = xh = a$. Ker je $xh \in A$ in ker je $A$ podgrupa, ki vsebuje množico $X$, je $h \in x^{-1}A = A$. Torej je $(XH)\cap A \subseteq X(A \cap H)$.
 
Ker velja $X(A \cap H) = (XH)\cap A$ in ker je po trditvi \ref{trd:prododp} množica $XH$ odprta, je množica $X(A \cap H)$ odprta v podprostoru $A$ topološke grupe $G$. Ker je $\tau(\lbrace xH ; x \in X \rbrace) = \lbrace x(A \cap H) ; x \in X \rbrace$, je po definiciji kvocientne topologije množica $\lbrace x(A\cap H) ; x \in X \rbrace$ odprta v prostoru $A/(A \cap H)$.

Za dokaz (\ref{podtrd:ioi2-2}) moramo upoštevajoč drugi izrek o izomorfizmih in točko (\ref{podtrd:ioi2-1}) dokazati le, da je tudi preslikava $\tau^{-1}$ odprta.
Oglejmo si naravno preslikavo $\varphi\colon G \to G/H$. Za njeno zožitev na podgrupo $A$ velja, da preslika podgrupo $A$ surjektivno v podgrupo $(AH)/H$ topološke grupe $G/H$. Ker je po predpostavki množica $AH$ lokalno kompaktna in podgrupa edinka $H$ zaprta v $G$, je po trditvi \ref{trd:kvockompakt} množica $(AH)/H$ lokalno kompaktna in po izreku \ref{izr:kvocreg} zadošča separacijskemu aksiomu $T_0$. Naravna preslikava $\varphi|_A$ je torej surjektiven in odprt homomorfizem iz lokalno kompaktne, $\sigma$-kompaktne grupe $A$ v lokalno kompaktno grupo $(AH)/H$, ki zadošča separacijskemu aksiomu $T_0$. Po trditvi \ref{trd:kompodp} je $\varphi|_A$ odprta preslikava.

Vzemimo odprto podmnožico $\lbrace y(A \cap H) ; y \in Y \rbrace \subset A/(A \cap H)$, kjer je $Y \subset A$, tj. množica $Y(A \cap H)$ je odprta v $A$ (glej dokaz točke \ref{podtrd:ioi2-1}). Ker je $\varphi|_A$ odprta preslikava, je množica $\varphi(Y(A \cap H)) = \lbrace yH ; y \in Y \rbrace$ odprta v $(AH)/H$. Po definiciji preslikave $\tau$ velja $\tau^{-1}(\lbrace y(A \cap H) ; y \in Y \rbrace) = \lbrace yH ; y \in Y \rbrace$, iz česar sledi, da je preslikava $\tau^{-1}$ odprta.
\end{dokaz}


\subsection{Tretji izrek o izomorfizmih}
Za tretji izrek o izomorfizmih za grupe potrebujemo grupo $G$ in njeno podgrupo edinko $N$. Če je $H$ taka podgrupa grupe $G$, da je $N \subseteq H$, potem je $H/N$ podgrupa grupe $G/N$. Še več, vsaka podgrupa grupe $G/N$ je oblike $H/N$ za neko podgrupo $H$ grupe $G$, kjer je $N \subseteq H$. Če je $H$ podgrupa edinka grupe $G$, potem je $H/N$ podgrupa edinka grupe $G/N$. Še več, vsaka podgrupa edinka grupe $G/N$ je oblike $H/N$ za neko podgrupo edinko $H$ grupe $G$. Potem sta grupi $(G/N)/(H/N)$ in $G/H$ izomorfni. Pri dokazu tretjega izreka o izomorfizmih za topološke grupe si bomo v naslednji trditvi najprej pomagali s pomožno topološko grupo $\widetilde{G}$.
\begin{trditev}\label{trd:predtretji}
	Naj bo $f\colon G \to \widetilde{G}$ odprt, surjektiven in zvezen homomorfizem topoloških grup in naj bo $\widetilde{H}$ podgrupa edinka v $\widetilde{G}$. Označimo $N := \ker f$ in $H := f^{-1}(\widetilde{H})$. Potem so grupe $(G/N)/(H/N)$, $G/H$ in $\widetilde{G}/\widetilde{H}$ topološko izomorfne.
\end{trditev}

\begin{dokaz}
Po izreku \ref{izr:kvocpovzetek} je naravna preslikava $\tilde{\varphi}\colon \widetilde{G} \to \widetilde{G}/\widetilde{H}$ odprt in zvezen homomorfizem. Ker je preslikava $f$ odprt in zvezen homomorfizem, je preslikava $\tilde{\varphi}\circ f\colon G \to \widetilde{G}/\widetilde{H}$ odprt in zvezen homomorfizem z jedrom $H$. Po izreku \ref{izr:prvitopizrek} sta topološki grupi $G/H$ in $\widetilde{G}/\widetilde{H}$ topološko izomorfni. Ker je po izreku \ref{izr:prvitopizrek} preslikava $f^{-1}\colon \widetilde{G} \to G/N$ topološki izomorfizem in ker je $f^{-1}(\widetilde{H}) = H/N$, je topološka grupa $\widetilde{G}/\widetilde{H}$ topološko izomorfna $(G/N)/(H/N)$.
\end{dokaz}
Izrek lahko preoblikujemo v obliko, ki je bolj podobna algebraični različici in ne vsebuje pomožne topološke grupe $\widetilde{G}$.
\begin{izrek}[Tretji izrek o izomorfizmih za topološke grupe]\label{izr:tretjitopizrek}
Naj bo $G$ topološka grupa in $N \subseteq H$ njeni podgrupi edinki. Potem sta kvocientni topološki grupi $G/H$ in $(G/N)/(H/N)$ topološko izomorfni.
\end{izrek}

\section{Metrizabilnost in povsem regularnost}
\subsection{Uniformni prostori}
Preden si ogledamo metrizabilnost na Hausdorffovih topoloških grupah, najprej potrebujemo pojem enakomerne zveznosti. Čeprav lahko enakomerno zveznost na topoloških grupah obravnavamo brez vpeljave uniformnih prostorov, s tem izgubimo del teorije. Zato bomo v tem kratkem podrazdelku vpeljali pojem uniformnega prostora in splošno definirali pojem enakomerne zveznosti. Nato bomo na topološki grupi definirali levo in desno uniformno strukturo in pokazali, da je vsaka topološka grupa uniformni prostor. Omenimo še, da teorija uniformnih prostorov obstaja samostojno, tako kot teorija metričnih in topoloških prostorov.
\begin{definicija}\label{def:uniform}
Naj bo $X$ neprazna množica. Neprazna poddružina $\mathcal{F} \subset \mathcal{P}(X)$ je \emph{filter} na množici $X$, če ima naslednje lastnosti:
\begin{enumerate}
\item $\emptyset \notin \mathcal{F}$,
\item za vsako množico $F \in \mathcal{F}$ je vsaka množica $E \subset X$, za katero velja $F \subseteq E$, tudi v družini $\mathcal{F}$,
\item presek $E \cap F$ množic $E, F \in \mathcal{F}$ je tudi v družini $\mathcal{F}$.
\end{enumerate}
\end{definicija}

\begin{primer}
Preprost primer filtra je družina vseh okolic poljubne točke $x$ topološkega prostora $X$. 
\end{primer}
\begin{definicija}
Filter $\mathcal{U}$ na neprazni množici $X \times X$ definira \emph{uniformno strukturo} $\Ucurl$ na množici $X$, če ima naslednje lastnosti:
\begin{enumerate}
\item vsaka množica $U \in \mathcal{U}$ vsebuje diagonalo $\Delta = \lbrace (x, x) ; x \in X \rbrace$,
\item za vsako množico $U \in \mathcal{U}$ je tudi množica $U^{-1} \in \mathcal{U}$,
\item za vsako množico $U \in \mathcal{U}$ obstaja taka množica $V \in \mathcal{U}$, da velja $V \circ V \subseteq U$.
\end{enumerate}
Množici z uniformno stukturo rečemo \emph{uniformni prostor} in označimo $(X, \Ucurl)$.
\end{definicija}

\begin{opomba}
	V zgornji definiciji so operacije na množicah mišljene v smislu relacij.
\end{opomba}

\begin{trditev}\label{trd:uniinduciranatopo}
Naj bo $X$ uniformni prostor z uniformno strukturo $\mathcal{U}$. Družina $\tau$ množic $T \subseteq X$, za katere za vsako točko $x \in T$ obstaja $U \in \mathcal{U}$, da velja $U_x = \lbrace y \in X ; (x, y) \in U \rbrace \subseteq T$, je topologija na množici $X$.
\end{trditev}

\begin{dokaz}
Brez škode za splošnost lahko privzamemo $\emptyset \in \tau$, saj je pogoj iz trditve na prazno izpolnjen. Ker je za vsak $x \in X$ in vsak $U \in \Ucurl$ množica $U_x \subseteq X$ (lahko tudi prazna), je tudi $X \in \tau$.

Vzemimo poljubno unijo $\bigcup_{\lambda \in \Lambda} T_\lambda$ množic iz $\tau$ in poljubno točko $x \in \bigcup_{\lambda \in \Lambda} T_\lambda$. Potem obstaja
tak indeks $\lambda_0$, da je $x \in T_{\lambda_0}$. Ker je $T_{\lambda_0} \in \tau$, obstaja taka množica $U_ {\lambda_0} \in \Ucurl$, da velja $(U_{\lambda_0})_x \subseteq T_{\lambda_0} \subseteq \bigcup_{\lambda \in \Lambda} T_\lambda$. Torej je tudi unija $\bigcup_{\lambda \in \Lambda} T_\lambda \in \tau$.

Vzemimo še presek $T_1 \cap T_2$ množic iz $\tau$ in poljubno točko $x \in T_1 \cap T_2$. Ker je $x \in T_1$ in $x \in T_2$, obstajata množici $U_1, U_2 \in \Ucurl$, da je $(U_{1})_x \subseteq T_1$ in $(U_{2})_x \subseteq T_2$. Ker je $\Ucurl$ filter, je $V = U_1 \cap U_2 \in \Ucurl$ in velja
$V_x = (U_{1})_x\cap (U_{2})_x \subseteq T_1 \cap T_2$, zato je tudi presek $T_1 \cap T_2 \in \tau$. Torej je $\tau$ res topologija na uniformnem prostoru $X$.
\end{dokaz}

\begin{definicija}\label{def:uniinduciranatopo}
Topologiji, definirani v trditvi \ref{trd:uniinduciranatopo}, pravimo topologija, inducirana z uniformno strukturo.
\end{definicija}

\begin{opomba}
	V nadaljevanju bomo okolico nekega elementa $x$ v topologiji, in\-du\-ci\-ra\-ni z uniformno strukturo $\mathcal{U}$, označevali kot $U_x$, kjer bo $U \in \mathcal{U}$.
\end{opomba}

\begin{definicija}\label{def:enakzveznost}
	Naj bosta $X$ in $Y$ zaporedoma uniformna prostora z uniformnima strukturama $\mathcal{U}$ in $\mathcal{V}$. Preslikava $f\colon X \to Y$ je \emph{enakomerno zvezna}, če za vsako množico $V \in \mathcal{V}$ obstaja taka množica $U \in \mathcal{U}$, da za vsak par $(x, y) \in U$ velja $(f(x), f(y)) \in V$.
\end{definicija}

\begin{trditev}\label{trd:enakzveznazvezna}
	Vsaka enakomerno zvezna preslikava uniformnih prostorov je zvezna v topologiji, inducirani z uniformnima strukturama.
\end{trditev}

\begin{dokaz}
	Naj bo $f\colon (X, \mathcal{U}) \to (Y, \mathcal{V})$ enakomerno zvezna preslikava med uniformnima prostoroma. Vzemimo okolico $V_{f(x)}$ elementa $f(x)$ v topologiji na $Y$ inducirani z $\mathcal{V}$. Ker je $f$ enakomerno zvezna, obstaja tak $U \in \mathcal{U}$, da za vsak par $(x, y) \in U$ velja $(f(x), f(y)) \in V$. Po definiciji inducirane topologije za vsak $y \in U_x$ zato velja $f(y) \in V_{f(x)}$, kar pomeni, da je $f$ zvezna preslikava glede na topologiji, ki ju inducirata uniformni strukturi.
\end{dokaz}

\begin{definicija}\label{def:levadesnauni}
	Naj bo $\Ucurl$ baza odprtih okolic enote $e$ topološke grupe $G$. Za vsako okolico $U \in \Ucurl$ definiramo \[L_U = \lbrace (x, y) \in G \times G ; x^{-1}y \in U \rbrace\] in analogno \[R_U = \lbrace (x, y) \in G \times G ; yx^{-1} \in U \rbrace.\] Družinama $\mathcal{L}(G) = \lbrace L_U ; U \in \Ucurl\rbrace$ in $\mathcal{R}(G) = \lbrace R_U ; U \in \Ucurl\rbrace$ zaporedoma pravimo \emph{leva} in \emph{desna uniformna struktura} na $G$.
\end{definicija}

\begin{trditev}\label{trd:topguniform}
	Vsaka topološka grupa je uniformni prostor glede na levo ali desno uniformno strukturo.
\end{trditev}

\begin{dokaz}
Naj bo filter $\Ucurl$ baza okolic enote $e$ topološke grupe $G$, sestavljene iz simetričnih in odprtih okolic. Oglejmo si levo in desno uniformno strukturo na $G$, ki sta definirani z $\Ucurl$. Pokazali bomo, da ustrezata definiciji uniformne strukture na množici $G$.

Ker je enota $e$ v vsaki okolici $U \in \Ucurl$, je diagonala $\Delta = \lbrace (x, x) ; x \in X \rbrace$ vsebovana v $L_U$ in $R_U$ za vsako okolico $U \in \Ucurl$.

Ker velja \[L_{U^{-1}} = \lbrace (x, y) \in G \times G ; x^{-1}y \in U^{-1} \rbrace = \lbrace (x, y) \in G \times G ; y^{-1}x \in U \rbrace = L^{-1}_U\]
in
\[R_{U^{-1}} = \lbrace (x, y) \in G \times G ; yx^{-1} \in U^{-1} \rbrace = \lbrace (x, y) \in G \times G ; xy^{-1} \in U \rbrace = R^{-1}_U,\]
in ker za vsako okolico $U \in \Ucurl$ velja $U = U^{-1}$, je $L^{-1}_U \in \mathcal{L}(G)$ in $R^{-1}_U \in \mathcal{R}(G)$.

Ker po trditvi \ref{trd:okolice} za vsako okolico $U \in \Ucurl$ obstaja okolica $V \in \Ucurl$, da velja $V^2 \subset U$, je $L_V \circ L_V \subset L_U$ in $R_V \circ R_V \subset R_U$.

Topološka grupa je z levo ali desno uniformno strukturo torej res uniformni prostor.
\end{dokaz}

\subsection{Metrizabilnost}
\emph{Psevdometrika} na neprazni množici $X$ je preslikava $\rho\colon X\times X \to  [0, \infty)$, ki zadošča naslednjim pogojem:
\begin{enumerate}
\item za vsako točko $x \in X$ velja $\rho (x, x) = 0$,
\item za vsaki dve točki $x, y \in X$ velja $\rho (x, y) = \rho (y, x)$,
\item za vsake tri točke $x, y, z \in X$ velja $\rho (x, z) \leq \rho (x, y) + \rho (y, z)$.
\end{enumerate}
Če za preslikavo $\rho$ velja še
\begin{enumerate}[resume]
\item $\rho(x,y) = 0$ natanko tedaj, ko $x = y$,
\end{enumerate}
potem ji rečemo \emph{metrika}.
Topološki prostor $X$ je \emph{metrizabilen}, če njegova topologija $\tau$ izhaja iz kakšne metrike $d$ na množici $X$.
Baza topologije metrizabilnega topološkega prostora $X$ je družina odprtih krogel $\lbrace K(x, \epsilon); x \in X, \epsilon \in \R \rbrace$.

\begin{definicija}
Psevdometrika na grupi $G$ je \emph{levoinvariantna}, če za vsaki dve točki $x, y \in G$ in za vsak element $a \in G$ velja $\rho(ax, ay) = \rho(x, y)$.
\end{definicija}

Naslednji izrek bo ključen pri obravnavi metrizabilnosti in posledično pri obravnavi povsem regularnosti Hausdorffovih topoloških grup.
V dokazu spodnjega izreka bomo potrebovali pojem \emph{diadičnega} racionalnega števila, tj. racionalno število z imenovalcem v okrajšani obliki enakim $2^n$, kjer je $n \in \N$. Vsota in produkt dveh diadičnih števil je prav tako diadično število. Trdimo, da lahko s končno vsoto števil oblike $2^{-n_i}$, kjer je $n_i \in \N$ in $n_i \neq n_j$ za $i \neq j$, dobimo vsako diadično število med $0$ in $1$.
Vzemimo poljubno diadično število $d = \frac{b}{2^n}$, kjer je $n \in \N$ in $b < 2^n$. Sedaj zapišimo število $d$ v binarnem zapisu. Ker je imenovalec števila $d$ potenca števila $2$, je ta binarni zapis končen. Sedaj vidimo, da je $d = \Sigma_{i=1}^n k_i2^{-i}$, kjer je $k_i$ enak $1$, če je v binarnem zapisu števila $d$ na $i$-tem mestu za vejico števka $1$ in $0$ sicer. 

\begin{izrek}\label{izr:pseudometrika}
	Naj bo $\lbrace U_k \rbrace_{k = 1}^{\infty}$ tako zaporedje simetričnih okolic enote $e$ v topološki grupi $G$, da za vsak $k \in \N$ velja $U_{k+1}^2 \subset U_k$. Potem obstaja taka levoinvariantna psevdometrika $\sigma$ na $G$ z naslednjimi lastnostmi:
	\begin{enumerate}
		\item $\sigma$ je enakomerno zvezna na levi uniformni strukturi od $G \times G$;\label{last:psevdo1}
		\item $\sigma (x, y) = 0$ natanko tedaj, ko $y^{-1}x \in H = \bigcap_{k=1}^{\infty} U_k$;\label{last:psevdo2}
		\item $\sigma (x, y) \leq 2^{-k+2}$, če $y^{-1}x \in U_k$;\label{last:psevdo3}
		\item $2^{-k} \leq \sigma (x, y)$, če $y^{-1}x \notin U_k$.\label{last:psevdo4}
	\end{enumerate}
	
	Če velja še $x U_k x^{-1} = U_k$ za vsak $x \in G$ in $k \in \N$, potem je $\sigma$ tudi desnoinvariantna in dodatno velja
	\begin{enumerate}[resume]
		\item $\sigma (x^{-1}, y^{-1}) = \sigma (x, y)$ za vsaka dva elementa $x, y \in G$.\label{last:psevdo5}
	\end{enumerate}
\end{izrek}

\begin{dokaz}
Najprej preimenujmo družino okolic $\lbrace U_k \rbrace_{k = 1}^\infty$. Najprej za vsak $k \in \N$ definiramo okolic $V_{2^{-k}} = U_k$. Za vsako diadično racionalno število $r \in (0, 1)$ nato definiramo množico $V_r$ na naslednji način. Če je
\[ r = 2^{-l_1} + \cdots + 2^{-l_n}, 0 < l_1 < \cdots < l_n, \]
potem definiramo
\[ V_r = V_{2^{-l_1}}\cdots V_{2^{-l_n}}. \]
Za vsa diadična racionalna števila $r \geq 1$ definiramo množico $V_r = G$.

Pokažimo najprej, da iz $r < s$ sledi $V_r \subset V_s$. Ker za $s \geq 1$ velja $v_r \subseteq G = V_s$, lahko privzamemo, da je $s < 1$.
Naj bo število $r$ definirano kot zgoraj in naj bo
\[ s = 2^{-m_1} + \cdots + 2^{-m_p}, 0 < m_1 < \cdots < m_p. \]
Ker je $r < s$, obstaja enolično določeno število $k \in \N$, da je $l_j = m_j$ za vsak $j < k$ in $l_k > m_k$. Naj bo $W = V_{2^{-l_1}}\cdots V_{2^{-l_{k-1}}}$.
Potem, upoštevajoč $V_{2^{-k-1}}^2 \subset V_{2^{-k}}$, velja
\begin{align*}
V_r &= WV_{2^{-l_k}}V_{2^{-l_{k+1}}}V_{2^{-l_{k+2}}}\cdots V_{2^{-l_n}} \\
&\subset WV_{2^{-l_k}}V_{2^{-l_k-1}}V_{2^{-l_k-2}}\cdots V_{2^{-l_n+1}}V_{2^{-l_n}}V_{2^{-l_n}} \\
&\subset WV_{2^{-l_k}}V_{2^{-l_k-1}}V_{2^{-l_k-2}}\cdots V_{2^{-l_n+1}}V_{2^{-l_n+1}} \subset \cdots \\
&\subset WV_{2^{-l_k}}V_{2^{-l_k}} \subset WV_{2^{-l_k+1}} \subset WV_{2^{-m_k}} \\
&= V_{2^{-l_1}}V_{2^{-l_2}}\cdots V_{2^{-l_{k-1}}}V_{2^{-m_k}} \\
&\subset V_{2^{-m_1}}V_{2^{-m_2}}\cdots V_{2^{-m_{k-1}}}V_{2^{-m_k}}V_{2^{-m_{k+1}}}\cdots V_{2^{-m_p}} = V_s
\end{align*}

Pokažimo še, da za vsak zgoraj definirani $r$ in vsak $l \in \N$ velja $V_rV_{2^{-l}} \subset V_{r + 2^{-l+2}}$. Ker za $r + 2^{-l+2} \geq 1$ velja $V_{r + 2^{-l+2}} = G$, lahko privzamemo, da je  $r + 2^{-l+2} < 1$.
Če je $l > l_n$, je po konstrukciji množic $V_rV_{2^{-l}} = V_{r + 2^{-l}} \subset V_{r + 2^{-l+2}}$. Če je $l \leq l_n$, naj bo $k \in \N$ tako število, da je $l_{k-1} < l \leq l_k$, kjer označimo $l_0 = 0$. Definiramo $r_1 = 2^{-l+1} - 2^{-l_k} - 2^{-l_{k+1}} - \cdots - 2^{-l_n}$ in $r_2 = r + r_1$. Ker je $r < r_2 < r + 2^{-l+1}$, velja
\[ V_rV_{2^{l}} \subset V_{r_2}V_{2^{-l}} = V_{r_2 + 2^{-l}} \subset V_{r + 2^{-l+1} + 2^{-l}} \subset V_{r + 2^{-l+2}}. \]

Na topološki grupi $G$ bomo sedaj konstruirali psevdometriko. Za vsak $x \in G$ naj bo \[\varphi(x) = \inf\lbrace r ; x \in V_r \rbrace.\]
Očitno je $\varphi(x) = 0$ natanko tedaj, ko je $x \in H = \bigcap_{k=1}^{\infty} U_k = \bigcap_{k=1}^{\infty} V_{2^{-k}}$.

Na prostoru $G \times G$ definiramo preslikavo $\sigma$ s predpisom
\[ \sigma(x, y) = \sup_{z \in G}\lbrace |\varphi(zx) - \varphi(zy)| \rbrace. \]
Trdimo, da je preslikava $\sigma$ iskana psevdometrika.
Očitno je $\sigma(x, x) = 0$ in zaradi simetričnosti absolutne vrednosti je $\sigma(x, y) = \sigma(y, x)$ za vsaka elementa $x, y \in G$. Trikotniška neenakost velja, saj za poljubne $x, y, w \in G$ velja
\begin{align*}
\sigma(x, w) &= \sup_{z \in G}\lbrace |\varphi(zx) - \varphi(zw)| \rbrace 
= \sup_{z \in G}\lbrace |\varphi(zx) - \varphi(zy) + \varphi(zy) - \varphi(zw)| \rbrace \\
&\leq \sup_{z \in G}\lbrace |\varphi(zx) - \varphi(zy)| + |\varphi(zy) - \varphi(zw)| \rbrace \\
&\leq \sup_{z \in G}\lbrace |\varphi(zx) - \varphi(zy)| \rbrace + \sup_{z \in G}\lbrace |\varphi(zy) - \varphi(zw)| \rbrace \\
&= \sigma(x, y) + \sigma(y, w).
\end{align*}
Ker velja še
\begin{align*}
\sigma(ax, ay) &= \sup\lbrace |\varphi(zax) - \varphi(zay)| ; z \in G \rbrace \\
&= \sup\lbrace |\varphi(zax) - \varphi(zay)| ; za \in G \rbrace = \sigma(x, y),
\end{align*}
je preslikava $\sigma\colon G \times G \to [0, \infty)$ res levoinvariantna psevdometrika na topološki grupi $G$.

Dokažimo lastnost (\ref{last:psevdo3}). Naj bo $l \in \N$, $u \in V_{2^{-l}}$ in $z \in G$. Če je $z \in V_r$, potem je po zgoraj dokazanem $zu \in V_rV_{2^{-l}} \subset V_{r + 2^{-l+2}}$. Po definiciji preslikave $\varphi$ torej sledi $\varphi(zu) \leq \varphi(z) + 2^{-l+2}$. Podobno, če je $zu \in V_r$, potem je $z \in V_rV_{2^{-l}}^{-1} = V_rV_{2^{-l}} \subset V_{r + 2^{-l+2}}$, saj je okolica $V_{2^{-l}} = U_l$ simetrična. Po definiciji preslikave $\varphi$ torej sledi $\varphi(z) \leq \varphi(zu) + 2^{-l+2}$. Od tod sledi $|\varphi(z) - \varphi(zu)| \leq 2^{-l+2}$ za vsak $u \in V_{2^{-l}}$ in $z \in G$, kar po definiciji preslikave $\sigma$ pomeni $\sigma(u, e) \leq 2^{-l+2}$ za vsak $u \in V_{2^{-l}}$. Ker je preslikava $\sigma$ levoinvariantna, sledi, da je $\sigma(x, y) \leq 2^{-l+2}$, če je $y^{-1}x \in V_{2^{-l}} = U_l$.

Sedaj dokažimo lastnost (\ref{last:psevdo1}). Vzemimo $(x, y), (x_1,y_1) \in G \times G$. Če sta $x_1^{-1}x, y_1^{-1}y \in U_l$, uporabimo levoinvariantnost psevdometrike $\sigma$, trikotniško neenakost ter lastnost (\ref{last:psevdo3}), da dobimo
\begin{align*}
&|\sigma(x, y) - \sigma(x_1, y_1)| = |\sigma(x_1^{-1}x, x_1^{-1}y) - \sigma(x_1^{-1}y, e) + \sigma(e, y^{-1}x_1) - \sigma(y^{-1}x_1, y^{-1}y_1)| \\
&\leq |\sigma(x_1^{-1}x, x_1^{-1}y) - \sigma(x_1^{-1}y, e)| + |\sigma(e, y^{-1}x_1) - \sigma(y^{-1}x_1, y^{-1}y_1)| \\
&\leq |\sigma(x_1^{-1}x, e)| + |\sigma(e, y^{-1}y_1)| \leq 2^{-l+2} + 2^{-l+2} = 2^{-l+3}
\end{align*}
Pokazali smo, da če sta $x_1^{-1}x, y_1^{-1}y \in U_l$, potem velja $|\sigma(x, y) - \sigma(x_1, y_1)| \leq 2^{-l+3}$. Po definiciji leve uniformne strukture na topološki grupi $G$ je psevdometrika $\sigma$ enakomerno zvezna preslikava.

Dokažimo lastnost (\ref{last:psevdo4}). Naj velja $y^{-1}x \notin U_l = V_{2^{-l}}$. Po definiciji preslikave $\varphi$ velja $\varphi(y^{-1}x) \geq 2^{-l}$. Upoštevamo levoinvariantnost psevdometrike $\sigma$ in dobimo
\[ \sigma(x, y) = \sigma(y^{-1}x, e) \geq |\varphi(ey^{-1}x) - \varphi(ee)| = \varphi(y^{-1}x) \geq 2^{-l}. \]

Lastnost (\ref{last:psevdo2}) sledi iz lastnosti (\ref{last:psevdo3}) in (\ref{last:psevdo4}). Če je $\sigma(x, y) = \sigma(y^{-1}x, e) = 0$, mora veljati $y^{-1}x \in U_k$ za vsak $k \in \N$, sicer bi po lastnosti (\ref{last:psevdo4}) obstajal $k_0 \in \N$, da bi veljalo $\sigma(x, y) \geq 2^{-k_0} > 0$. Torej je $y^{-1}x \in H = \bigcap_{k=1}^\infty U_k$. Če pa je $y^{-1}x \in H = \bigcap_{k=1}^\infty U_k$, po lastnosti (\ref{last:psevdo3}) velja $\sigma(x,y) \leq 2^{-k}$ za vsak $k \in \N$. Od tod očitno sledi $\sigma(x, y) = 0$.

Dokažimo še dodatek. Privzemimo, da velja $xU_kx^{-1} = U_k$ za vsak $x \in G$ in $k \in \N$. Potem za vsako diadično racionalno število $r > 0$ velja
\begin{align*}
xV_rx^{-1} &= xV_{2^{-l_1}}V_{2^{-l_2}}\cdots V_{2^{-l_n}}x^{-1} = xV_{2^{-l_1}}x^{-1}xV_{2^{-l_2}}x^{-1}\cdots xV_{2^{-l_n}}x^{-1} \\
&= xU_{l_1}x^{-1}xU_{l_2}x^{-1}\cdots xU_{l_n}x^{-1} = U_{l_1}U_{l_2}\cdots U_{l_n} \\ 
&= V_{2^{-l_1}}V_{2^{-l_2}}\cdots V_{2^{-l_n}} = V_r.
\end{align*}
Zato za vsaka $x, y \in G$ velja $\varphi(xyx^{-1}) = \inf\lbrace r ; xyx^{-1} \in V_r \rbrace = \inf\lbrace r ; y \in x^{-1}V_rx \rbrace = \inf\lbrace r ; y \in V_r \rbrace = \varphi(y)$.
Za vsake elemente $x, y, a \in G$ od tod sledi
\begin{align*}
\sigma(xa, ya) &= \sup_{z \in G}\lbrace |\varphi(zxa) - \varphi(zya)| \rbrace = \sup_{z \in G}\lbrace |\varphi(azx) - \varphi(azy)| \rbrace \\
&= \sup_{z \in G}\lbrace |\varphi(zx) - \varphi(zy)| \rbrace = \sigma(x, y),
\end{align*}
torej je psevdometrika $\sigma$ tudi desnoinvariantna.
Lastnost (\ref{last:psevdo5}) sledi iz levo in desnoinvariantnosti psevdometrike $\sigma$. Velja
\[ \sigma(x^{-1}, y^{-1}) = \sigma(e, y^{-1}x) = \sigma(y, x) ) = \sigma(x, y). \qedhere\]
\end{dokaz}

V dokazu izreka o metrizabilnosti topološke grupe, ki zadošča separacijskemu aksiomu $T_0$, si bomo pomagali z novo karakterizacijo separacijskega aksioma $T_2$.
\begin{trditev}\label{trd:hauskar}
Topološki prostor $X$ je Hausdorffov natanko tedaj, ko za vsak $x \in X$ velja
\[ \bigcap\lbrace \closure{U} ; U \text{ okolica za } x \rbrace = \lbrace x \rbrace. \]
\end{trditev}

\begin{dokaz}
Vzemimo točko $x \in X$ in poljubno točko $y \neq x$ Hausdorffovega prostora $X$. Potem obstajata taki disjunktni okolici $U$ in $V$ zaporedoma za točki $x$ in $y$, da velja $y \in V \subseteq X\setminus \closure{U}$. Ker $y \notin \closure{U}$, velja $y \notin \bigcap\lbrace \closure{U} ; U \text{ okolica za } x \rbrace$. Ker je bil $y$ poljuben, je $\bigcap\lbrace \closure{U} ; U \text{ okolica za } x \rbrace = \lbrace x \rbrace$.

Pokažimo še obratno trditev. Vzemimo poljubni različni točki $x, y \in X$. Ker je $\bigcap\lbrace \closure{U} ; U \text{ okolica za } x \rbrace = \lbrace x \rbrace$, obstaja neka odprta okolica $U_0$ točke $x$, da $y \notin \closure{U_0}$. Potem sta množici $U_0$ in $X \setminus \closure{U_0}$ disjunktni odprti okolici zaporedoma za točki $x$ in $y$. Ker sta $x$ in $y$ poljubno izbrani različni točki, je $X$ Hausdorffov topološki prostor.
\end{dokaz}

\begin{izrek}\label{izr:metrizabilnost}
	Topološka grupa $G$, ki zadošča separacijskemu aksiomu $T_0$, je me\-tri\-za\-bi\-len topološki prostor natanko tedaj, ko obstaja števna baza odprtih okolic enote.
\end{izrek}

\begin{dokaz}
Če je $G$ metrizabilen topološki prostor, lahko za števno bazo odprtih okolic enote $e$ izberemo kar družino odprtih krogel $\lbrace K(e, 2^{-n}) \rbrace_{n \in \N}$.

Za dokaz obratne trditve predpostavimo, da je $\lbrace V_k \rbrace_{k = 1}^\infty$ števna baza odprtih okolic enote. Induktivno bomo konstruirali novo bazo okolic enote na naslednji način. Najprej definiramo okolico $U_1 = V_1 \cap V_1^{-1}$. Recimo, da smo že skonstruirali okolice $U_1,\dots,U_{k-1}$. 
Po trditvah \ref{trd:okolice} in \ref{trd:sim} lahko
izberemo tako okolico $U_k$, da zanjo velja $U_k \subset U_1 \cap \dots \cap U_{k-1}\cap V_k$, $U_k = U_k^{-1}$ in $U_k^2 \subset U_{k-1}$ za vsak $k \geq 2$.

Ker $G$ zadošča separacijskemu aksiomu $T_0$, je po trditvi \ref{trd:t0haus} $G$ Hausdorffova. Upoštevamo, da je družina $\lbrace U_k \rbrace_{k=1}^\infty$ baza okolic, in trditev \ref{trd:hauskar} ter dobimo
\[ H = \bigcap_{k=1}^\infty U_k = \bigcap\lbrace \closure{U} ; U \text{ okolica enote } e \rbrace = \lbrace e \rbrace. \]
Ker množica $H$ očitno vsebuje vsaj enoto $e$, je neprazna in $H = \lbrace e \rbrace$.
Ker baza $\lbrace U_k \rbrace_{k = 1}^\infty$ zadošča predpostavkam izreka \ref{izr:pseudometrika}, na topološki grupi $G$ obstaja psevdometrika $\sigma$ z lastnostmi (\ref{last:psevdo1})-(\ref{last:psevdo4}).

Po lastnosti (\ref{last:psevdo2}) psevdometrike $\sigma$ je $\sigma(x, y) = 0$ natanko tedaj, ko $y^{-1}x \in H$. Ker je $H = \lbrace e \rbrace$, velja $\sigma(x, y) = 0$ natanko tedaj, ko je $x = y$. Preslikava $\sigma$ je torej metrika na $G$. Preveriti moramo le še, da topologija $\tau$ na $G$ in topologija $\tau_\sigma$, inducirana z metriko $\sigma$, sovpadata.

Po lastnostih (\ref{last:psevdo3}) in (\ref{last:psevdo4}) metrike $\sigma$ za vsak $k \in \N$ velja
\[ \lbrace x \in G ; \sigma(x, e) \leq 2^{-k} \rbrace \subset U_k \subset \lbrace x \in G ; \sigma(x, e) \leq 2^{-k+2} \rbrace.\]
Torej za vsak $k \in \N$ velja
\[ K(e, 2^{-k}) \subset U_k \subset K(e, 2^{-k+2}). \]
Vsaka okolica enote $e$ v topologiji $\tau$ torej vsebuje okolico enote $e$ v topologiji $\tau_\sigma$ in vsaka okolica enote $e$ v topologiji $\tau_\sigma$ vsebuje okolico enote $e$ v topologiji $\tau$.
Po trditvi \ref{izr:bazaokolice} sta topologiji $\tau$ in $\tau_\sigma$  ekvivalentni, iz česar sledi, da je $G$ metrizabilen topološki prostor.
\end{dokaz}


\subsection{Separacijski aksiom $T_{3 \frac{1}{2}}$}
V tem podrazdelku bomo definirali separacijski aksiom $T_{3 \frac{1}{2}}$ in pojem povsem regularnosti. Separacijski aksiom $T_{3 \frac{1}{2}}$ bomo uvrstili med že znane separacijske aksiome iz splošne topologije in pokazali, da sta za topološke grupe separacijska aksioma $T_0$ in $T_{3 \frac{1}{2}}$ ekvivalentna.
\begin{definicija}
	Topološki prostor $X$ zadošča separacijskemu aksiomu $T_{3 \frac{1}{2}}$, če za poljubno zaprto množico $A \subseteq X$ in poljubno točko $b \in X\backslash A$ obstaja taka zvezna funkcija $\psi\colon G \to [0, 1]$, da je $\psi (b) = 1$ in $\psi (x) = 0$ za vsak $x \in A$.
\end{definicija}

\begin{definicija}
	Topološku prostoru, ki zadošča $T_1$ in $T_{3 \frac{1}{2}}$, pravimo \emph{povsem regularen} topološki prostor.
\end{definicija}

\begin{trditev}\label{pos:reghaus}
Naj bo $X$ topološki prostor.
\begin{enumerate}
\item Če je $X$ povsem regularen topološki prostor, je regularen.\label{podtrd:reghaus1}
\item Če je $X$ normalen topološki prostor, je povsem regularen.\label{podtrd:reghaus2}
\end{enumerate}
\end{trditev}

\begin{dokaz}
Za dokaz (\ref{podtrd:reghaus1}) za dano zaprto množico $A \subset X$ izberemo poljubno točko $b \in X \setminus A$. Potem obstaja taka zvezna funkcija $\psi\colon X \to [0, 1]$, da je $\psi(b) = 1$ in $\psi(A) \equiv 0$.
Ker sta množici $[0, \frac{1}{2})$ in $(\frac{1}{2}, 1]$ odprti glede na inducirano evklidsko topologijo na intervalu $[0, 1]$ in je funkcija $\psi$ zvezna, sta množici $\psi^{-1}([0, \frac{1}{2}))$ in $\psi^{-1}((\frac{1}{2}, 1])$ disjunktni odprti okolici za množico $A$ in točko $b$.
S tem smo dokazali, da topološki prostor $X$ zadošča separacijskemu aksiomu $T_3$.

Za dokaz (\ref{podtrd:reghaus2}) naj bo $X$ normalen topološki prostor. 
Vzemimo zaprto množico $A \subset X$ in poljubno točko $b \in X \setminus A$.
Ker je prostor $X$ normalen, je množica $\lbrace b \rbrace$ zaprta,
zato po Urysohnovi karakterizaciji separacijskega aksioma $T_4$ (glej \cite{bib:top}) obstaja zvezna funkcija $\psi\colon X \to [0, 1]$, da je $\psi(b) = 1$ in $\psi(A) \equiv 0$.
\end{dokaz}

Kot smo dokazali v trditvi \ref{trd:t0haus} in posledici \ref{izr:t3}, hitro vidimo, da sta za topološke grupe regularnost in separacijski aksiom $T_0$ ekvivalentna. Že pri teoriji topoloških prostorov pa težko najdemo primer prostora, ki je regularen, ni pa povsem regularen. Tukaj primera ne bomo eksplicitno navajali, saj so vsi primeri izrazito težko razumljivi. Nekaj se jih nahaja v \cite{bib:counterexamples}, posebej pa je potrebno omeniti članek \cite{bib:clanekt3pol}. V njem A.~Mysior namreč poda prvi relativno preprost način konstrukcije regularnega, ne pa povsem regularnega, topološkega prostora. Ta konstrukcija je bila kasneje uporabljena v mnogih knjižnih in internetnih publikacijah.
\begin{izrek}\label{izr:t3pol}
	Topološka grupa, ki zadošča separacijskemu aksiomu $T_0$, je povsem regularen topološki prostor.
\end{izrek}

\begin{dokaz}
Za dano zaprto množico $F$ vzemimo poljuben element $a \in G\setminus F$.
Naj bo $\Ucurl$ baza simetričnih okolic enote $e$ in naj bo $U$ taka odprta okolica elementa $a$, da je $U \cap F = \emptyset$. Potem je $a^{-1}U$ okolica enote $e$ in, ker je leva translacija po trditvi \ref{trd:trans} homeomorfizem, velja $a^{-1}U \cap a^{-1}F = \emptyset$. Vzemimo tak $U_1 \in \Ucurl$, da je $U_1 \subseteq a^{-1}U$. Ker je leva translacija homeomorfizem, je množica $aU_1$ odprta okolica elementa $a$ in $aU_1 \cap F = \emptyset$. Po trditvi \ref{trd:okolice} lahko potem induktivno izberemo take okolice $U_2, U_3,... \in \Ucurl$, da velja $U_{k+1}^2 \subset U_k$ za vsak $k \in \N$. S tem smo zadostili predpostavkam izreka \ref{izr:pseudometrika}, zato obstaja na $G$ psevdometrika $\sigma$ z lastnostmi (\ref{last:psevdo1})-(\ref{last:psevdo4}). Definiramo funkcijo $\psi\colon G \to [0, \infty)$ s predpisom
\[ \psi(x) = 1 - \min\lbrace 1, 2\sigma(x, a)\rbrace. \]
Ker je psevdometrika $\sigma$ po lastnosti (\ref{last:psevdo1}) v izreku \ref{izr:pseudometrika} enakomerno zvezna glede na levo uniformno strukturo na $G$, je po trditvi \ref{trd:enakzveznazvezna} zvezna, iz česar sledi, da je $\psi$ zvezna funkcija.

Ker je po definiciji prevdometrike $\sigma(a, a) = 0$, je $\psi(a) = 1$.
Vzemimo element $x \in F$. Ker po konstrukciji množice $U_1$ velja $a^{-1}x \notin U_1$, po lastnosti (\ref{last:psevdo4}) v izreku \ref{izr:pseudometrika}, $\sigma(x, a) \geq 2^{-1} = \frac{1}{2}$, sledi $\psi(x) = 0$ za vsak element $x \in F$.

Topološka grupa $G$ zadošča separacijskemu aksiomu $T_{3\frac{1}{2}}$. Ker zadošča tudi separacijskemu aksiomu $T_0$, je po trditvi \ref{trd:t0haus} Hausdorffova in zato povsem regularna.
\end{dokaz}

\section{Separacijski aksiom $T_4$}

\subsection{Proste topološke grupe} 
Za dokaz izreka o obstoju povsem regularne topološke grupe, ki ni normalna, potrebujemo pojem \emph{proste grupe}. Vzemimo neprazno množico $X$. \emph{Beseda} je bodisi prazna (pišemo $e$) bodisi končni formalni produkt $x_1^{\delta_1}\cdots x_n^{\delta_n}$ elementov iz $X$, kjer je $\delta_k \in \lbrace -1, 1\rbrace$ za $k = 1,\dots,n$ in $n \in \N$. Beseda je \emph{reducirana}, če je prazna ali pa je $\delta_k = \delta_{k+1}$, kadar je $x_k = x_{k+1}$. Naj bo $F$ množica vseh reduciranih besed nad množico $X$. Na množici $F$ definiramo operacijo na naslednji način: produkt besed $x$ in $y$ je beseda, ki jo dobimo, če besedi $x$ in $y$ najprej staknemo, nato pa rekurzivno okrajšamo vse pare $x_k$, $y_1$, za katere velja $x_n = y_1$ in $\delta_n^x \neq \delta_1^y$, dokler ne dobimo okrajšane besede. Trdimo, da je množica $F$ s tako definirano operacijo grupa. Res, če za enoto $e$ vzamemo prazno besedo, inverz pa definiramo kot $(x_1^{\delta_1}\cdots x_n^{\delta_n})^{-1} = x_n^{-\delta_n}\cdots x_1^{-\delta_1}$, dobimo grupno strukturo, kot smo pokazali pri predmetu Algebra 3.

\begin{izrek}\label{izr:prostatopgrupa}
Za vsak povsem regularen topološki prostor $X$ obstaja taka topološka grupa $F_X$, da velja:
\begin{enumerate}
	\item topološka grupa $F_X$ je prosta grupa nad prostorom $X$,\label{podtrd:prosta1}
	\item topološki prostor $X$ je zaprt podprostor v $F_X$,\label{podtrd:prosta2}
	\item za vsako zvezno preslikavo $\varphi\colon X \to G$, kjer je $G$ poljubna topološka grupa, obstaja zvezen homomorfizem $\Phi\colon F_X \to G$, da je $\Phi(x) = \varphi(x)$ za vsak $x \in X$.\label{podtrd:prosta3}
\end{enumerate}
\end{izrek}

\begin{dokaz}
Popoln dokaz izreka je izredno dolg in obsežen, zato ga tukaj izpustimo. Dokaz se nahaja v \cite[(8.8)]{bib:aha1}, navedli pa bomo le idejo dokaza.

Naj $\aleph_1$ označuje kontinuum, torej $\aleph_1 = |\R|$. Vzemimo povsem regularen topološki prostor $X$. Naj bo $\mathcal{G}$ družina vseh paroma neizomorfnih topoloških grup, za katere velja:
\begin{enumerate}[label=(\roman*)]
\item za vsako topološko grupo $G \in \mathcal{G}$ je $|G| \leq \max\lbrace |X|, \aleph_1\rbrace$,
\item za vsako topološko grupo $H$, za katero je $|H| \leq \max\lbrace |X|, \aleph_1\rbrace$, obstaja taka topološka grupa $G \in \mathcal{G}$, da sta $G$ in $H$ topološko izomorfni.
\end{enumerate}
Definiramo množico $\lbrace (G_\iota, \varphi_\iota)\rbrace_{\iota \in I}$ vseh parov $(G_\iota, \varphi_\iota)$, kjer je $G_\iota \in \mathcal{G}$ in je $\varphi_\iota\colon X \to G_\iota$ zvezna preslikava. Po definiciji družine $\mathcal{G}$ za vsako topološko grupo $H$ in zvezno preslikavo $\varphi\colon X \to H$, kjer velja $|H| \leq \max\lbrace |X|, \aleph_1\rbrace$, obstaja tak indeks $\iota_0$, da sta topološki grupi $G_{\iota_0}$ in $H$ topološko izomorfni s topološkim izomorfizmom $\tau$ in velja $\tau\circ\varphi_{\iota_0} = \varphi$.
V tem primeru identificiramo par $(H, \varphi)$ s parom $(G_{\iota_0}, \varphi_{\iota_0})$.

Definiramo kartezični produkt $G_0 = \Pi_{\iota \in I}G_\iota$ in označimo enoto grupe $G_0$ z $e$. Za vsak $x \in X$ po komponentah definiramo $\nu(x) \in G_0$ tako: $\nu(x)_\iota = \varphi_\iota(x)$. Pokažemo, da je preslikava $\nu\colon X \to \nu(X)$ homeomorfizem.
Od tukaj naprej lahko torej identificiramo prostor $X$ s podmnožico $\nu(X) \subset G_0$ in pojmujemo $X \subset G_0$.
Naj bo potem podgrupa $F_X \leq G$ tista podgrupa, ki je generirana z množico $X$.

Naj $\mathfrak{S}_n$ označuje grupo vseh permutacij množice z $n$ elementi in naj $\mathfrak{U}(n)$ označuje grupo vseh unitarnih matrik velikost $n\times n$. Izberemo si poljubno število $l \in \N$. Potem naj bo za vsako permutacijo $P \in \mathfrak{S}_l$ matrika $U_P \in \mathfrak{U}(l)$ permutacijska matrika, torej je $u_{jk} = 1$, če je $j = P(k)$ in $u_{jk} = 0$ sicer. Preslikava, definirana s predpisom $P \mapsto U_P$, je izomorfizem iz $\mathfrak{S}_l$ v $\mathfrak{U}(l)$.

Da preverimo točko (\ref{podtrd:prosta2}), vzamemo poljubno reducirano besedo $x_1^{\delta_1}\cdots x_n^{\delta_n}$ dolžine $n$, sestavljeno iz elementov prostora $X$.
Obstaja preslikava $A\colon\lbrace x_1, \dots,x_n\rbrace \to \mathfrak{U}(l)$, kjer je $l = n + 1$ ali $l = n + 2$, da velja $A(x_1)^{\delta_1}\cdots A(x_n)^{\delta_n} \neq I_l$, kjer $I_l$ označuje identično matriko velikosti $l \times l$. Ker je topološka grupa $\mathfrak{U}(l)$ povezana s potmi in ker je topološki prostor $X$ povsem regularen, obstaja taka zvezna preslikava $\varphi\colon X \to \mathfrak{U}$, da velja $\varphi(x_k) = A(x_k)$ za vsak $k = 1,\dots,n$. Za nek indeks $\iota_0 \in I$ mora biti par $(\mathfrak{U}(l), \varphi)$ enak ali identificiran s parom $(G_{\iota_0}, \varphi_{\iota_0})$, iz česar sledi
\[ (x_1^{\delta_1}\cdots x_n^{\delta_n})_{\iota_0} = A(x_1)^{\delta_1}\cdots A(x_n)^{\delta_n} \neq I_l, \]
torej velja $x_1^{\delta_1}\cdots x_n^{\delta_n} \neq e$. Z drugimi besedami, $F_X$ je prosta grupa, generirana z elementi iz prostora $X$.

Da dokažemo točko (\ref{podtrd:prosta1}) vzemimo poljuben element množice $F_X\setminus X$. Zapišemo ga lahko kot besedo $x_1^{\delta_1}\cdots x_n^{\delta_n}$, kjer je $n > 1$ ali pa je $n = 1$ in je $\delta_1 = -1$. Z uporabo izomorfizma iz dveh odstavkov nazaj z upoštevanjem, da je $l = n + 1$ ali $l = n + 2$, dobimo take matrike $A(x_1),\dots,A(x_n) \in \mathfrak{U}(l)$, da je matrika $B = A(x_1)^{\delta_1}\cdots A(x_n)^{\delta_n}$ različna od vsake matrike $A(x_k)$ za $k = 1,\dots,n$. Ker je topološka grupa $\mathfrak{U}(l)$ lokalno evklidska, lahko dobimo okolico $\mathcal{B}$ matrike $B$, da je množica $\closure{\mathcal{B}}\cap (\mathfrak{U}(l)\setminus \mathcal{B})$ homeomorfna sferi $\mathcal{S}_{l^2 - 1}$ in da so $A(x_1),\dots,A(x_n) \in \closure{\mathcal{B}}$. Ker je $\mathfrak{U}\cap\mathcal{B}^{\mathsf{c}}$ povezana s potmi, lahko najdemo takšno zvezno preslikavo $\psi\colon X \to \mathfrak{U}\cap\mathcal{B}^{\mathsf{c}}$, da je $\psi(x_k) = A(x_k)$ za vsak $k = 1,\dots,n$. Potem je par $(\mathfrak{U}(l), \psi)$ enak ali identificiran s parom $(G_{\iota_0}, \varphi_{\iota_0})$. Okolica elementa $x_1^{\delta_1}\cdots x_n^{\delta_n}$ v $F_X$ je sestavljena iz vseh $(y_\iota) \in F_X$, za katere $y_{\iota_0}\in\mathcal{B}$ ne fiksira nobene točke iz prostora $X$. Od tod sledi, da je $F_X \setminus X$ odprta množica, torej je $X$ zaprt podprostor v $F_X$.

Da dokažemo točko (\ref{podtrd:prosta3}) le vzamemo zvezno preslikavo $\varphi\colon X \to H$, kjer je $H$ poljubno izbrana topološka grupa, za katero velja $\varphi(X) \subseteq J$, kjer je $J$ podgrupa topološke grupe $H$ in $|J| \leq \max\lbrace |X|, \aleph_1\rbrace$. Potem je par $(J, \varphi)$ enak ali identificiran z nekim parom $(G_{\iota_0}, \varphi_{\iota_0})$. Preslikava $\varphi$ je le projekcija na $\iota_0$-to komponento v $F_X \subset \Pi_{\iota \in I}G_\iota$ in jo lahko razširimo do zveznega homomorfizma iz $F_X$ v $J \subseteq H$.
\end{dokaz}

\begin{izrek}\label{izr:prostaizo}
Naj bo $X$ povsem regularen topološki prostor, $F_X$ prosta topološka grupa nad $X$ in naj bo $\widetilde{F}$ taka topološka grupa, da zanjo velja:
\begin{enumerate}
	\item topološki prostor $X$ je topološki podprostor v $\widetilde{F}$,\label{last:prosta1}
	\item topološka grupa $\widetilde{F}$ je najmanjša zaprta podgrupa v $\widetilde{F}$, ki vsebuje $X$,\label{last:prosta2}
	\item za vsako zvezno preslikavo $\varphi\colon X \to G$, kjer je $G$ poljubna topološka grupa, obstaja zvezen homomorfizem $\Phi\colon \widetilde{F} \to G$, da je $\Phi(x) = \varphi(x)$ za vsak $x \in X$.\label{last:prosta3}
\end{enumerate}
Tedaj obstaja topološki izomorfizem $\tau\colon F_X \to \widetilde{F}$, da je $\tau(x) = x$ za vsak $x \in X$.
\end{izrek}

\begin{dokaz}
Iz točke (\ref{podtrd:prosta3}) izreka \ref{izr:prostatopgrupa} in lastnosti (\ref{last:prosta3}) sledi, da obstajata zvezen homomorfizem $\Phi\colon F_X \to \widetilde{F}$ in zvezen homomorfizem $\widetilde{\Phi}\colon \widetilde{F} \to F_X$, da velja $\Phi(x) = \widetilde{\Phi}(x) = x$ za vsak $x \in X$. Kompozitum $\widetilde{\Phi}\circ\Phi\colon F_X \to F_X$ je zvezen homomorfizem, ki je identična preslikava na prostoru $X$. Ker je $F_X$ po točki (\ref{podtrd:prosta2}) izreka \ref{izr:prostatopgrupa} prosta grupa nad $X$, je kompozitum $\widetilde{\Phi}\circ\Phi\colon F_X \to F_X$ identična preslikava na $F_X$. Kompozitum $\Phi\circ\widetilde{\Phi}$ je identična preslikava na podgrupi topološke grupe $\widetilde{F}$, ki je generirana z $X$. Ker je ta podgrupa po lastnosti (\ref{last:prosta2}) gosta v topološki grupi $\widetilde{F}$ in ker je $\Phi\circ\widetilde{\Phi}$ zvezna preslikava, je kompozitum $\Phi\circ\widetilde{\Phi}$ identična preslikava na $\widetilde{F}$. Od tod sledi, da je $\widetilde{\Phi} = \Phi^{-1}$, torej lahko vzamemo $\tau = \Phi$.
\end{dokaz}

\begin{izrek}
Obstaja povsem regularna topološka grupa, ki ni normalna.
\end{izrek}

\begin{dokaz}
Naj bo $X$ poljuben povsem regularen topološki prostor, ki ni normalen. Po izreku \ref{izr:prostatopgrupa} je $X$ zaprt v prosti topološki grupi $F_X$. Ker je vsak zaprt topološki podprostor normalnega prostora normalen, $F_X$ ne more biti normalen topološki prostor.
\end{dokaz}

Oglejmo si še konkreten primer topološke grupe, ki je povsem regularna, vendar ni normalna.

\begin{izrek}\label{izr:t4protiprimer}
Če je $m$ neštevno kardinalno število, potem je $\Z^{m}$ povsem regularna topološka grupa, ki ni normalna.
\end{izrek}

\begin{dokaz}
Ker topološka grupa $\Z$ glede na inducirano evklidsko topologijo zadošča separacijskemu aksiomu $T_0$ in je separacijski aksiom $T_0$ multiplikativna lastnost, je tudi $\Z^m$ topološka grupa, ki zadošča separacijskemu aksiomu $T_0$, zato je po izreku \ref{izr:t3pol} topološka grupa $\Z^m$ povsem regularna. Za dokaz nenormalnosti pišimo $\Z^m$ raje kot kartezični produkt $\Pi_{\lambda \in \Lambda}\Z_\lambda$, kjer je $|\Lambda| = m$ in $\Z_\lambda \cong \Z$ za vsak $\lambda \in \Lambda$. Definirajmo množici
\[ A = \lbrace (x_\lambda) \in \Z^m ; \text{ za vsak $n \in \Z$ in $n \neq 0$ obstaja največ en $\lambda \in \Lambda$, da je $x_\lambda = n$}\rbrace \]
in
\[ B = \lbrace (x_\lambda) \in \Z^m ; \text{ za vsak $n \in \Z$ in $n \neq 1$ obstaja največ en $\lambda \in \Lambda$, da je $x_\lambda = n$}\rbrace. \]
Če $(x_\lambda) \notin A$, potem obstajata različna indeksa $\lambda_0, \lambda_1 \in \Lambda$, da velja $x_{\lambda_0} = x_{\lambda_1} = n$ za nek $n \in \Z$ in $n \neq 0$. Ker so vse projekcijske preslikave $\pr_{\lambda}\colon \Z^m \to \Z_\lambda$ zvezne, je $\lbrace (y_\lambda) \in \Z^m ; y_{\lambda_0} = y_{\lambda_1} = n \rbrace$ odprta množica, ki vsebuje $(x_\lambda)$, in je disjunktna množici $A$. Vsaka točka iz $\Z^m\setminus A$ ima torej odprto okolico, ki je disjunktna množici $A$, iz česar sledi, da je $\Z^m\setminus A$ odprta množica, zato je množica $A$ zaprta. Z enakim premislekom utemeljimo, da je tudi množica $B$ zaprta. Množici $A$ in $B$ sta očitno disjunktni. Res, vsak element $(x_\lambda) \in A$ ima po konstrukciji množice $A$ na neštevno mnogo indeksih vrednost $0$ in zato $(x_\lambda) \notin B$. Premislek lahko ponovimo še v obratni smeri.

Vzemimo poljubni dve odprti okolici $U$ in $V$ zaporedoma za množici $A$ in $B$. Pokazali bomo, da velja $U \cap V \neq \emptyset$, iz česar bo takoj sledilo, da topološka grupa $Z^m$ ni normalen topološki prostor.

Naj ima $(x_\lambda^{(1)}) \in \Z^m$ vrednost $0$ za vsak indeks $\lambda \in \Lambda$. Očitno je $(x_\lambda^{(1)}) \in A \subset U$, zato obstaja tako število $m_1 \in \N$ in taki paroma različni indeksi $\lambda_1,\dots, \lambda_{m_1} \in \Lambda$, da velja
\[ (x_\lambda^{(1)}) \in \lbrace (x_\lambda) \in \Z^m ; x_{\lambda_k} = 0 \text{ za } k=1,\dots,m_1\rbrace \subset U. \]
Naj ima $(x_\lambda^{(2)}) \in \Z^m$ vrednost $k$ na indeksih $\lambda_k$, kjer je $1 \leq k \leq m_1$, in vrednost $0$ drugod. Ker je $(x_\lambda^{(2)}) \in A \subset U$, obstaja tako število $m_2 \in \N$, kjer $m_2 > m_1$, in taki paroma različni indeksi $\lambda_{m_1+1},\dots,\lambda_{m_2} \in \Lambda$, različni od vseh indeksov $\lambda_1,\dots, \lambda_{m_1}$, da velja
\begin{align*}
(x_\lambda^{(2)}) \in \lbrace &(x_\lambda) \in \Z^m ; x_{\lambda_k} = k \text{ za } k=1,\dots,m_1 \text{ in }\\
& x_{\lambda_k} = 0 \text{ za } k = m_1+1,\dots, m_2\rbrace \subset U.
\end{align*}
Tako nadaljujemo in induktivno definiramo zaporedje $\lbrace (x_\lambda^{(n)}) \rbrace_{n = 1}^\infty$ elementov topološke grupe $\Z^m$, zaporedje indeksov$\lbrace\lambda_k\rbrace_{k = 1}^\infty$ in strogo naraščajoče zaporedje naravnih števil $\lbrace m_n \rbrace_{n = 1}^\infty$ na naslednji način. Če smo že definirali $(x_\lambda^{(n-1)})$ in paroma različne indekse $\lambda_{m_{n-2}+1},\dots, \lambda_{m_{n-1}}$, naj ima $(x_\lambda^{(n)})$ vrednost $k$ na indeksih $\lambda_k$, kjer je $1 \leq k \leq m_{n-1}$, in vrednost $0$ sicer. Ker je $(x_\lambda^{(n)}) \in A \subset U$, obstaja tako število $m_n \in \N$, kjer $m_n > m_{n-1}$, in taki paroma različni indeksi $\lambda_{m_{n-1}+1},\dots,\lambda_{m_n}$, različni od vseh prej tako definiranih indeksov, da velja
\begin{align*}
(x_\lambda^{(n)}) \in \lbrace &(x_\lambda) \in \Z^m ; x_{\lambda_k} = k \text{ za } k=1,\dots,m_{n-1} \text{ in }\\
& x_{\lambda_k} = 0 \text{ za } k = m_{n-1}+1,\dots, m_n\rbrace \subset U.
\end{align*}

Definirajmo še $(y_\lambda) \in \Z^m$. Naj bo $(y_\lambda) = k$, če je $\lambda = \lambda_k$ za vsak $k \in \N$ in naj bo $y_\lambda = 1$ drugod. Očitno je $(y_\lambda) \in B$, zato za neko končno podmnožico $K \subset \Lambda$ velja
\[ \lbrace (x_\lambda) \in \Z^m ; x_\lambda = y_\lambda \text{ za vse } \lambda \in K \rbrace \subset V. \]
Ker je množica $K$ končna, obstaja tak $n_0 \in \N$, da $\lambda_k \notin K$ za vse $k > m_{n_0}$.

Definirajmo še $(z_\lambda) \in \Z^m$ na naslednji način:
\begin{align*}
	z_\lambda &= k \text{, če je $\lambda = \lambda_k$ in $k \leq m_{n_0}$}, k \in \N; \\
	z_\lambda &= 0 \text{, če je $\lambda = \lambda_k$ in $m_{n_0} + 1 \leq k \leq m_{n_0+1}$}, k \in \N; \\
	z_\lambda &= 1 \text{ sicer.}
\end{align*}
Potem je $(z_\lambda) \in \lbrace (x_\lambda) \in \Z^m ; x_\lambda = y_\lambda \text{ za vse } \lambda  \in K \rbrace \subset V$ in hkrati
\begin{align*}
(z_\lambda) \in \lbrace &(x_\lambda) \in \Z^m ; x_{\lambda_k} = k \text{ za } k=1,\dots,m_{n_0} \text{ in }\\
& x_{\lambda_k} = 0 \text{ za } k = m_{n_0}+1,\dots, m_{n_0 + 1}\rbrace \subset U.
\end{align*}
Od tod sledi, da $U \cap V \neq \emptyset$.
\end{dokaz}

\subsection{Parakompaktni topološki prostori}
V tem podrazdelku bomo dokazali, da je ključni pogoj, ki regularni topološki grupi manjka do normalnosti, lokalna kompaktnost. Najprej bomo vpeljali pojem parakompaktnega topološkega prostora. Za regularne topološke prostore bomo najprej navedli in dokazali alternativne definicije parakompaktnosti, nato pa bomo pokazali, da je vsak parakompakten Hausdorffov topološki prostor normalen. Nazadnje bomo dokazali, da je vsaka lokalno kompaktna Hausdorffova topološka grupa parakompaktna, torej normalen topološki prostor.

\begin{definicija}\label{def:parakompakt}
	\begin{enumerate}
		\item Naj bosta $\mathcal{U}$ in $\mathcal{V}$ družini podmnožic topološkega prostora $X$. Družina $\mathcal{V}$ je \emph{pofinitev} družine $\mathcal{U}$, če za vsako množico $V \in \mathcal{V}$ obstaja takšna množica $U \in \mathcal{U}$, da je $V \subset U$.
		\item Družina podmnožic $\mathcal{U}$ topološkega prostora $X$ je \emph{lokalno končna}, če ima vsaka točka $x \in X$ okolico, ki seka samo končno mnogo množic iz družine $\mathcal{U}$.
		\item Družina podmnožic je \emph{$\sigma$-lokalno končna}, če je števna unija lokalno končnih družin podmnožic.
		\item Topološki prostor $X$ je \emph{parakompakten}, če ima vsako njegovo odprto pokritje kakšno pofinitev, ki je lokalno končno odprto pokritje prostora $X$.
	\end{enumerate}
\end{definicija}

\begin{opomba}\label{opo:lokkon} % v zvezku STOP
Iz splošne topologije vemo, da za lokalno končno družino $\lbrace C_\lambda \rbrace_{\lambda \in \Lambda}$ velja
\[ \closure{\bigcup_{\lambda \in \Lambda} C_\lambda} = \bigcup_{\lambda \in \Lambda} \closure{C_\lambda}. \]
\end{opomba}

\begin{trditev}\label{trd:lokkonzap}
Za lokalno končno družino $\lbrace U_\lambda ; \lambda \in \Lambda \rbrace$ podmnožic topološkega prostora $X$ veljajo naslednje trditve:
\begin{enumerate} 
\item Lokalno končna je tudi družina $\lbrace \closure{U_\lambda} ; \lambda \in \Lambda \rbrace$.\label{podtrd:lokkonzap1}
\item Za vsako podmnožico indeksov $I \subseteq \Lambda$ je unija $\bigcup_{\lambda \in I}\closure{U_\lambda}$ zaprta v $X$.\label{podtrd:lokkonzap2}
\end{enumerate}
\end{trditev}

\begin{dokaz}
Za dokaz točke (\ref{podtrd:lokkonzap1}) izberimo $x \in X$ in poiščimo takšno odprto okolico $V_x$ točke $x$, da velja $U_\lambda \cap V_x = \emptyset$ za vse, razen za končno indeksov $\lambda \in \Lambda$. Ker je množica $X \setminus V_x$ zaprta, velja
\[ U_\lambda \cap V_x = \emptyset \implies U_\lambda \subseteq X \setminus V_x \implies \closure{U_\lambda} \subseteq X \setminus V_x \implies \closure{U_\lambda} \cap V_x = \emptyset \]
za vse, razen za končno indeksov $\lambda \in \Lambda$, iz česar sledi, da je tudi $\lbrace \closure{U_\lambda} ; \lambda \in \Lambda \rbrace$ lokalno končna družina.

Za dokaz trditve (\ref{podtrd:lokkonzap2}) vzemimo podmnožico indeksov $I \subseteq \Lambda$ in definiramo množico $B = \bigcup_{\lambda \in I}\closure{U_\lambda}$. Ker je vsaka poddružina lokalno končne družine očitno lokalno končna, po točki (\ref{podtrd:lokkonzap1}) obstaja za poljuben $x \in X \setminus B$ okolica $V_x$, da ima z največ končno množicami $\closure{U_\lambda}$, $\lambda \in I$, neprazen presek. Naj bodo te množice $\closure{U_{\lambda_1}},\dots,\closure{U_{\lambda_n}}$. Potem je množica $V_x \cap (\bigcap_{k=1}^n(X \setminus \closure{U_{\lambda_k}}))$ okolica točke $x$ ki ne seka množice $B$. Ker je $x \in X \setminus B$ poljubno izbran, je množica $X \setminus B$ odprta, zato je $B$ zaprta v $X$.
\end{dokaz}

\begin{trditev}\label{trd:lokkonvlozitev}
Naj bo $\lbrace E_\alpha ; \alpha \in \mathcal{A} \rbrace$ družina podmnožic topološkega prostora $X$ in naj bo $\lbrace B_\beta ; \beta \in \mathcal{B} \rbrace$ tako lokalno končno pokritje prostora $X$ iz zaprtih množic, da za vsak $\beta \in \mathcal{B}$ množica $B_\beta$ seka kvečjemu končno mnogo množic $E_\alpha$.
Tedaj za vsak $\alpha \in \mathcal{A}$ obstaja takšna odprta množica $U_\alpha$, da je $E_\alpha \subseteq U_\alpha$, družina množic $\lbrace U_\alpha ; \alpha \in \mathcal{A} \rbrace$ pa je lokalno končna.
\end{trditev}

\begin{dokaz}
Za vsak $\alpha \in \mathcal{A}$ definiramo množico
\[ U_\alpha = X \setminus \bigcup\lbrace B_\beta ; B_\beta \cap E_\alpha = \emptyset\rbrace. \]
Po trditvi \ref{trd:lokkonzap} je za vsak $\alpha \in \mathcal{A}$ množica $U_\alpha$ odprta. Preverimo, da je družina množic $\lbrace U_\alpha ; \alpha \in \mathcal{A} \rbrace$ lokalno končna. Vsak $x \in X$ ima okolico $V_x$, ki leži v končni uniji $\bigcup_{i = 1}^n B_{\beta_i}$. Ker $B_\beta \cap U_\alpha \neq \emptyset$ natanko tedaj, kadar $B_\beta \cap E_\alpha \neq \emptyset$, in ker po predpostavki vsak $B_{\beta_i}$ seka kvečjemu končno množic $E_\alpha$, unija $\bigcup_{i = 1}^n B_{\beta_i}$ seka kvečjemu končno množic $U_\alpha$, zato tudi $V_x$ seka kvečjemu končno množic $E_\alpha$. Po konstrukciji je $E_\alpha \subseteq U_\alpha$.
\end{dokaz}

Definicija \ref{def:parakompakt} parakompaktnega topološkega prostora $X$ je splošna, v naslednji trditvi pa si bomo ogledali še nekaj alternativnih definicij, če je topološki prostor $X$ regularen.

\begin{trditev}\label{trd:parakompkar}
Za regularen topološki prostor $X$ so naslednje trditve ekvivalentne:
\begin{enumerate}
\item Topološki prostor $X$ je parakompakten.\label{podtrd:parakompkar1}
\item Za vsako odprto pokritje $\mathcal{U}$ topološkega prostora $X$ obstaja $\sigma$-lokalno končna pofinitev pokritja $\mathcal{U}$ iz odprtih množic, ki je tudi sama pokritje prostora $X$.\label{podtrd:parakompkar2}
\item Za vsako odprto pokritje $\mathcal{U}$ topološkega prostora $X$ obstaja lokalno končna pofinitev pokritja $\mathcal{U}$, ki je tudi sama pokritje prostora $X$.\label{podtrd:parakompkar3}
\item Za vsako odprto pokritje $\mathcal{U}$ obstaja lokalno končna pofinitev pokritja $\mathcal{U}$ iz zaprtih množic, ki je tudi sama pokritje prostora $X$.\label{podtrd:parakompkar4}
\end{enumerate}
\end{trditev}

\begin{dokaz}
(\ref{podtrd:parakompkar1}) $\Rightarrow$ (\ref{podtrd:parakompkar2}):
Ker je vsaka lokalno končna družina podmnožic tudi $\sigma$-lokalno končna, ta implikacija očitno drži.

(\ref{podtrd:parakompkar2}) $\Rightarrow$ (\ref{podtrd:parakompkar3}):
Vzemimo poljubno odprto pokritje $\Ucurl$ topološkega prostora $X$. Po točki (\ref{podtrd:parakompkar2}) obstaja pofinitev iz odprtih množic $\mathcal{V} = \lbrace V_{n,\lambda} ; (n, \lambda) \in \N \times \Lambda \rbrace$, kjer je za vsak $n_0 \in \N$ družina $\lbrace V_{n_0, \lambda} ; \lambda \in \Lambda\rbrace$ lokalno končna, ki ni nujno pokritje. Za vsak $n \in \N$ definirajmo $W_n = \bigcup_{\lambda \in \Lambda} V_{n, \lambda}$. Potem je družina $\lbrace W_n ; n \in \N \rbrace$ odprto pokritje prostora $X$. Sedaj definiramo množico $A_1 = W_1$, nato pa za vsak $i > 1$, $i \in \N$, definiramo množico $A_i = W_i \setminus (\bigcup_{j=1}^{i-1}W_j)$. Ker je $A_i \subseteq W_i$ za vsak $i \in \N$, je družina $\lbrace A_i ; i \in \N \rbrace$ pofinitev pokritja $\lbrace W_n ; n \in \N \rbrace$,prav tako pa je pokritje prostora $X$. Res, za vsak $x \in X$ je $x \in A_{n(x)}$, kjer je $n(x) \in \N$ najmanjše naravno število, da je $x \in W_{n(x)}$. Ker okolica $W_{n(x)}$ točke $x$ po konstrukciji množic $\lbrace A_i ; i \in \N\rbrace$ ne seka nobene množice $A_i$ za $i > n(x)$, je družina $\lbrace A_i ; i \in \N \rbrace$ lokalno končna.

Trdimo, da je družina množic $\lbrace A_n \cap V_{n, \lambda}\rbrace$ iskano pokritje. Ker je podpokritje $\mathcal{V}$ pofinitev pokritja $\Ucurl$, je tudi $\lbrace A_n \cap V_{n, \lambda}\rbrace$ pofinitev pokritja $\Ucurl$. Vzemimo poljuben $x \in X$. Ker ima po zgoraj dokazanem točka $x$ okolico, ki seka kvečjemu končno mnogo množic $A_n$, in ker je za vsak $n_0 \in \N$ družina $\lbrace V_{n_0, \lambda} ; \lambda \in \Lambda \rbrace$ lokalno končna, je tudi $\lbrace A_n \cap V_{n, \lambda}\rbrace$ lokalno končna družina množic.

(\ref{podtrd:parakompkar3}) $\Rightarrow$ (\ref{podtrd:parakompkar4}):
Vzemimo poljubno odprto pokritje $\Ucurl$ topološkega prostora $X$. Za vsak $x \in X$ izberemo množico $U_x \in \Ucurl$, da je $x \in U_x$. Ker je $X$ regularen topološki prostor, obstaja taka odprta množica $V_x$, da velja $x \in V_x \subseteq \closure{V_x} \subseteq U_x$. Ker je družina množic $\mathcal{V} = \lbrace V_x ; x \in X \rbrace$ odprto pokritje prostora $X$, po točki (\ref{podtrd:parakompkar3}) obstaja lokalno končna pofinitev $\lbrace A_x ; x \in X \rbrace$ pokritja $\mathcal{V}$, ki je tudi sama pokritje prostora $X$. Po trditvi \ref{trd:lokkonzap} je tudi družina $\lbrace \closure{A_x} ; x \in X \rbrace$ lokalno končna. Ker za vsak $x \in X$ velja $\closure{A_x} \subseteq \closure{V_x} \subseteq U_x$, je družina množic $\lbrace \closure{A_x} ; x \in X \rbrace$ iskana pofinitev iz zaprtih množic.

(\ref{podtrd:parakompkar4}) $\Rightarrow$ (\ref{podtrd:parakompkar1}):
Vzemimo poljubno odprto pokritje $\Ucurl$ topološkega prostora $X$. Po točki (\ref{podtrd:parakompkar4}) obstaja lokalno končna pofinitev $\mathcal{E}$ pokritja $\Ucurl$ iz zaprtih množic, ki je tudi sama pokritje za prostor $X$. Potem ima vsak $x \in X$ okolico $V_x$, ki seka kvečjemu končno množic iz $\mathcal{E}$. Po točki (\ref{podtrd:parakompkar4}) obstaja lokalno končna pofinitev $\mathcal{B} = \lbrace B \rbrace$ pokritja $\lbrace V_x ; x \in X \rbrace$ iz zaprtih množic. Ker vsaka množica iz $\mathcal{B}$ seka kvečjemu končno množic iz $\mathcal{E}$, po trditvi \ref{trd:lokkonvlozitev} za vsako množico $E \in \mathcal{E}$ obstaja odprta množica $G_E$, ki vsebuje množico $E$, da je družina $\lbrace G_E ; E \in \mathcal{E} \rbrace$ lokalno končna. Ker je $\mathcal{E}$ pofinitev pokritja $\Ucurl$, naj bo za vsako množico $E \in \mathcal{E}$ množica $U_E \in \Ucurl$ neka množica, ki vsebuje $E$. Ker so vse množice $G_E$ in $U_E$ odprte in ker je družina množic $\mathcal{E}$ pokritje prostora $X$, je družina $\lbrace G_E \cap U_E ; E \in \mathcal{E} \rbrace$ lokalno končna pofinitev pokritja $\Ucurl$ iz odprtih množic, ki je tudi sama pokritje prostora $X$.
\end{dokaz}

\begin{definicija}
Topološki prostor ima Lindel\"ofovo lastnost, če vsako njegovo odprto pokritje vsebuje števno podpokritje.
\end{definicija}

Očitno ima vsak kompakten prostor Lindel\"ofovo lastnost, saj vsako odprto pokritje vsebuje končno podpokritje, ki je očitno števno. Velja pa tudi naslednje.
\begin{trditev}\label{trd:lindel}
Vsak $\sigma$-kompakten prostor ima Lindel\"ofovo lastnost.
\end{trditev}
\begin{dokaz}
Ker je $\sigma$-kompakten prostor unija števno mnogo kompaktnih prostorov in ker vsako odprto pokritje vsakega od njih vsebuje končno podpokritje, vsako odprto pokritje $\sigma$-kompaktnega prostora vsebuje pod\-pok\-rit\-je, ki je sestavljeno iz števno mnogo končnih pokritji. Ker je števna unija končnih množic števna množica, je to podpokritje števno.
\end{dokaz}

\begin{trditev}\label{trd:parkompnorm} % Mrčun, Topologija, Trditev 4.21
Vsak parakompakten Hausdorffov topološki prostor je normalen.
\end{trditev}

\begin{dokaz}
Vzemimo zaprti, neprazni in disjunktni podmnožici $A$ in $B$ parakompaktnega Hausdorff\-o\-ve\-ga prostora $X$.

Vzemimo najprej poljubno točko $b \in B$. Ker je prostor $X$ Hausdorffov, sta vsaki dve točki ločeni z disjunktnima okolicama. Za vsako točko $a \in A$ torej obstaja taka odprta množica $Q_a \subset X$, da je $a \in Q_a$ in $b \in X \setminus \closure{Q_a}$. Ker je $A$ zaprta, je $X \setminus A$ odprta, iz česar sledi, da je $\mathcal{W} = (X \setminus A)\cup \lbrace Q_a\rbrace_{a \in A}$ odprto pokritje prostora $X$. Ker je $X$ parakompakten topološki prostor, obstaja lokalno končno odprto pokritje $\mathcal{W}'$ prostora $X$, ki je pofinitev pokritja $\mathcal{W}$.

Oglejmo si družino množic \[ \mathcal{Q} = \lbrace W \in \mathcal{W}'; W \cap A \neq \emptyset \rbrace. \]
Ker je pofinitev $\mathcal{W}'$ lokalno končna, je tudi družina množic $\mathcal{Q}$ lokalno končna.
Za vsako množico $W \in \mathcal{Q}$ po definiciji pofinitve obstaja takšna točka $a \in A$, da je $W \subset Q_a$.
 Potem velja $b \in X \setminus \closure{Q_a} \subset X \setminus \closure{W}$. Ker je $\mathcal{W}'$ odprto pokritje prostora $X$, je $S = \bigcup\mathcal{Q}$ odprta okolica množice $A$, in ker je $\mathcal{Q}$ lokalno končna družina, po opombi \ref{opo:lokkon} velja \[ b \in X \setminus \bigcup_{W \in \mathcal{Q}} \closure{W} = X \setminus \closure{S}. \] Množica $T_b = X \setminus \closure{S}$ je odprta okolica točke $b$, ki je disjunktna z množico $S$.

Po zgoraj dokazanem za vsako točko $b \in B$ obstaja odprta okolica $T_b$ točke $b$, da je $A \cap \closure{T_b} = \emptyset$, zato je $\Ucurl = (X\setminus B) \cup \lbrace T_b \rbrace_{b \in B}$ odprto pokritje prostora $X$. Ker je $X$ parakompakten topološki prostor, obstaja lokalno končno odprto pokritje $\Ucurl'$, ki je pofinitev pokritja $\Ucurl$. Naj bo \[ \mathcal{V} = \lbrace U \in \Ucurl' ; U \cap B \neq \emptyset \rbrace. \]
Ker je družina $\Ucurl'$ pofinitev pokritja $\Ucurl$, za vsako množico $U \in \mathcal{V}$ obstaja takšna točka $b \in B$, da je $U \subset T_b$, iz česar sledi $A \cap \closure{U} \subset A \cap \closure{T_b} = \emptyset$. Ker je množica $V = \bigcup \mathcal{V}$ odprta okolica množice $B$ in ker je $\mathcal{V}$ lokalno končna družina, po opombi \ref{opo:lokkon} velja \[ A \cap \closure{V} = A \cap \bigcup_{U \in \mathcal{V}}\closure{U} = \emptyset.\]

Množica $X \setminus \closure{V}$ je torej odprta okolica množice $A$, ki je disjunktna z odprto okolico $V$ množice $B$. Hausdorffov topološki prostor $X$ s tem zadošča separacijskemu aksiomu $T_4$ in je zato normalen.
\end{dokaz}

\begin{izrek}\label{izr:t4}
	Vsaka lokalno kompaktna topološka grupa, ki zadošča separacijskemu aksiomu $T_0$, je parakompakten topološki prostor.
\end{izrek}

\begin{dokaz}
V dokazu izreka \ref{izr:odpzapsigma} smo videli, da je podgrupa $L = \bigcup_{n = 1}^\infty U^n = \bigcup_{n = 1}^\infty \closure{U}^n$ $\sigma$-kompaktna, ki ima po trditvi \ref{trd:lindel} Lindel\"ofovo lastnost. Ker je leva translacija po trditvi \ref{trd:trans} homeomorfizem, ima za vsak $x \in G$ tudi levi odsek $xL$ Lindel\"ofovo lastnost.

Vzemimo poljubno odprto pokritje $\mathcal{V}$ topološke grupe $G$ in točko $x \in G$.
Ker je $\mathcal{V}$ pokritje tudi za levi odsek $xL \subseteq G$, obstaja števna poddružina $\lbrace V_{xL}^{(n)} \rbrace_{n = 1}^{\infty}$ pokritja $\mathcal{V}$, da je $xL \subseteq \bigcup_{n=1}^{\infty}V_{xL}^{(n)}$. Za vsak $n \in \N$ definirajmo družino množic $\mathcal{W}_n = \lbrace V_{xL}^{(n)} \cap (xL) ; xL \in G/L \rbrace$.
Trdimo, da je družina množic $\mathcal{W} = \bigcup_{n=1}^{\infty}\mathcal{W}_n$ pofinitev pokritja $\mathcal{V}$. Res, za vsako množico $(V_{xL}^{(n)} \cap (xL)) \in \mathcal{W}$, kjer je $x \in G$ in $n \in \N$, velja
\[ V_{xL}^{(n)} \cap (xL) \subseteq V_{xL}^{(n)} \in \mathcal{V}. \]
Ker je vsak $x \in G$ vsebovan v natanko enem odseku $xL \subseteq G$, je družina množic $\mathcal{W}_n$ lokalno končna, torej je $\mathcal{W}$ $\sigma$-lokalno končna pofinitev pokritja $\mathcal{V}$, ki je tudi sama pokritje topološke grupe $G$. Ker je $G$ po posledici \ref{izr:t3} regularna topološka grupa, je po trditvi \ref{trd:parakompkar} topološka grupa $G$ parakompaktna.
\end{dokaz}

\begin{posledica}
Vsaka lokalno kompaktna topološka grupa, ki zadošča separacijskemu aksiomu $T_0$, je normalen topološki prostor.
\end{posledica}

\begin{dokaz}
Naj bo $G$ lokalno kompaktna topološka grupa, ki zadošča separacijskemu aksiomu $T_0$. Po trditvi \ref{trd:t0haus} je $G$ Hausdorffova, po izreku \ref{izr:t4} pa je $G$ parakompaktna. Ker je po trditvi \ref{trd:parkompnorm} vsak parakompakten Hausdorffov topološki prostor normalen, je $G$ normalna topološka grupa.
\end{dokaz}

\section*{Slovar strokovnih izrazov}

\geslo{completely regular}{povsem regularen}
\geslo{coset}{odsek}
\geslo{dyadic number}{diadično število}
\geslo{free group}{prosta grupa}
\geslo{left invariant}{levoinvarianten}
\geslo{locally finite}{lokalno končen}
\geslo{metrizable}{metrizabilen}
\geslo{natural mapping}{naravna preslikava}
\geslo{normal subgroup}{podgrupa edinka}
\geslo{paracompact}{parakompakten}
\geslo{pseudo-metric}{psevdometrika}
\geslo{refinement}{pofinitev}
\geslo{symmetric neighbourhood}{simetrična okolica}
\geslo{uniform structure}{uniformna struktura}
\geslo{uniformly continuous}{enakomerno zvezen}

% seznam uporabljene literature
\begin{thebibliography}{99}

\bibitem{bib:uniform}
S.~Bhowmik, \emph{Introduction to Uniform Spaces}, 10.13140/RG.2.1.3743.8967, junij 2014, [ogled 1.~4.~2019], dostopno na \url{https://www.researchgate.net/publication/305196408_INTRODUCTION_TO_UNIFORM_SPACES}.
\bibitem{bib:dugundji}
J.~Dugundji, \emph{Topology}, Allyn and Bacon series in advanced mathematics, Allyn and Bacon, 1966
\bibitem{bib:aha1}
E.~Hewitt in K.~A.~Ross, \emph{Abstact Harmonic Analysis I}, Springer-Verlag, New York, 1979.
\bibitem{bib:top}
J.~Mrčun, \emph{Topologija}, Izbrana poglavja iz matematike in računalništva \textbf{44} DMFA-založništvo, Ljubljana, 2008.
\bibitem{bib:clanekt3pol}
A.~Mysior, \emph{A Regular Space which is not Completely Regular}, Proc. Amer. Math. Soc. \textbf{81} (1981), 652--653
\bibitem{bib:counterexamples}
J.~A.~Seebach, Jr. in L.~A.~Steen, \emph{Counterexamples in Topology, Second Edition}, Springer-Verlag, New York, 1978

\end{thebibliography}

\end{document}

