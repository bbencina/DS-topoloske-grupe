\documentclass[mat1]{fmfdelo}
% \documentclass[fin1]{fmfdelo}
% \documentclass[isrm1]{fmfdelo}
% \documentclass[mat2]{fmfdelo}
% \documentclass[fin2]{fmfdelo}
% \documentclass[isrm2]{fmfdelo}

% naslednje ukaze ustrezno napolnite
\avtor{Benjamin Benčina}

\naslov{Topološke grupe}
\title{Topological groups}

% navedite ime mentorja s polnim nazivom: doc.~dr.~Ime Priimek,
% izr.~prof.~dr.~Ime Priimek, prof.~dr.~Ime Priimek
% uporabite le tisti ukaz/ukaze, ki je/so za vas ustrezni
\mentor{doc.~dr.~Marko Kandić}
% \mentorica{}
% \somentor{}
% \somentorica{}
% \mentorja{}{}
% \mentorici{}{}

\letnica{2019} % leto diplome

%  V povzetku na kratko opišite vsebinske rezultate dela. Sem ne sodi razlaga organizacije dela --
%  v katerem poglavju/razdelku je kaj, pač pa le opis vsebine.
\povzetek{povzetek HERE}

%  Prevod slovenskega povzetka v angleščino.
\abstract{ABSTRACT HERE}

% navedite vsaj eno klasifikacijsko oznako --
% dostopne so na www.ams.org/mathscinet/msc/msc2010.html
\klasifikacija{43-00}
\kljucnebesede{grupa topologija} % navedite nekaj ključnih pojmov, ki nastopajo v delu
\keywords{group topology} % angleški prevod ključnih besed

\zapisiMetaPodatke  % poskrbi za metapodatke in veljaven PDF/A-1b standard

% aktivirajte pakete, ki jih potrebujete
% \usepackage{tikz}
\usepackage[slovene]{babel}
\usepackage[utf8]{inputenc}
\usepackage[T1]{fontenc}
\usepackage{lmodern}
\usepackage{amsmath}
\usepackage{amssymb}
\usepackage{amsthm}
\usepackage{amsfonts}
\usepackage{mathtools}
\usepackage{enumitem}

% za številske množice uporabite naslednje simbole
\newcommand{\R}{\mathbb R}
\newcommand{\N}{\mathbb N}
\newcommand{\Z}{\mathbb Z}
\newcommand{\C}{\mathbb C}
\newcommand{\Q}{\mathbb Q}

\newcommand{\Ucurl}{\mathcal{U}}
\newcommand{\closure}[1]{\overline{#1}}
\newcommand{\setcomp}[1]{{#1}^{\mathsf{c}}}
% matematične operatorje deklarirajte kot take, da jih bo Latex pravilno stavil
% \DeclareMathOperator{\conv}{conv}

\DeclareMathOperator{\interior}{int}
\DeclareMathOperator{\pr}{pr}

% vstavite svoje definicije ...
%  \newcommand{}{}

\begin{document}

\section{Uvod}

\section{Preliminarna poglavja}
V tem poglavju bomo ponovili že znane pojme iz algebre in splošne topologije, ki nam bodo kasneje prišli prav. Dogovorili se bomo tudi o zapisu operacij na množicah.

\subsection{Operacije na množicah}\label{sec:opnamnozicah}
Vse operacije na množicah, če ne bo drugače zaznamovano, delujejo na elementih. Tako je na primer produkt množic $U$ in $V$ enak \[U \cdot V = \lbrace u \cdot v ; u \in U, v \in V \rbrace, \] inverz množice $U$ pa je \[ U^{-1} = \lbrace u^{-1} ; u \in G \rbrace. \] Tukaj se v obeh primerih predpostavlja, da so množice vložene v neki grupi, kjer so operacije na elementih smiselno definirane. Grupno strukturo bomo bolj podrobno opisali v naslednjem podrazdelku.

Pomembnejša izjema temu pravilu so operacije na množicah v smislu relacij. Predpostavimo torej, da imamo množico $X$ in nas zanimajo podmnožice kartezičnega produkta $X \times X$. Inverz take množice $U$ je definiran kot \[ U^{-1} = \lbrace (y, x) ; (x, y) \in U \rbrace, \]
analogna operacija množenju pa je kompozitum množic \[ V \circ U = \lbrace (x, z) ; \text{ obstaja tak element } y \in X, \text{ da je } (x, y) \in V \text{ in } (y, z) \in U \rbrace. \]
Takšni notaciji operacij bosta vedno posebej označeni.

\subsection{Teorija grup}
V tem podpoglavju bomo ponovili nekaj osnovnih algebraičnih pojmov, predvsem iz teorije grup.

Neprazna množica $G$ z binarno operacijo $*$ je \emph{grupa}, če:
\begin{enumerate}
\item je množica $G$ zaprta za operacijo $*$,
\item je operacija $*$ asociativna v množici $G$,
\item v $G$ obstaja tak element $e$ (imenujemo ga \emph{enota}), da za vsak element $x$ množice $G$ velja \[ x*e = e*x = x, \]
\item za vsak element $x$ množice $G$ obstaja element $y$ tudi iz množice $G$, da velja \[ x*y = y*x = e. \]
\end{enumerate}
Oznaka za grupo je ($G$, $*$) ali samo $G$, če je operacija znana ali drugače očitna. Od tukaj naprej bo zapis operacije vedno multiplikativen, razen če bo drugače povdarjeno. To pomeni, da bo grupna operacija označena s $\cdot$ ali pa bo izpuščena.

Med grupami lahko definiramo nekaj tipov preslikav. Našteli bomo dva, ki ju bomo v nadaljevanju najbolj potrebovali.
Preslikava $f\colon G \to \widetilde{G}$ je \emph{homomorfizem} grup, če za vsaka dva elementa $a, b \in G$ velja $f(a\cdot b) = f(a)\cdot f(b)$.
Preslikava je \emph{izomorfizem} grup, če je bijektivna in homomorfizem grup.

V nadaljevanju si bomo ogledali nekaj podstruktur grupe.
Podmnožica $H$ grupe $G$ je \emph{podgrupa}, če je tudi sama grupa za isto operacijo.
Množicama $aH = \lbrace ah ; h \in H \rbrace$ in $Ha = \lbrace ha ; h \in H \rbrace$ zaporedoma pravimo \emph{levi} in \emph{desni odsek} grupe $G$ elementa $a \in G$ po podgrupi $H$.

Podgrupi $H$ grupe $G$ rečemo podgrupa \emph{edinka}, če za vsak element $a \in G$ velja \[aHa^{-1} \subseteq H.\]
Množici $G/H = \lbrace aH ; a \in G \rbrace$ rečemo \emph{kvocientna množica} grupe $G$ po podgrupi $H$.
\emph{Naravna preslikava} na kvocientno množico $G/H$ je preslikava $\varphi: G \to G/H$, $a \mapsto aH$.

Če je $N$ podgrupa edinka grupe $G$, je kvocientna množica $G/N$ grupa za operacijo $*$, kjer je $aH\cdot bH = (a\cdot b)H$, naravna preslikava $\varphi$ pa je homomorfizem grup in mu rečemo \emph{naravni homomorfizem}.

\subsection{Topološki prostori}
V tem podpoglavju bomo ponovili nekaj pojmov iz splošne topologije, ki jih bomo kasneje podrobneje obravnavali na topoloških grupah.

\emph{Topologija} na neprazni množici $X$ je neprazna družina podmnožic $\tau \subseteq 2^X$ z lastnostmi:
\begin{enumerate}
\item $X \in \tau$, $\emptyset \in \tau$,
\item za poljubni dve množici $U,V \in \tau$ je tudi presek $U \cap V \in \tau$,
\item za poljubno poddružino $\lbrace U_{\lambda} \rbrace_{\lambda \in \Lambda} \subseteq \tau$ je tudi unija $\bigcup\limits_{\lambda \in \Lambda}^{} U_{\lambda} \in \tau$.
\end{enumerate}
Množici $X$, opremljeni s topologijo $\tau$, rečemo \emph{topološki} prostor, ki ga označimo z $(X, \tau)$. Množice v družini $\tau$ imenujemo \emph{odprte} množice v topološkem prostoru $X$, \emph{zaprte} množice pa definiramo kot komplemente odprtih množic glede na množico $X$.

Družina $B$ je \emph{baza} za topologijo $\tau$, če je vsaka množica iz topologije $\tau$ unija nekaterih množic iz $B$, družina $P$ pa je \emph{podbaza} za topologijo $\tau$, če je družina vseh končnih presekov množic iz $P$ neka baza za topologijo $\tau$.

Množica $U \subseteq X$ je \emph{okolica za točko} $x \in X$, če obstaja taka odprta množica $V \in \tau$, da velja $V \subseteq U$ in $x \in V$. Enako lahko definiramo okolico za množico.
Množica $U \subseteq X$ je \emph{okolica} množice $A \subseteq X$, če obstaja taka odprta množica $V \in \tau$, da velja $A \subseteq V \subseteq U$..
Družina okolic $\Ucurl_x$ točke $x \in X$ se imenuje \emph{baza okolic} za $x$, če za poljubno okolico $V$ točke $x$ velja, da obstaja tak $U \in \Ucurl_x$, da je $U \subseteq V$.

Točka $a \in A$ je \emph{notranja točka} množice $A$, če je $A$ okolica za točko $a$.
\emph{Notranjost} množice $A$ je množica vseh njenih notranjih točk. Notranjost množice označimo z $\interior(A)$. Očitno velja $\interior(A) \subseteq A$ in tudi $\interior(A) = A \iff A \in \tau$.
\emph{Zaprtje} množice $A$ je najmanjša zaprta množica v $X$, ki vsebuje $A$. Zaprtje množice označimo z $\closure{A}$. Očitno velja $A \subseteq \closure{A}$ in tudi $\closure{A} = A \iff A$ je zaprta množica.

S pomočjo odprtih in zaprtih množic topološkega prostora $X$ lahko sedaj definiramo zveznost in odprtost preslikave med dvema topološkima prostoroma ter pojem homeomorfizma.

Tako kot med grupami lahko tudi med topološkimi prostori definiramo nekaj tipov preslikav. Ogledali si bomo nekaj za nas najpomembnejših tipov.
Naj bo $f\colon (X, \tau_1) \to (Y, \tau_2)$ preslikava med topološkima prostoroma.
Preslikava $f$ je \emph{zvezna}, kadar je praslika vsake odprte množice v topološkem prostoru $(Y, \tau_2)$  preslikave $f$ odprta tudi v topološkem prostoru $(X, \tau_1)$.
Preslikava $f$ je \emph{odprta}, kadar je slika vsake odprte množice v topološkem prostoru $(X, \tau_1)$ preslikave $f$ odprta tudi v topološkem prostoru $(Y, \tau_2)$.
Preslikava $f$ je \emph{homeomorfizem}, če je bijektivna, zvezna in ima zvezen inverz.

Osnovna podstruktura topološkega prostora je topološki podprostor.
Najprej vzemimo topološki prostor $X$ s topologijo $\tau$ in množico $A \subseteq X$. \emph{Inducirana} ali \emph{relativna topologija} na množici $A$, inducirana s $\tau$, je družina množic $\lbrace A \cap U ; U \in \tau \rbrace$. Prostoru $A$ rečemo \emph{topološki podprostor} prostora $X$.

Oglejmo si še produkt topoloških prostorov.
Naj bosta $X$ in $Y$ topološka prostora s topologijama $\tau_1$ in $\tau_2$. \emph{Produktna topologija} na kartezičnemu produktu $X \times Y$ je topologija, generirana z bazo $\lbrace U \times V ; U \in \tau_1, V \in \tau_2 \rbrace$. \emph{Produkt} topoloških prostorov $X$ in $Y$ je topološki prostor $X \times Y$, opremljen s produktno topologijo. Produkt topoloških prostorov je opremljen še z dvema projekcijskima preslikavama $\pr_1\colon X \times Y \to X$, $\pr_1(x, y) = x$ in $\pr_2\colon X \times Y \to Y$, $\pr_2(x, y) = y$. Obe projekciji sta zvezni in odprti preslikavi glede na primerni topologiji.

V nadaljevanju si bomo ogledali nekaj pojmov povezanih s kompaktnostjo topoloških prostorov, ki si jih bomo kasneje ogledali v kontekstu topoloških grup, definirali pa bomo tudi nove.
Družini $\mathcal{A}$ množic rečemo \emph{pokritje} topološkega prostora $X$, če je $X \subseteq \bigcup \mathcal{A}$,družini $\mathcal{B} \subseteq \mathcal{A}$ pa rečemo \emph{podpokritje} topološkega prostora $X$, če je $\mathcal{B}$ tudi sama pokritje za $X$.
Topološki prostor je \emph{kompakten}, če vsako njegovo odprto pokritje, tj. pokritje z odprtimi množicami, vsebuje kakšno končno podpokritje.
Topološki prostor je \emph{lokalno kompakten}, če ima vsaka točka $x \in X$ kakšno kompaktno okolico.

Ponovimo še do sedaj obravnavane separacijske aksiome.
Topološki prostor $(X, \tau)$ zadošča separacijskemu aksiomu
\begin{enumerate}
\item $T_0$, če za poljubni različni točki $a, b \in X$ obstaja okolica $V$ za eno od točk $a, b$, ki ne vsebuje druge od točk $a, b$;
\item $T_1$, če za poljubno točko $a \in X$ in točko $b \in X\setminus\lbrace a \rbrace$ obstaja okolica $V$ točke $a$, ki ne vsebuje točke $b$;
\item $T_2$, če za poljubni različni točki $a, b \in X$ obstajata disjunktni okolici za točki $a$ in $b$;
\item $T_3$, če za poljubno zaprto množico $A \subseteq X$ in točko $b \in X\backslash A$ obstajata disjunktni okolici za množico $A$ in točko $b$;
\item $T_4$, če za poljubni disjunktni zaprti množici $A, B \subseteq X$ obstajata disjunktni okolici za množici $A$ in $B$.
\end{enumerate}

Iz definicije je razvidno, da $T_2 \implies T_1 \implies T_0$.
Topološkemu prostoru, ki zadošča separacijskemu aksiomu $T_2$, pravimo \emph{Hausdorffov} topološki prostor.
Topološku prostoru, ki zadošča $T_1$ in $T_3$ pravimo \emph{regularen} topološki prostor.
Topološku prostoru, ki zadošča $T_1$ in $T_4$, pravimo \emph{normalen} topološki prostor.

Separacijske aksiome v povezavi z metrizabilnostjo na topoloških grupah si bomo podrobneje ogledali v kasnejših poglavjih.

\section{Kaj je topološka grupa}

Iz zgornje definicije je razvidno, da nam grupna struktura na množici porodi dve strukturni preslikavi:
\begin{itemize}
\item \emph{množenje} $\mu\colon G \times G \to G$, $(x, y) \mapsto xy$,
\item \emph{invertiranje} $\iota\colon G \to G$, $x \mapsto x^{-1}$.
\end{itemize}

Končno lahko strukturi združimo in povežemo ter definiramo pojem topološke grupe.
\begin{definicija}\label{def:topgrupa}
\emph{Topološka grupa} je grupa $G$ opremljena s takšno topologijo $\tau$, da sta za $\tau$ strukturni operaciji množenja in invertiranja zvezni. 
\end{definicija}

Potrebujemo le še tip preslikave med topološkimi grupami, ki bo ohranjal tako algebraično kot topološko strukturo.
\begin{definicija}\label{def:topizo}
Preslikava med dvema topološkima grupama je \emph{topološki izomorfizem}, če je izomorfizem in homeomorfizem.
\end{definicija}


\begin{trditev}\label{trd:trans}
Naj bo $G$ topološka grupa in $a \in G$.
\begin{enumerate}
\item Leva translacija $l_a\colon x \mapsto ax$ in desna translacija $r_a\colon x \mapsto xa$ za $a$ sta homeomorfizma iz $G$ v $G$
\item Invertiranje $\iota\colon x \mapsto x^{-1}$ je homeomorfizem iz $G$ v $G$.
\item Konjugiranje $\gamma\colon x \mapsto axa^{-1}$ je homeomorfizem iz $G$ v $G$.
\end{enumerate}
\end{trditev}

\begin{dokaz}
Vemo že, da so leva translacija, desna translacija, invertiranje in konjugiranje avtomorfizmi grupe $G$, torej bijektivne preslikave. Dokazujemo še zveznost preslikave in njenega inverza.

Naj bo $\text{c}_a\colon x \mapsto a$ konstantna preslikava. Vemo, da je konstantna preslikava zvezna.

Levo translacijo lahko zapišemo kot kompozitum zveznih preslikav \[l_a(x) = \mu(\text{c}_a(x), x) = ax.\]
Kot kompozitum zveznih preslikav je leva translacija zvezna. Inverzna preslikava leve translacije za element $a$ je leva translacija za element $a^{-1}$, ki je tudi zvezna. Leva translacija je zato homeomorfizem.

Desno translacijo lahko na analogen način zapišemo kot kompozitum zveznih preslikav \[r_a(x) = \mu(x, \text{c}_a(x)) = xa.\]
Kot kompozitum zveznih preslikav je desna translacija zvezna in njena inverzna preslikava je desna translacija za element $a^{-1}$, torej tudi zvezna. Desna translacija je zato homeomorfizem.

Invertiranje je homeomorfizem, ker je zvezno po definiciji topološke grupe in samo sebi inverz.

Konjugiranje lahko zapišemo kot kompozitum zveznih preslikav \[\gamma = r_{a^{-1}} \circ l_a.\]
Kot kompozitum zveznih preslikav je konjugiranje zvezno in njegov inverz je konjugiranje za element $a^{-1}$, torej tudi zvezen. Konjugiranje je zato homeomorfizem.
\end{dokaz}

\begin{trditev}\label{trd:prododp}
Naj bosta $A$ in $B$ podmnožici topološke grupe $G$. Če je ena od njiju odprta, sta odprti tudi množici $AB$ in $BA$.
\end{trditev}

\begin{dokaz}
Brez škode za splošnost privzemimo, da je $A$ odprta. Velja $AB = \lbrace Ab ; b \in B \rbrace$. Ker je $A$ odprta, so po trditvi \ref{trd:trans} odprte tudi vse množice $Ab$, saj je desna translacija homeomorfizem. Množica $AB$ je odprta kot unija odprtih množic.

Dokaz je popolnoma simetričen za množico $BA$.
\end{dokaz}

\begin{trditev}\label{trd:okolice}
Za topološko grupo $G$ in bazo $\Ucurl$ odprtih okolic enote $e$ veljajo naslednje trditve:
\begin{enumerate}
\item za vsako množico $U \in \Ucurl$ obstaja taka množica $V \in \Ucurl$, da velja $V^{2} \subset U$;
\item za vsako množico $U \in \Ucurl$ obstaja taka množica $V \in \Ucurl$, da velja $V^{-1} \subset U$;
\item za vsako množico $U \in \Ucurl$ in vsak element $x \in U$ obstaja taka množica $V \in \Ucurl$, da velja $xV \subset U$;
\item za vsako množico $U \in \Ucurl$ in vsak element $x \in G$ obstaja taka množica $V \in \Ucurl$, da velja $xVx^{-1} \subset U$.
\end{enumerate}
\end{trditev}

\begin{dokaz}
Naj bo $U \in \Ucurl$  odprta okolica enote $e$. Ker je množenje zvezno, obstaja v produktni topologiji na $G \times G$ odprta okolica $W = V_1 \times V_2$ enote $(e, e)$, za katero velja $V_1V_2 \subset U$. Po definiciji produktne topologije sta $V_1$ in $V_2$ odprti okolici enote $e$ v $G$. Definiramo $V' = V_1 \cap V_2$ okolico za $e$. Po definiciji baze okolic obstaja $V \in \Ucurl$, da velja $V \subseteq V'$. Ker je $V \subseteq V' \subseteq V_1$ in $V \subseteq V' \subseteq V_2$, velja \[V^2 \subseteq V'^2 \subseteq V_1V_2 \subset U.\] To dokaže prvo trditev.

Vzemimo poljuben $V \in \Ucurl$, $V \subset U$ (obstaja po definiciji baze okolic). Ker je invertiranje zvezno, obstaja v $G$ odprta okolica $W$ enote $e$, za katero velja $W \subset V$. Po trditvi \ref{trd:trans} je invertiranje homeomorfizem, zato je $W = V^{-1}$. Velja \[V^{-1} \subset V \subset U.\] To dokaže drugo trditev.

Vzemimo poljubno točko $x \in U$. Ker je $U$ odprta množica, obstaja okolica $W \subset U$ za točko $x$. Naj bo $V' = x^{-1}W$. Ker je po trditvi \ref{trd:trans} leva translacija homeomorfizem, je $V'$ odprta okolica enote $e$. Vzemimo $V \in \Ucurl$, $V \subset V'$ (obstaja po definiciji baze okolic). Velja \[xV \subset xV' = xx^{-1}W = W \subset U.\]
To dokaže tretjo trditev.

Naj bo $x \in U$ poljubna točka. Ker je po trditvi \ref{trd:trans} konjugiranje homeomorfizem, obstaja odprta okolica enote $e$ oblike $xVx^{-1} \subseteq U$.
To dokaže še četrto trditev.
\end{dokaz}

\begin{trditev}\label{trd:okolice2}
Naj bo $G$ grupa (ne topološka) in $\Ucurl$ družina podmnožic množice $G$, za katero veljajo vse štiri lastnosti iz trditve \ref{trd:okolice}. Naj bodo poljubni končni preseki množic iz $\Ucurl$ neprazni. Tedaj je družina $\lbrace xU \rbrace$, kjer $U \in \Ucurl$ in $x \in G$ odprta podbaza za neko topologijo na $G$. S to topologijo je $G$ topološka grupa. Družina $\lbrace Ux \rbrace$ je podbaza za isto topologijo.

Če velja še, da za vsaki množici $U,V \in \Ucurl$ obstaja množica $W \in \Ucurl$, da velja $W \subset U \cap V$, potem sta družini $\lbrace xU \rbrace$ in $\lbrace Ux \rbrace$ tudi bazi za to topologijo.
\end{trditev}

\begin{definicija}\label{def:sim}
Množici, za katero velja $U = U^{-1}$, rečemo \emph{simetrična} množica.
\end{definicija}

\begin{trditev}\label{trd:sim}
Vsaka topološka grupa ima bazo $\Ucurl$ odprtih in simetričnih okolic enote.
\end{trditev}

\begin{dokaz}
Naj bo $\mathcal{V}$ neka baza odprtih okolic enote. Za vsako okolico $V \in \mathcal{V}$ konstruiramo množico $U = V \cup V^{-1}$. Kot presek dveh odprtih množic je $U$ odprta. Ker je $e \in V$ in $e \in V^{-1}$, je $U$ odprta okolica enote, ki je po konstrukciji simetrična. Ker po definiciji preseka velja še $U \subseteq V$, je družina $\Ucurl = \lbrace V \cap V^{-1}; V \in \mathcal{V} \rbrace$ res baza odprtih in simetričnih okolic enote $e$.
\end{dokaz}

\begin{posledica}\label{pos:sim}
Za vsako okolico $U$ enote $e$ topološke grupe $G$ obstaja taka okolica $V$ enote $e$, da velja $\closure{V} \subset U$.
\end{posledica}

\begin{dokaz}
Naj bo $U$ neka okolica enote in naj bo $\mathcal{V}$ baza odprtih in simetričnih okolic enote (obstaja po trditvi \ref{trd:sim}). Naj bo $V \in \mathcal{V}$ takšna okolica, da velja $V^2 \subset U$. Takšna okolica obstaja po trditvi \ref{trd:okolice}. Vzemimo $x \in \closure{V}$. Velja $(xV) \cap V \neq \emptyset$.
Res: ker je $V$ okolica enote, je $xV$ okolica elementa $x$. Če je $x \in V$, zgornji presek ni prazen, ker je $V$ odprta množica, če pa je $x$ iz roba množice $V$, vsaka njegova okolica seka množico $V$.

Obstajata torej $v_1, v_2 \in V$, da velja $x v_1 = v_2$. Sledi \[x = v_2 v_1^{-1} \in VV^{-1} = V^2 \subset U.\]
Torej res velja $\closure{V} \subset U$.
\end{dokaz}

\begin{trditev}\label{trd:t0haus}
Za topološko grupo $G$ velja, da zadošča separacijskemu aksiomu $T_0$ natanko tedaj, kadar je Hausdorffova.
\end{trditev}

\begin{dokaz}
Dokazujemo implikacijo $T_0 \implies T_2$.

Dokažimo najprej, da je $G$ Hausdorffova natanko tedaj, ko je $\lbrace e \rbrace$ zaprta množica.

Če je $G$ Hausdorffova, zadošča tudi separacijskemu aksiomu $T_1$, kar pomeni, da so vse enoelementne množice zaprte, torej tudi $\lbrace e \rbrace$.

Privzemimo, da je $\lbrace e \rbrace$ zaprta množica. Oglejmo si preslikavo $f\colon G \times G \to G$, $(x, y) \mapsto xy^{-1}$. Preslikava $f$ je zvezna kot kompozitum množenja in invertiranja, ki sta zvezni preslikavi po definiciji topološke grupe. Zato je $f^{-1}(\lbrace e \rbrace) = \lbrace (x, x) ; x \in G \rbrace$ zaprta množica v $G \times G$, to pa je ekvivalentno temu, da je $G$ Hausdorffova.

Za dokaz implikacije je dovolj pokazati, da je v vsaki topološki grupi, ki zadošča separacijskemu aksiomu $T_0$, množica $\lbrace e \rbrace$ zaprta. Pokažimo, da je $G\setminus\lbrace e \rbrace$ odprta.

Vzemimo točko $x \in G\setminus\lbrace e \rbrace$. Ker $G$ zadošča separacijskemu aksiomu $T_0$, obstaja bodisi okolica za točko $x$, ki ne vsebuje enote $e$, bodisi okolica $V$ za enoto $e$, ki ne vsebuje točke $x$. V prvem primeru to pomeni, da je $G\setminus\lbrace e \rbrace$ odprta množica in je implikacija dokazana. Zato naj bo $V$ okolica za enoto $e$, ki ne vsebuje točke $x$. Množica $x^{-1}V$ je potem okolica za točko $x^{-1}$, ki ne vsebuje enote $e$, zato je $\iota(x^{-1}V)$ okolica za točko $x$, ki ne vsebuje enote $e$.

Velja \[G \text{ zadošča } T_0 \implies \lbrace e \rbrace \text{ je zaprta } \implies G \text{ zadošča } T_2. \]
\end{dokaz}

\begin{izrek}\label{izr:t3}
	Vsaka topološka grupa $G$, ki zadošča separacijskemu aksiomu $T_0$, je regularen topološki prostor.
\end{izrek}

\begin{dokaz}
Po trditvi \ref{trd:t0haus} je $G$ Hausdorffova in zato zadošča separacijskemu aksiomu $T_1$.

Po posledici \ref{pos:sim} za vsako okolico $U$ enote $e$ obstaja okolica $V$ enote $e$, da je $\closure{V} \subset U$. Ker je po trditvi \ref{trd:trans} leva translacija homeomorfizem, to velja v vsaki točki, saj $\closure{aV} \subset aU$, kar pa je ekvivalentno separacijskemu aksiomu $T_3$. Topološka grupa $G$ je res regularna.
\end{dokaz}


\subsection{Primeri topoloških grup}

\begin{primer}
Vsaka grupa $G$ je za diskretno topologijo $\tau_d = 2^G$ in trivialno topologijo $\tau_t = \lbrace \emptyset, G \rbrace$ topološka grupa, saj je glede na njiju zvezna vsaka preslikava na $G$ ali $G \times G$.
\end{primer}

\begin{primer}
Realna števila za operacijo $+$ so topološka grupa z evklidsko topologijo.

V tem primeru sta strukturni operaciji $\mu(x, y) = x + y$ in $\iota(x) = -x$. Preverimo, da sta res zvezni.

Naj bosta preslikavi $pr_x(x, y) = x$ in $pr_y(x, y) = y$ projekciji. Po definiciji produktne topologije sta to zvezni preslikavi. Ker je vsota zveznih preslikav zvezna preslikava, je strukturna preslikava seštevanja $\mu = pr_x + pr_y$ zvezna.

Zveznost invertiranja je dovolj preveriti na baznih množicah evklidske topologije. Vzemimo interval $(-b, -a)$. Tudi $\iota^{-1}((-b, -a)) = \iota((-b, -a)) = (a, b)$ je bazna množica in zato odprta, torej je invertiranje zvezno.
\end{primer}

\begin{primer}
Realna števila za operacijo $+$ niso topološka grupa s topologijo $\tau = \lbrace (a, \infty) ; a \in \R \rbrace$.

Preverimo lahko zveznost strukturnih strukturnih preslikav in se prepričamo, da invertiranje ni zvezno, lahko pa se prepričamo tudi drugače.

Po trditvi \ref{trd:t0haus} za vsako topološko grupo velja, da zadošča separacijskemu aksiomu $T_0$ natanko tedaj, ko je Hausdorffova. Topološki prostor $(\R, +, \tau)$ zadošča separacijskemu aksiomu $T_0$, saj je za vsaki dve točki $a < b$ množica $(\frac{a+b}{2}, \infty)$ okolica za točko $b$, ki ne vsebuje točke $a$. Hkrati pa je očitno, da ta prostor ni Hausdorffov, saj se vsaki dve odprti množici sekata. Topološki prostor $(\R, +, \tau)$ torej ne more biti topološka grupa.
\end{primer}

\begin{primer}
Enotska krožnica v kompleksni ravnini $S = \lbrace z \in \C ; |z| = 1 \rbrace$ s podedovanim množenjem je topološka grupa za relativno topologijo kot podprostor $\R^2$ (spomnimo se, da je $\C \cong \R^2$).
\end{primer}


\section{Kvocienti topoloških grup}

\begin{trditev}\label{trd:toppodgrupa}
Naj bo $G$ topološka grupa in $H$ njena podgrupa. Če $H$ opremimo z relativno topologijo, potem je tudi $H$ topološka grupa.
\end{trditev}

\begin{dokaz}
Preslikavi $\mu|_H$ in $\iota|_H$ sta zvezni v relativni topologiji na $H$ kot zožitvi zveznih preslikav na topološki podprostor $H$. Ker je $H$ za zoženo množenje grupa (po definiciji podgrupe), je $H$ topološka grupa.
\end{dokaz}

\begin{trditev}\label{trd:zaprtost}
Za $A$ in $B$ podmnožici topološke grupe $G$ veljajo naslednje trditve:
\begin{enumerate}
\item $\closure{A}\ \closure{B} \subset \closure{A B}$,
\item $(\closure{A})^{-1} = \closure{A^{-1}}$,
\item $x \closure{A} y = \closure{x A y}$ za vsaka dva elementa $x, y \in G$.

\item Če $G$ ustreza separacijskemu aksiomu $T_0$ in za vsaka dva elementa $a \in A$ in $b \in B$ velja enakost $ab = ba$, potem velja enakost $ab = ba$ tudi za vsaka dva elementa $a \in \closure{A}$ in $b \in \closure{B}$.
\end{enumerate}
\end{trditev}

\begin{dokaz}
Naj bosta $A$ in $B$ podmnožici topološke grupe $G$.

Za dokaz prve trditve vzemimo točki $x \in \closure{A}$ in $y \in \closure{B}$ ter neko okolico $U$ enote $e$. Ker je množenje zvezno, obstaja enote $V_1$ in $V_2$, da je $(xV_1)(yV_2) \subset xyU$. Definiramo okolico enote $V = V_1 \cap V_2$. Velja $(xV)(yV) \subset xyU$. Ker je $xV$ okolica za $x$ in $yV$ okolica za $y$ ter vsaka okolica točke iz zaprtja množice seka množico samo, obstajata $a \in A$ in $B \in B$, da je $a \in xV$ in $b \in yV$. Velja torej $ab \in (AB) \cap (xyU)$ in, ker je $xyU$ okolica za $xy$, tudi $xy \in \closure{AB}$. Inkluzija v prvi trditvi je s tem dokazana.

Enakost v drugi trditvi sledi iz tega, da je invertiranje homeomorfizem (trditev \ref{trd:trans}). Vzemimo množico $A \subset G$. Ker je invertiranje zvezno, je $\iota(\closure{A}) \subseteq \closure{\iota(A)}$. Ker je samo sebi inverz, velja tudi obratno $\closure{\iota(A)} \subseteq \iota(\closure{A})$. Torej $\closure{A}^{-1} = \closure{A^{-1}}$.

Enakost v tretji trditvi sledi iz tega, da sta leva in desna translacija homeomorfizma (trditev \ref{trd:trans}). Naj bosta $x, y \in G$. Tedaj je tudi $f = r_y \circ l_x$ homeomorfizem. Ker je $\closure{A}$ zaprta, je $x\closure{A}y$ najmanjša zaprta množica, ki vsebuje množico $xAy$. Torej res velja $x\closure{A}y = \closure{xAy}$.

Za dokaz četrte trditve privzemimo še, da $G$ zadošča separacijskemu aksiomu $T_0$ in velja $ab = ba$ za vsaka dva elementa $a \in A$ in $b \in B$. Preslikava $(a,b) \mapsto aba^{-1}b^{-1}$ je zvezna, saj je kompozitum množenj in invertiranj. Ker je po izreku \ref{izr:t3} množica $\lbrace e \rbrace$ zaprta, je zaprta tudi množica $H = \lbrace (a, b) \in G \times G; aba^{-1}b^{-1} = e \rbrace$, saj je njena $f$-praslika. Po predpostavki velja $A \times B \subseteq H$ in po definiciji produktne topologije velja $\closure{A \times B} = \closure{A} \times \closure{B}$. Sledi, da je $\closure{A} \times \closure{B} \subseteq H$, torej je $ab = ba$ za vsaka dva elementa $a \in \closure{A}$ in $b \in \closure{B}$.
\end{dokaz}

\begin{trditev}\label{trd:odpzap}
Naj bo $H$ podgrupa topološke grupe $G$. $H$ je odprta natanko tedaj, ko ima neprazno notranjost. Vsaka odprta podgrupa $H$ topološke grupa $G$ je tudi zaprta.
\end{trditev}

\begin{dokaz}
Denimo, da obstaja element $x$ v notranjosti $H$. Potem obstaja okolica $U$ enote $e$, da je $xU \subset H$, saj je notranjost množice $H$ odprta množica, ki je vsebovana v $H$. Vzemimo element $y \in H$. Velja \[yU = yx^{-1}xU \subset yx^{-1}H = H,\] saj $H$ kot podgrupa vsebuje tudi $x^{-1}$. Vsak element v $H$ ima torej odprto okolico, ki je vsebovana v $H$, kar pomeni, da je $H$ odprta množica.

Obratno, če je $H$ odprta, vsaka njena točka leži tudi v njeni notranjosti, torej ima neprazno notranjost.

Privzemimo, da je $H$ odprta podgrupa grupe $G$. Ker je $H$ zaprta za množenje, je $\setcomp{H} = \bigcup \lbrace xH ; x \notin H \rbrace$. Vsaka množica $xH$ je odprta, ker je $H$ odprta in je po trditvi \ref{trd:trans} leva translacija homeomorfizem. Potem je tudi $\setcomp{H}$ odprta množica, torej je $H$ zaprta množica.
\end{dokaz}

\begin{trditev}\label{trd:podgrupaunija}
Naj bo $U$ simetrična okolica enote $e$ v topološki grupi $G$. Potem je $L = \bigcup_{n=1}^{\infty} U^n$ odprta in zaprta podgrupa topološke grupe $G$.
\end{trditev}

\begin{dokaz}
Vzemimo $x \in U^k$ in $y \in U^l$. Velja $xy \in U^kU^l \subseteq U^{k+l}$, torej je $L$ zaprta za množenje. Ker je $U$ simetrična, velja tudi $x^{-1} \in (U^{-1})^k = U^k$, torej je $L$ zaprta za invertiranje. Sledi, da je $L$ podgrupa topološke grupe $G$. V njeni notranjosti je zagotovo enota $e$, saj je $U$ okolica za $e$. Po trditvi \ref{trd:odpzap} je $L$ odprta in zaprta podgrupa topološke grupe $G$.
\end{dokaz}


\begin{izrek}\label{izr:topkvocienta}
Naj bo $G$ topološka grupa, $H$ njena podgrupa in $\varphi\colon G \to G/H$ naravna preslikava. Definiramo $\theta(G/H) = \lbrace U ; \varphi^{-1}(U)$ odprta v $G \rbrace$.
Veljajo naslednje trditve:
\begin{enumerate}
\item družina $\theta(G/H)$ je topologija na kvocientu $G/H$,
\item glede na topologijo $\theta(G/H)$ je $\varphi$ zvezna preslikava,
\item družina $\theta(G/H)$ je najmočnejša topologija na kvocientu $G/H$, glede na katero je $\varphi$ zvezna preslikava,
\item $\varphi: G \to G/H$ je odprta preslikava.
\end{enumerate}
\end{izrek}

\begin{dokaz}
Naj bo $\theta(G/H) = \lbrace uH ; u \in U_\lambda \rbrace_{\lambda \in \Lambda}$ družina odprtih množic v $G/H$, kjer so $U_\lambda$ odprte množice v $G$. Potem je njihova unija $\bigcup_{\lambda \in \Lambda} \lbrace uH ; u \in U_\lambda \rbrace = \lbrace uH ; u \in \bigcup_{\lambda \in \Lambda}U_\lambda \rbrace$ prav tako odprta v $G/H$, saj je $\bigcup_{\lambda \in \Lambda}U_\lambda$ odprta v $G$. Presek dveh takih množic $\lbrace uH ; u \in U_\lambda \rbrace \cap \lbrace uH ; u \in U_\mu \rbrace = \lbrace uH ; u \in U_\lambda \cap U_\mu \rbrace$ je tudi odprt, saj je presek $U_\lambda \cap U_\mu$ odprt v $G$. Velja tudi $\emptyset \in \theta(G/H)$, če vzamemo $U_\lambda = \emptyset$, ki je odprta v $G$. Če vzamemo $U_\lambda = G$, dobimo tudi $G/H \in \theta(G/H)$. Preverili smo, da je $\theta(G/H)$ res topologija na kvocientu $G/H$.

Preslikava $\varphi$ je zvezna po definiciji topologije $\theta(G/H)$ in topologija $\theta(G/H)$ je res najmočnejpa topologija na kvocientu $G/H$, glede na katero je $\varphi$ zvezna, po konstrukciji $\theta(G/H)$.

Za dokaz četrte trditve vzemimo odprto množico $U \in G$. Po trditvi \ref{trd:prododp} je množica $UH$ odprta v $G$, torej je $\varphi(U) = \lbrace uH ; u \in U \rbrace$ odprta v $G/H$.
\end{dokaz}

Topologiji $\theta(G/H)$ pravimo \emph{kvocientna topologija}, kvocientu $G/H$ pa \emph{kvocientni prostor}.


\begin{trditev}\label{trd:okolicevkvoc}
Naj bo $G$ topološka grupa, $H$ njena podgrupa in $U, V$ tako okolici enote $e$ v $G$, da velja $V^{-1}V \subset U$. Naj bo $\varphi: G \to G/H$ naravna preslikava. Potem velja $\closure{\varphi(V)} \subset \varphi(U)$.
\end{trditev}

\begin{dokaz}
Vzemimo odsek $xH \in \closure{\varphi(V)}$. Ker je $V$ okolica enote, je množica $\lbrace vxH ; v \in V \rbrace$ okolica odseka $xH$ in ima zato s $\varphi(V)$ neprazen presek. Po definiciji naravne preslikave obstajata točki $v_1, v_2 \in V$, da je $v_1xH = v_2H$. Velja \[xH = v_1^{-1}v_2H \in \lbrace wH ; w \in V^{-1}V \rbrace \subset \lbrace uH ; u \in U \rbrace = \varphi(U). \]
Torej je res $\closure{\varphi(V)} \subset \varphi(U)$.
\end{dokaz}

\begin{izrek}\label{izr:kvocreg}
Za topološko grupo $G$ in njeno podgrupo $H$ veljajo naslednje trditve:
\begin{enumerate}
\item kvocientni prostor $G/H$ je diskreten natanko tedaj, ko je $H$ odprta v $G$,
\item če je $H$ zaprta v $G$, potem je kvocient $G/H$ regularen topološki prostor,
\item če kvocientni prostor $G/H$ zadošča separacijskemu aksiomu $T_0$, potem je $H$ zaprta v $G$ in velja, da je kvocient $G/H$ regularen topološki prostor.
\end{enumerate}
\end{izrek}

\begin{dokaz}
Za dokaz prve trditve privzemimo, da je $H$ odprta v $G$. Ker je leva translacija homeomorfizem (trditev \ref{trd:trans}), je množica $aH$ odprta množica za vsak element $a \in G$ in zato tudi $\varphi^{-1}(\lbrace aH \rbrace) = aH$ za vsak element $aH \in G/H$. Po izreku \ref{izr:topkvocienta} je $\varphi$ odprta preslikava, zato je vsaka točka v kvocientnem prostoru $G/H$ odprta kot enoelementna množica in $G/H$ je diskreten topološki prostor.

Obratno, če je $G/H$ diskreten topološki prostor, potem je vsaka njegova točka odprta kot enoelementna množica, torej tudi $\lbrace H \rbrace$. Ker je naravna preslikava zvezna, je $\varphi^{-1}(\lbrace H \rbrace) = H$ odprta množica v $G$.

Za dokaz druge trditve privzemimo, da je $H$ zaprta v $G$. Ker je leva translacija homeomorfizem (trditev \ref{trd:trans}), je zaprta tudi množica $aH$ za vsak element $a \in G$. Po definiciji zaprtosti je $\setcomp{(aH)} = \bigcup \lbrace xH ; xH \neq aH \rbrace$ odprta v $G$. Ker je po izreku \ref{izr:topkvocienta} naravna preslikava odprta, je zato komplement vsake točke $\lbrace aH \rbrace$ odprt v $G/H$. Po definiciji zaprtosti je vsaka točka $\lbrace aH \rbrace$ zaprta v $G/H$, kar je ekvivalentno separacijskemu aksiomu $T_1$. Naj bosta $U$ in $V$ okolici enote $e$ iz trditve \ref{trd:okolicevkvoc}. Če za $V$ vzamemo simetrično okolico (to lahko naredimo po trditvi \ref{trd:sim}), potem po trditvi \ref{trd:okolice} tak $V$ obstaja za vsako okolico $U$, saj lahko vzamemo $V^{-1}V = V^2 \subset U$. Torej za vsako okolico $\varphi(U)$ enote $H$ obstaja takšna okolica $\varphi(V)$ enote $H$, da $\closure{\varphi(V)} \subset \varphi(U)$. Ker je leva translacija homeomorfizem, to velja za vsako točko $aH \in G/H$, kar pa je ekvivalentno separacijskemu aksiomu $T_3$. Kvocientni prostor $G/H$ je res regularen.

Za dokaz tretje trditve privzemimo, da $G/H$ zadošča separacijskemu aksiomu $T_0$. Po izreku \ref{izr:t3} je $G/H$ regularna. Vse enoelementne množice v $G/H$ so zaprte, zato tudi $\lbrace H \rbrace$. Ker je naravna preslikava zvezna, je množica $\varphi^{-1}(\lbrace H \rbrace) = H$ zaprta v $G$.
\end{dokaz}

\begin{izrek}\label{izr:kvocpovzetek}
Naj bo $H$ podgrupa edinka topološke grupe $G$. Naj bo kvocient $G/H$ opremljen s kvocientno topologijo $\theta$. Veljajo naslednje trditve:
\begin{enumerate}
\item kvocient $G/H$ je topološka grupa s topologijo $\theta$,
\item naravni homomorfizem je odprta in zvezena preslikava,
\item kvocient $G/H$ je diskreten natanko tedaj, ko je podgrupa $H$ odprta v $G$,
\item kvocient $G/H$ zadošča separacijskemu aksiomu $T_0$ natanko tedaj, ko je podgrupa $H$ zaprta v $G$.
\end{enumerate}
\end{izrek}

\section{Izreki o izomorfizmih}

\begin{trditev}\label{trd:homogenkvoc}
	Naj bo $G$ topološka grupa in $H$ njena podgrupa. Naj bo za vsak element $a \in G$ na kvocientu $G/H$ definirana preslikava $\psi_a$ s predpisom $\psi_a(xH) = (ax)H$.
	Za vsak element $a \in G$ je $\psi_a$ homeomorfizem na prostoru $G/H$.
\end{trditev}

\begin{trditev}\label{trd:kvockompakt}
	Naj bo $H$ podgrupa (lokalno) kompaktne topološke grupe $G$. Potem je tudi kvocientni prostor $G/H$ (lokalno) kompakten.
\end{trditev}

\subsection{Prvi izrek o izomorfizmih}
\begin{izrek}[Prvi izrek o izomorfizmih za topološke grupe]\label{izr:prvitopizrek}
Naj bosta $G$ in $\widetilde{G}$ topološki grupi. Naj bo $f: G \to \widetilde{G}$ odprt, zvezen in surjektiven homomorfizem. Potem je ker$f$ podgrupa edinka v grupi $G$ in množice $f^{-1}(\tilde{x})$, kjer je $\tilde{x} \in \widetilde{G}$, so disjunktni odseki ker$f$ v grupi $G$. Preslikava $\Phi:\widetilde{G} \to G/\text{ker}f$ s predpisom $\tilde{x} \mapsto f^{-1}(\tilde{x})$ je topološki izomorfizem.
\end{izrek}

\subsection{Drugi izrek o izomorfizmih}
\begin{izrek}[Drugi izrek o izomorfizmih za topološke grupe]\label{izr:drugitopizrek}
Naj bo $G$ topološka grupa, $A$ njena podgrupa in $H$ podgrupa edinka grupe $G$. Naj bo $\tau$ izomorfizem iz kvocienta $(AH)/H$ v kvocient $A/(A \cap H)$ s predpisom $\tau (aH) = a(A \cap H)$, kjer je $a \in A$.
\begin{enumerate}
\item Preslikava $\tau$ slika odprte množice iz $(AH)/H$ v odprte množice iz $A/(A \cap H)$.
\item Če je $A$ še lokalno kompaktna in $\sigma$-kompaktna, $H$ zaprta v $G$ in $AH$ lokalno kompaktna, potem je $\tau$ homeomorfizem ter topološki grupi $(AH)/H$ in $A/(A \cap H)$ sta topološko izomorfni.
\end{enumerate}
\end{izrek}


\subsection{Tretji izrek o izomorfizmih}
\begin{izrek}\label{izr:predtretji}
	Naj bo $f\colon G \to \widetilde{G}$ odprt, zvezen homomorfizem topoloških grup in naj bo $\widetilde{H}$ podgrupa edinka v $\widetilde{G}$. Potem so grupe $(G/\text{ker}f)/(f^{-1}(\widetilde{H})/\text{ker}f)$, $G/f^{-1}(\widetilde{H})$ in $\widetilde{G}/\widetilde{H}$ topološko izomorfne.
\end{izrek}

Izrek lahko preoblikujemo v obliko, ki je bolj podobna algebraični različici in ne vsebuje pomožne topološke grupe $\widetilde{G}$.
\begin{izrek}[Tretji izrek o izomorfizmih za topološke grupe]\label{izr:tretjitopizrek}
Naj bo $G$ topološka grupa in $N \subseteq H$ njeni podgrupi edinki. Potem sta kvocientni topološki grupi $G/H$ in $(G/N)/(H/N)$ topološko izomorfni.
\end{izrek}

\section{Izreki tipa ``2 od 3''}

\section{Metrizabilnost in povsem regularnost}

\subsection{Uniformni prostori}

\begin{definicija}\label{def:uniform}
	Naj bo $X$ neprazna množica.
	\begin{enumerate}
		\item Neprazna poddružina $\mathcal{F} \subset \mathcal{P}(X)$ je \emph{filter} množice $X$, če ima naslednje lastnosti:
		\begin{enumerate}
			\item družina $\mathcal{F}$ ne vsebuje prazne množice,
			\item za vsako množico $F \in \mathcal{F}$ je vsaka množica $E \in X$, za katero velja $F \subseteq E$, tudi v družini $\mathcal{F}$,
			\item presek $E \cap F$ množic $E, F \in \mathcal{F}$ je tudi v družini $\mathcal{F}$.
		\end{enumerate}
		\item Filter $\mathcal{U}$ na množici $X \times X$ definira \emph{uniformno strukturo} na množici $X$, če ima naslednje lastnosti:
		\begin{enumerate}
			\item vsaka množica $U \in \mathcal{U}$ vsebuje diagonalo $\Delta = \lbrace (x, x) ; x \in X \rbrace$,
			\item za vsako množico $U \in \mathcal{U}$ je tudi množica $U^{-1} \in \mathcal{U}$,
			\item za vsako množico $U \in \mathcal{U}$ obstaja taka množica $V \in \mathcal{U}$, da velja $V \circ V \subseteq U$.
		\end{enumerate}
		Množici z uniformno stukturo rečemo \emph{uniformni prostor}.
	\end{enumerate}
\end{definicija}

\begin{opomba}
	V zgornji definiciji so operacije na množicah mišljene v smislu relacij (glej podrazdelek \ref{sec:opnamnozicah}).
\end{opomba}

\begin{definicija}\label{def:uniinduciranatopo}
	Naj bo $X$ uniformni prostor z uniformno strukturo $\mathcal{U}$. Topologija, inducirana z $\mathcal{U}$ je taka družina $\tau$ množic $T \subseteq X$, za katere za vsako točko $x \in T$ obstaja $U \in \mathcal{U}$, da velja $U_x = \lbrace y \in X ; (x, y) \in U \rbrace \subseteq T$.
\end{definicija}

\begin{opomba}
	V nadaljevanju bomo okolico nekega elementa $x$ v topologiji, inducirani z uniformno strukturo $\mathcal{U}$, označevali kot $U_x$, kjer bo $U \in \mathcal{U}$.
\end{opomba}

\begin{definicija}\label{def:enakzveznost}
	Naj bosta $X$ in $Y$ uniformna prostora z uniformnima strukturama $\mathcal{U}$ in $\mathcal{V}$. Preslikava $f: X \to Y$ je \emph{enakomerno zvezna}, če za vsako množico $V \in \mathcal{V}$ obstaja taka množica $U \in \mathcal{U}$, da za vsak par $(x, y) \in U$ velja $(f(x), f(y)) \in V$.
\end{definicija}

\begin{trditev}\label{trd:enakzveznazvezna}
	Vsaka enakomerno zvezna preslikava uniformnih prostorov je zvezna v topologiji, inducirani z uniformnima strukturama.
\end{trditev}

\begin{dokaz}
	Naj bo $f\colon (X, \mathcal{U}) \to (Y, \mathcal{V})$ enakomerno zvezna preslikava med uniformnima prostoroma. Vzemimo okolico $V_{f(x)}$ elementa $f(x)$ v topologiji na $Y$ inducirani s $\mathcal{V}$. Ker je $f$ enakomerno zvezna, obstaja $U \in \mathcal{U}$, da za vsak par $(x, y) \in U$ velja $(f(x), f(y)) \in V$. To je v induciranih topologijah ekvivalentno temu, da za vsak $y \in U_x$ velja $f(y) \in V_{f(x)}$, torej je $f$ zvezna preslikava glede na topologiji, ki ju inducirata uniformni strukturi.
\end{dokaz}

\begin{definicija}\label{def:levadesnauni}
	Naj bo $\Ucurl$ baza odprtih okolic enote $e$ topološke grupe $G$. Za vsako okolico $U \in \Ucurl$ definiramo $L_U = \lbrace (x, y) \in G \times G ; x^{-1}y \in U \rbrace$ in $R_U = \lbrace (x, y) \in G \times G ; yx^{-1} \in U \rbrace$. Družinama $\mathcal{L}(G) = \lbrace L_U \rbrace_{U \in \Ucurl}$ in $\mathcal{R}(G) = \lbrace R_U \rbrace_{U \in \Ucurl}$ pravimo leva in desna uniformna struktura na $G$.
\end{definicija}

\begin{trditev}\label{trd:topguniform}
	Vsaka topološka grupa je uniformni prostor.
\end{trditev}

\begin{dokaz}
Naj bo filter $\Ucurl$ baza odprtih in simetričnih okolic enote $e$ topološke grupe $G$. Oglejmo si levo in desno uniformno strukturo na $G$, definirani naj bosta s pomočjo $\Ucurl$.

Ker je enote $e$ v vsaki okolici $U \in \Ucurl$, je diagonala $\Delta = \lbrace (x, x) ; x \in X \rbrace$ vsebovana v $L_U$ in $R_U$ za vsako okolico $U \in \Ucurl$.

Velja \[L_{U^{-1}} = \lbrace (x, y) \in G \times G ; x^{-1}y \in U^{-1} \rbrace = \lbrace (x, y) \in G \times G ; y^{-1}x \in U \rbrace = L^{-1}_U\]
in
\[R_{U^{-1}} = \lbrace (x, y) \in G \times G ; yx^{-1} \in U^{-1} \rbrace = \lbrace (x, y) \in G \times G ; xy^{-1} \in U \rbrace = R^{-1}_U.\]
Ker za vsako okolico $U \in \Ucurl$ velja $U = U^{-1}$, je $L^{-1}_U \in \mathcal{L}(G)$ in $R^{-1}_U \in \mathcal{R}(G)$.

Po trditvi \ref{trd:okolice} za vsako okolico $U \in \Ucurl$ obstaja okolica $V \in \Ucurl$, da velja $V^2 \subset U$. Velja torej, da je $L_V \circ L_V \subset L_U$ in $R_V \circ R_V \subset R_U$.

Topološka grupa je z levo ali desno uniformno strukturo torej res uniformni prostor.
\end{dokaz}

\subsection{Metrizabilnost}

\begin{definicija}\label{def:metrika}
	\emph{Psevdometrika} na neprazni množici $X$ je preslikava $\rho\colon X \times X \to  [0, \infty)$, ki zadošča naslednjim pogojem:
	\begin{enumerate}
		\item za vsako točko $x \in X$ velja $\rho (x, x) = 0$;
		\item za vsaki dve točki $x, y \in X$ velja $\rho (x, y) = \rho (y, x)$;
		\item za vsake tri točke $x, y, z \in X$ velja $\rho (x, z) \leq \rho (x, y) + \rho (y, z)$.
	\end{enumerate}
	Če za preslikavo $\rho$ velja še
	\begin{enumerate}[resume]
		\item $\rho(x,y) = 0$ natanko tedaj, ko $x = y$,
	\end{enumerate}
	potem ji rečemo \emph{metrika}.
\end{definicija}

\begin{definicija}
Psevdometrika na grupi $G$ je levoinvariantna, če je invariantna na levo translacijo, torej če za vsaki dve točki $x, y \in G$ in za vsak element $a \in G$ velja $\rho(ax, ay) = \rho(x, y)$.

Podobno, psevdometrika je desnoinvariantna, če je invariantna na desno translacijo.
\end{definicija}

\begin{izrek}\label{izr:pseudometrika}
	Naj bo $\lbrace U_k \rbrace_{k = 1}^{\infty}$ tako zaporedje simetričnih okolic enote $e$ v topološki grupi $G$, da za vsak $k \in \N$ velja $U_{k+1}^2 \subset U_k$. Označimo $H = \bigcap_{k=1}^{\infty} U_k$. Potem obstaja taka levoinvariantna psevdometrika $\sigma$ na $G$ z naslednjimi lastnostmi:
	\begin{enumerate}
		\item $\sigma$ je enakomerno zvezna na levi uniformni strukturi od $G \times G$;
		\item $\sigma (x, y) = 0$ natanko tedaj, ko $y^{-1}x \in H$;
		\item $\sigma (x, y) \leq 2^{-k+2}$, če $y^{-1}x \in U_k$;
		\item $2^{-k} \leq \sigma (x, y)$, če $y^{-1}x \notin U_k$.
	\end{enumerate}
	
	Če velja še $x U_k x^{-1} = U_k$ za vsak $x \in G$ in $k \in \N$, potem je $\sigma$ tudi desnoinvariantna in velja
	\begin{enumerate}[resume]
		\item $\sigma (x^{-1}, y^{-1}) = \sigma (x, y)$ za vsaka dva elementa $x, y \in G$.
	\end{enumerate}
\end{izrek}

\begin{definicija}\label{def:metrizabilnost}
Topološki prostor $X$ je \emph{metrizabilen}, če njegova topologija $\tau$ izhaja iz kakšne metrike $d$ na množici $X$.
\end{definicija}

\begin{opomba}
Baza topologije metrizabilnega topološkega prostora $X$ je družina odprtih krogel $\lbrace K(x, \epsilon); x \in X, \epsilon \in \R \rbrace$.
\end{opomba}

\begin{izrek}\label{izr:metrizabilnost}
	Topološka grupa $G$, ki zadošča separacijskemu aksiomu $T_0$, je metrizabilen topološki prostor natanko tedaj, ko obstaja števna baza odprtih okolic enote.
\end{izrek}

\begin{dokaz}
Če je $G$ metrizabilen topološki prostor, lahko za števno bazo odprtih okolic enote $e$ izberemo kar družino odprtih krogel s središčem v enoti $\lbrace K(e, 2^{-n}) \rbrace_{n \in \N}$.

Naj bo $\lbrace V_k \rbrace_{k = 1}^\infty$ števna baza odprtih okolic enote. Induktivno definiramo novo bazo okolic enote na sledeč način. Najprej definiramo okolico $U_1 = V_1 \cap V_1^{-1}$, nato pa vsako naslednjo tako, da zadošča $U_k \subset U_1 \cap \dots \cap U_{k-1}\cap V_k$, $U_k = U_k^{-1}$ in $U_k^2 \subset U_{k-1}$ za vsak $k \geq 2$. Tako zaporednje si lahko izberemo po trditvi \ref{trd:okolice}. Ker $G$ zadošča separacijskemu aksiomu $T_0$ in je po izreku \ref{izr:t3} regularna, je $H = \bigcap_{k=1}^\infty = \lbrace e \rbrace$. Baza $\lbrace U_k \rbrace_{k = 1}^\infty$ zadošča predpostavkam izreka \ref{izr:pseudometrika}, zato na topološki grupi $G$ obstaja psevdometrika $\sigma$.

Po drugi lastnosti psevdometrike $\sigma$ je $\sigma(x, y) = 0$ natansko tedaj, ko $y^{-1}x \in H$. Ker $H = \lbrace e \rbrace$, velja $\sigma(x, y) = 0$ natanko tedaj, ko $x = y$. Preslikava $\sigma$ je torej metrika na $G$. Preveriti moramo le še, da topologija $\tau$ na $G$ in topologija $\tau_\sigma$, inducirana z metriko $\sigma$, sovpadata.

Po tretji in četrti lastnosti metrike $\sigma$ za vsak $k \in \N$ velja
\[ \lbrace x \in G ; \sigma(x, e) \leq 2^{-k} \rbrace \subset U_k \subset \lbrace x \in G ; \sigma(x, e) \leq 2^{-k+1} \rbrace.\]
Torej za vsak $k \in \N$ velja
\[ K(e, 2^{-k}) \subset U_k \subset K(e, 2^{-k+2}). \]
Vsaka okolica enote $e$ v topologiji $\tau$ torej vsebuje okolico enote $e$ v topologiji $\tau_\sigma$ in vsaka okolica enote $e$ v topologiji $\tau_\sigma$ vsebuje okolico enote $e$ v topologiji $\tau$.
Topologiji $\tau$ in $\tau_\sigma$ sta zato ekvivalentni in $G$ je metrizabilen topološki prostor.
\end{dokaz}

\subsection{Separacijski aksiom T$_{3 \frac{1}{2}}$}

\begin{definicija}
	Topološki prostor $X$ zadošča separacijskemu aksiomu $T_{3 \frac{1}{2}}$, če za poljubno zaprto množico $A \subseteq X$ in točko $b \in X\backslash A$ obstaja taka zvezna realna funkcija $\psi$, definirana na $G$, da je $\psi (b) = 0$ in $\psi (x) = 1$ za vsak $x \in A$.
\end{definicija}

\begin{opomba}
	Topološku prostoru, ki zadošča $T_1$ in $T_{3 \frac{1}{2}}$, pravimo \emph{povsem regularen} topološki prostor.
\end{opomba}


\begin{trditev}\label{pos:reghaus}
\begin{enumerate}
\item Vsak povsem regularen topološki prostor je regularen.
\item Vsak normalen topološki prostor je povsem regularen.
\end{enumerate}
\end{trditev}

\begin{dokaz}
Naj bo $F$ zaprta podmnožica povsem regularnega topološkega prostora $X$ in točka $a \in X \setminus F$. Potem obstaja zvezna funkcija $\psi\colon X \to [0, 1]$, da je $\psi(a) = 0$ in $\psi(F) \equiv 1$. Množici $\psi^{-1}([0, \frac{1}{2}))$ in $\psi^{-1}((\frac{1}{2}, 1])$ sta disjunktni odprti okolici za točko $a$ in množico $F$, saj je funkcija $\psi$ zvezna, množici $[0, \frac{1}{2})$ in $(\frac{1}{2}, 1]$ pa sta odprti v inducirani evklidski topologiji na interval $[0, 1]$.
Izpolnjen je separacijski aksiom $T_3$ in $X$ je regularen topološki prostor.

Naj bo $X$ normalen topološki prostor. Po Urysohnovi karakterizaciji separacijskega aksioma $T_4$ (glej \cite{bib:top}) za vsaki dve disjunktni zaprti množici $A$ in $B$ obstaja zvezna funkcija $\psi\colon X \to [0, 1]$, da je $\psi(A) \equiv 0 $ in $\psi(B) \equiv 1$. Ker $X$ zadošča separacijskemu aksiomu $T_1$, so vse enoelementne množice zaprte. Če za množico $A$ vzamemo enoelementno množico $\lbrace a \rbrace$, vidimo, da je $X$ povsem regularen topološki prostor.
\end{dokaz}

\begin{izrek}\label{izr:t3pol}
	Topološka grupa, ki zadošča separacijskemu aksiomu $T_0$, je povsem regularen topološki prostor.
\end{izrek}

\begin{dokaz}
Vzemimo zaprto množico $F$ in element $a \in G\setminus\lbrace a \rbrace$.
Naj bo $\Ucurl$ baza simetričnih okolic enote $e$ in naj bo $U_1 \in \Ucurl$ taka množica, da je $(aU_1)\cap F = \emptyset$. Taka množica $U_1$ obstaja, saj je $G\setminus\lbrace a \rbrace$ odprta množica, $aU_1$ pa je odprta okolica elementa $a$. Izberemo okolice $U_2, U_3,... \in \Ucurl$ take, da velja $U_{k+1}^2 \subset U_k$ za vsak $k \in \Z$ (obstajajo po trditvi \ref{trd:okolice}). S tem smo zadostili predpostavkam izreka \ref{izr:pseudometrika}, zato obstaja na $G$ psevdometrika $\sigma$. Definiramo funkcijo \[ \psi(x) = \min\lbrace 1, 2\sigma(x, a)\rbrace. \] Ker je psevdometrika $\sigma$ enakomerno zvezna glede na levo uniformno strukturo na $G$, je po trditvi \ref{trd:enakzveznazvezna} zvezna, zato je $\psi$ zvezna funkcija.

Očitno velja, da je $\psi(a) = 0$, saj je $\sigma(a, a) = 0$ po definiciji psevdometrike.

Vzemimo element $x \in F$. Po konstrukciji množice $U_1$ velja $a^{-1}x \notin U_1$. Po četrti lastnosti v izreku \ref{izr:pseudometrika} je $\sigma(x, a) \geq 2^{-1} = \frac{1}{2}$. Sledi, da je $\psi(x) = 1$ za vsak element $x \in F$.

Topološka grupa $G$ s tem zadošča separacijskemu aksiomu $T_{3\frac{1}{2}}$ in je povsem regularna.
\end{dokaz}

\subsection{Separacijski aksiom T$_4$}
\begin{izrek}\label{izr:t4protiprimer}
	Če je $m$ katerokoli neštevno kardinalno število, potem je $\Z^{m}$ povsem regularna topološka grupa, ki ni normalna.
\end{izrek}

\begin{definicija}\label{def:parakompakt}
	\begin{enumerate}
		\item Naj bosta $\mathcal{U}$ in $\mathcal{V}$ družini podmnožic topološkega prostora $X$. Družina $\mathcal{V}$ je \emph{pofinitev} družine $\mathcal{U}$, če za vsako množico $V \in \mathcal{V}$ obstaja takšna množica $U \in \mathcal{U}$, da je $V \subset U$.
		\item Družina podmnožic $\mathcal{U}$ topološkega prostora $X$ je \emph{lokalno končna}, če ima vsaka točka $x \in X$ okolico, ki seka samo končno mnogo množic iz družine $\mathcal{U}$.
		\item Topološki prostor $X$ je \emph{parakompakten}, če ima vsako njegovo odprto pokritje kakšno pofinitev, ki je lokalno končno odprto pokritje prostora $X$.
	\end{enumerate}
\end{definicija}

\begin{definicija}
\begin{enumerate}
\item Topološki prostor je $\sigma$-kompakten, če ge je možno zapisati kot števno unijo kompaktnih topoloških prostorov.
\item Topološki prostor ima Lindel\"ofovo lastnost, če vsako njegovo odprto pokritje vsebuje kakšno števno podpokritje.
\end{enumerate}
\end{definicija}

\begin{opomba}
Očitno je, da ima vsak kompakten prostor Lindel\"ofovo lastnost, saj vsako odprto pokritje vsebuje končno podpokritje, ki je trivialno števno. Velja pa tudi, da ima vsak $\sigma$-kompakten prostor Lindel\"ofovo lastnost, saj je števna unija končnih pokritij števno pokritje.
\end{opomba}

\begin{trditev}\label{trd:parkompnorm} % Mrčun, Topologija, Trditev 4.21
Vsak parakompakten Hausdorffov topološki prostor je normalen.
\end{trditev}

\begin{dokaz}
Vzemimo $A$ in $B$ zaprti disjunktni podmnožici parakompaktnega Hausdorffovega prostora $X$. Zanju iščemo disjunktni okolici.

Predpostavimo najprej, da je $B = \lbrace b \rbrace$ enoelementna množica za neko točko $X\setminus A$. Ker je prostor $X$ Hausdorffov, sta vsaki dve točki ločeni z disjunktnima okolicama. Za vsako točko $a \in A$ torej obstaja odprta množica $Q_a \subset X$, da je $a \in Q_a$ in $b \in X \setminus \closure{Q_a}$. Ker je $A$ zaprta, je $X \setminus A$ odprta in velja, da je $\mathcal{W} = (X \setminus A)\cup \lbrace Q_a ; a \in A \rbrace$ odprto pokritje prostora $X$. Ker je $X$ parakompakten topološki prostor, obstaja lokalno končno odprto pokritje $\mathcal{W}'$ prostora $X$, ki je pofinitev pokritja $\mathcal{W}$.

Oglejmo si družino \[ \mathcal{Q} = \lbrace W \in \mathcal{W}' ; W \cap A \neq \emptyset \rbrace. \]
Za vsako množico $W \in \mathcal{Q}$ po definiciji pofinitve obstaja takšna točka $a \in A$, da je $W \subset Q_a$. Velja $b \in X \setminus \closure{Q_a} \subset X \setminus \closure{W}$ za vsak $W \in \mathcal{Q}$ (in primeren $a$). Ker je $\mathcal{W}'$ odprto pokritje prostora $X$, je $S = \cup\mathcal{Q}$ odprta okolica množice $A$ in ker je $\mathcal{Q}$ lokalno končna družina, velja \[ b \in X \setminus \bigcup_{W \in \mathcal{Q}} \closure{W} = X \setminus \closure{S}. \] Množica $T = X \setminus \closure{S}$ je odprta okolica točke $b$, ki je disjunktna s $S$.

Naprej vzemimo poljubno zaprto podmnožico $B$, ki je disjunktna z množico $A$. Po prejšnjem argumentu za vsako točko $b \in B$ obstaja odprta okolica $T_b$ točke $b$, da je $A \cap \closure{T_b} = \emptyset$. Ponovno je $\Ucurl = (X\setminus B) \cup \lbrace T_b ; b \in B \rbrace$ odprto pokritje prostora $X$. Ker je $X$ parakompakten topološki prostor, obstaja lokalno končno odprto pokritje $\Ucurl'$, ki je pofinitev pokritja $\Ucurl$. Naj bo \[ \mathcal{V} = \lbrace U \in \Ucurl' ; U \cap B \neq \emptyset \rbrace. \]
Podobno kot prej za vsako množico $U \in \mathcal{V}$ obstaja takšna točka $b \in B$, da je $U \subset T_b$, zato je $A \cap \closure{U} \subset A \cap \closure{T_b} = \emptyset$. Množica $V = \cup \mathcal{V}$ je odprta okolica množice $B$ in, ker je $\mathcal{V}$ lokalno končna družinca, velja \[ A \cap \closure{V} = A \cap \bigcup_{U \in \mathcal{V}} = \emptyset.\]

Množica $X \setminus \closure{V}$ je torej odprta okolica množice $A$, ki je disjunktna z odprto okolico $V$ množice $B$. Hausdorffov topološki prostor $X$ zadošča separacijskemu aksiomu $T_4$ in je zato normalen.
\end{dokaz}

\begin{izrek}\label{izr:t4}
	Vsaka lokalno kompaktna topološka grupa, ki zadošča separacijskemu aksiomu $T_0$, je normalen topološki prostor.
\end{izrek}

\begin{dokaz}
Naj bo $G$ lokalno kompaktna topološka grupa, ki zadošča separacijskemu aksiomu $T_0$. Po izreku \ref{trd:t0haus} je $G$ Hausdorffova. Zato po trditvi \ref{trd:parkompnorm} zadošča pokazati, da je $G$ parakompaktna topološka grupa.

Ker je $G$ lokalno kompaktna, obstaja kompaktna okolica $K$ enote $e$. Po posledici \ref{pos:sim} obstaja takšna simetrična okolica $U$ enote $e$, da je $\closure{U} \subset K$. Ker so zaprte podmnožice kompaktnih prostorov kompaktne, je $\closure{U}$ tudi kompaktna okolica enote $e$.

Naj bo $L = \bigcup_{n=1}^\infty U^n$. Množica $L$ je po trditvi \ref{trd:podgrupaunija} odprta in zaprta podgrupa topološke grupe $G$. HERE
\end{dokaz}

\section*{Slovar strokovnih izrazov}

\geslo{}{}
\geslo{}{}

% seznam uporabljene literature
\begin{thebibliography}{99}

\bibitem{bib:uniform}
S.~Bhowmik, \emph{Introduction to Uniform Spaces}, 10.13140/RG.2.1.3743.8967, junij 2014, [ogled 1.~4.~2019], dostopno na \url{https://www.researchgate.net/publication/305196408_INTRODUCTION_TO_UNIFORM_SPACES}.
\bibitem{bib:aha1}
E.~Hewitt in K.~A.~Ross, \emph{Abstact Harmonic Analysis I}, Springer-Verlag, New York, 1979.
\bibitem{bib:top}
J.~Mrčun, \emph{Topologija}, Izbrana poglavja iz matematike in računalništva \textbf{44} DMFA-založništvo, Ljubljana, 2008.

\end{thebibliography}

\end{document}

